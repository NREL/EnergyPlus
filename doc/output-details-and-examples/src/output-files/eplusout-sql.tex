\section{eplusout.sql}

eplusout.sql is an optional output format for EnergyPlus. The eplusout.sql output file is a sqlite3 database file (see \url{http://www.sqlite.org}) and includes all of the data found in EnergyPlus’ eplustbl.* files, eplusout.eso and eplusout.mtr output files (i.e., EnergyPlus’ standard variable and meter output files) plus a number of reports that are found in the eplusout.eio output file.

A discussion of the individual data tables is presented below followed by a discussion about how to access data within the SQL file.

\subsection{List of Available SQLite Tables}

This initial release of the SQL database output option includes a variety of data in the following tables:

\begin{itemize}
  \item ComponentSizes Table
  \item ConstructionLayers Table
  \item Constructions Table
  \item Materials Table
  \item NominalBaseboardHeaters Table
  \item NominalElectricEquipment Table
  \item NominalGasEquipment Table
  \item NominalHotWaterEquipment Table
  \item NominalInfiltration Table
  \item NominalLighting Table
  \item NominalOtherEquipment Table
  \item NominalPeople Table
  \item NominalSteamEquipment Table
  \item NominalVentilation Table
  \item ReportData Table
  \item ReportDataDictionary Table
  \item ReportExtendedData Table
  \item RoomAirModels Table
  \item Schedules Table
  \item Surfaces Table
  \item SystemSizes Table
  \item Time Table
  \item ZoneGroups Table
  \item Zones Table
  \item ZoneLists Table
  \item ZoneSizes Table
  \item ZoneInfoLists Table
  \item Simulations Table
  \item EnvironmentPeriods Table
  \item TabularData Table
  \item Strings Table
  \item StringTypes Table
  \item TabularDataWithStrings Table
  \item Errors Table
\end{itemize}

A short discussion of contents of each of the above SQL tables is given in the sections that follow.

\subsection{Report Data}

Data in the following four tables are also found in EnergyPlus’ standard output file (i.e., eplusout.eso – see that section for more information) or in EnergyPlus' eplusout.mtr (i.e. meter) output file. As with the standard output file, the “Report Variable” and “Report Meter” commands control the data in these tables.

\subsubsection{ReportDataDictionary Table}

The ReportDataDictionary table provides the equivalent of the dictionary portion of the ESO file (i.e., the first section of the .eso file) or eplusout.mtr. Please see the Report Variable section of the Input-Output Reference for more information.  Like in the eplusout.mtr file (see the eplusout.mtr section), only data requested by "Report Meter" commands will set isMeter to 1 or True.

{\scriptsize
\begin{longtable}[c]{>{\raggedright}p{1.5in}>{\raggedright}p{1.5in}>{\raggedright}p{1.5in}>{\raggedright}p{1.5in}}
\caption{SQL ReportDataDictionary Contents \label{table:table-7.-sql-reportdatadictionary-contents}} \tabularnewline
\toprule 
Field Name & Field Type & Description & Enumeration \tabularnewline
\midrule
\endfirsthead

\caption[]{SQL ReportDataDictionary Contents} \tabularnewline
\toprule 
Field Name & Field Type & Description & Enumeration \tabularnewline
\midrule
\endhead

Report\-Data\-Dictionary\-Index & INTEGER PRIMARY KEY & The ReportDataDictionaryIndex links the dictionary data to the variable data (see ReportData table) & ~ \tabularnewline
IsMeter & INTEGER & Boolean flag whether dictionary data is report of meter data & 0, 1 \tabularnewline
Type & TEXT & Nature of data type with respect to state, (e.g. average or sum type of data). & “Sum” “Avg” \tabularnewline
IndexGroup & TEXT & The reporting group (e.g., Zone, Plant, etc.) & ~ \tabularnewline
TimestepType & TEXT & Type of timestep for data & “Zone”
    “HVAC System” \tabularnewline
KeyValue & TEXT & The identifying “key name” for the data & ~ \tabularnewline
Name & TEXT & The actual report data name & ~ \tabularnewline
ReportingFrequency & TEXT & The reporting frequency of the variable, e.g. & “HVAC System Timestep”,~ “Zone Timestep”,~ “Hourly”, “Daily”, “Monthly”, “Run Period”. \tabularnewline
ScheduleName & TEXT & The name of the schedule that controls reporting frequency & ~ \tabularnewline
Units & TEXT & The data units & ~ \tabularnewline
\bottomrule
\end{longtable}}

\subsubsection{ReportData Table}

The ReportData table contains the report or meter variable data (e.g., the hourly, daily, and monthly report variable data). Please see the Report Variable section of the Input-Output Reference for more information.

{\scriptsize
\begin{longtable}[c]{>{\raggedright}p{1.5in}>{\raggedright}p{1.5in}>{\raggedright}p{3.0in}}
\caption{SQL ReportData Table Contents \label{table:table-8.-sql-reportdata-table-contents}} \tabularnewline
\toprule 
Field Name & Field Type & Description \tabularnewline
\midrule
\endfirsthead

\caption[]{SQL ReportData Table Contents} \tabularnewline
\toprule 
Field Name & Field Type & Description \tabularnewline
\midrule
\endhead

ReportDataIndex & INTEGER PRIMARY KEY &  \tabularnewline
TimeIndex & INTEGER & This index links the record to its time record (see the Time table below) \tabularnewline
ReportData\-Dictionary\-Index & INTEGER & The ReportDataDictionaryIndex links the data to the respective data dictionary record (see ReportDataDictionary table above) \tabularnewline
Value & REAL & The data value \tabularnewline
\bottomrule
\end{longtable}}

\subsubsection{ReportExtendedData Table}

The ReportExtendedData table contains additional data (e.g., monthly maximums and minimums) that is available for certain report or meter variables.

{\scriptsize
\begin{longtable}[c]{>{\raggedright}p{1.5in}>{\raggedright}p{1.5in}>{\raggedright}p{3.0in}}
\caption{SQL ReportExtendedData Table Contents \label{table:table-9.-sql-reportextendeddata-table}} \tabularnewline
\toprule 
Field Name & Field Type & Description \tabularnewline
\midrule
\endfirsthead

\caption[]{SQL ReportExtendedData Table Contents} \tabularnewline
\toprule 
Field Name & Field Type & Description \tabularnewline
\midrule
\endhead

ReportExtended\-Data\-Index & INTEGER PRIMARY KEY &  \tabularnewline
ReportDataIndex & INTEGER & Associates ReportData record with ReportExtendedData record \tabularnewline
MaxValue & REAL & The maximum value during the reporting interval \tabularnewline
MaxMonth & INTEGER & The month in which the maximum value occurred \tabularnewline
MaxDay & INTEGER & The day on which the maximum value occurred \tabularnewline
MaxHour & INTEGER & The hour in which the maximum value occurred \tabularnewline
MaxStartMinute & INTEGER & The starting minute of the interval in which the maximum value occurred \tabularnewline
MaxMinute & INTEGER & The minute that the maximum value occurred \tabularnewline
MinValue & REAL & The minimum value \tabularnewline
MinMonth & INTEGER & The month in which the minimum value occurred \tabularnewline
MinDay & INTEGER & The day on which the minimum value occurred \tabularnewline
MinHour & INTEGER & The hour in which the minimum value occurred \tabularnewline
MinStartMinute & INTEGER & The starting minute of the interval in which the minimum value occurred \tabularnewline
MinMinute & INTEGER & The minute that the minimum value occurred \tabularnewline
\bottomrule
\end{longtable}}

\subsubsection{Time Table}

The Time table provides the time information for both the “report variable” and “report meter” variables (the ReportData table).

\begin{longtable}[c]{>{\raggedright}p{1.5in}>{\raggedright}p{1.5in}>{\raggedright}p{3.0in}}
\caption{Time Table Contents \label{table:table-10.-time-table-contents}} \tabularnewline
\toprule 
Field Name & Field Type & Description \tabularnewline
\midrule
\endfirsthead

\caption[]{Time Table Contents} \tabularnewline
\toprule 
Field Name & Field Type & Description \tabularnewline
\midrule
\endhead

TimeIndex & INTEGER PRIMARY KEY & Connects the time information with the report variables and meter variables (see the ReportData table) \tabularnewline
Month & INTEGER & Month \tabularnewline
Day & INTEGER & Day \tabularnewline
Hour & INTEGER & Hour \tabularnewline
Minute & INTEGER & Minute \tabularnewline
DST & INTEGER & Daylight saving time indicator \tabularnewline
Interval & INTEGER & Length of the reporting interval \tabularnewline
IntervalType & INTEGER & The index for the type of reporting interval. (enum: -1=detailed HVAC system timestep; 0=zone timestep, 1=hourly, 2=daily, 3=monthly, 4=runperiod) \tabularnewline
SimulationDays & INTEGER & Day of simulation.~ This number resets after warmup and at the beginning of an environment period \tabularnewline
DayType & TEXT & The type of day. (enum: “Sunday”, “Monday”, “Tuesday”, Wednesday”, “Thursday”, “Friday”, “Saturday”, “Holiday”, “SummerDesignDay”, “WinterDesignDay”, “CustomDay1”, “CustomDay2”) \tabularnewline
Environment\-Period\-Index & INTEGER & Foreign Key to the EnvironmentPeriods table \tabularnewline
WarmupFlag & INTEGER & 1 during warmup, 0 otherwise \tabularnewline
\bottomrule
\end{longtable}

\subsection{One time (EIO) File Data}

Data in the tables below can also be found in EnergyPlus input output file (i.e., in the eplusout.eio output file).

\subsubsection{Zones Table}

The Zones table provides a variety of information about the zones specified within EnergyPlus. One of its most common uses is to provide zone name and area information for the other tables within the SQL database (e.g., use the ZoneIndex to look up the ZoneName).

\begin{longtable}[c]{>{\raggedright}p{1.5in}>{\raggedright}p{1.5in}>{\raggedright}p{3.0in}}
\caption{SQL Zones Table Contents \label{table:table-14.-sql-zones-table-contents}} \tabularnewline
\toprule 
Field Name & Field Type & Description \tabularnewline
\midrule
\endfirsthead

\caption[]{SQL Zones Table Contents} \tabularnewline
\toprule 
Field Name & Field Type & Description \tabularnewline
\midrule
\endhead

ZoneIndex & INTEGER PRIMARY KEY & The ZoneIndex is used to link this table to related tables \tabularnewline
ZoneName & TEXT & Zone Name \tabularnewline
RelNorth & REAL & Relative North, in degrees \tabularnewline
OriginX & REAL & X origin, in meters \tabularnewline
OriginY & REAL & Y origin, in meters \tabularnewline
OriginZ & REAL & Z origin, in meters \tabularnewline
CentroidX & REAL & X position of zone Centroid, in meters \tabularnewline
CentroidY & REAL & Y position of zone Centroid, in meters \tabularnewline
CentroidZ & REAL & Z position of zone Centroid, in meters \tabularnewline
OfType & INTEGER & (not used yet) \tabularnewline
Multiplier & REAL & Zone multiplier \tabularnewline
ListMultiplier & REAL & Zone Group multiplier \tabularnewline
MinimumX & REAL & Minimum X value, in meters \tabularnewline
MaximumX & REAL & Maximum X value, in meters \tabularnewline
MinimumY & REAL & Minimum Y value, in meters \tabularnewline
MaximumY & REAL & Maximum Y value, in meters \tabularnewline
MinimumZ & REAL & Minimum Z value, in meters \tabularnewline
MaximumZ & REAL & Maximum Z value, in meters \tabularnewline
CeilingHeight & REAL & Ceiling height, in meters \tabularnewline
Volume & REAL & Zone volume, in cubic meters \tabularnewline
Inside\-Convection\-Algo & INTEGER & Inside convection algorithm (enum: 1=simple, 2=detailed, 3=Ceiling diffuser, 4=TrombeWall) \tabularnewline
Outside\-Convection\-Algo & INTEGER & Outside convection algorithm (enum: 1=simple, 2=detailed, 6=MoWitt, 7=DOE-2) \tabularnewline
FloorArea & REAL & Zone floor area, in square meters \tabularnewline
ExtGrossWall\-Area & REAL & Zone external gross wall area (includes windows and doors), in square meters \tabularnewline
ExtNetWallArea & REAL & Zone net wall area (excludes windows and doors), and square meters \tabularnewline
ExtWindowArea & REAL & Zone window area (includes glass doors), and square meters \tabularnewline
IsPartOfTotalArea & INTEGER & See Zone input object documentation \tabularnewline
\bottomrule
\end{longtable}

Please see the Zone object in the Group-Thermal Zone Description/Geometry section of the Input-Output Reference for more information.

\subsubsection{NominalPeople Table}

An overview of the NominalPeople SQL table is shown below.

{\scriptsize
\begin{longtable}[c]{>{\raggedright}p{1.5in}>{\raggedright}p{1.5in}>{\raggedright}p{3.0in}}
\caption{SQL NominalPeople Table Contents \label{table:table-15.-sql-nominalpeople-table-contents}} \tabularnewline
\toprule 
Field Name & Field Type & Description \tabularnewline
\midrule
\endfirsthead

\caption[]{SQL NominalPeople Table Contents} \tabularnewline
\toprule 
Field Name & Field Type & Description \tabularnewline
\midrule
\endhead

NominalPeopleIndex & INTEGER & The internal statement number \tabularnewline
ObjectName & TEXT & The name of the People object \tabularnewline
ZoneIndex & INTEGER & Connects the NominalPeople table to the Zones table \tabularnewline
NumberOfPeople & INTEGER & Nominal number of people in the zone \tabularnewline
NumberOfPeople\-Schedule\-Index & INTEGER & Number of people schedule number (see Schedule table) \tabularnewline
ActivitySchedule\-Index & INTEGER & Activity level schedule (see People object documentation) \tabularnewline
FractionRadiant & REAL & see People object documentation \tabularnewline
FractionConvected & REAL & see People object documentation \tabularnewline
WorkEfficiency\-Schedule\-Index & INTEGER & Work efficiency schedule number (see schedule table and people object documentation) \tabularnewline
ClothingEfficiency\-Schedule\-Index & INTEGER & Clothing efficiency schedule number (see schedule table and people object documentation) \tabularnewline
AirVelocitySchedule\-Index & INTEGER & Air velocity schedule number (see schedule table and people object documentation) \tabularnewline
Fanger & INTEGER & Flag indicating whether Fanger calculations are active \tabularnewline
Pierce & INTEGER & Flag indicating whether Pierce calculations are active \tabularnewline
KSU & INTEGER & Flag indicating whether KSU calculations are active \tabularnewline
MRTCalcType & INTEGER & see People object documentation \tabularnewline
SurfaceIndex & INTEGER & see Surfaces table and People object documentation \tabularnewline
UserSpecifeid\-Sensible\-Fraction & REAL & see People object documentation \tabularnewline
Show55Warning & INTEGER & see People object documentation \tabularnewline
\bottomrule
\end{longtable}}

Please see the People object in the Group-Internal Gains section of the Input-Output Reference for more information.

\subsubsection{NominalLighting Table}

An overview of the NominalLighting SQL table is shown below.

\begin{longtable}[c]{>{\raggedright}p{1.5in}>{\raggedright}p{1.5in}>{\raggedright}p{3.0in}}
\caption{SQL NominalLighting Table Contents \label{table:table-16.-sql-nominallighting-table-contents}} \tabularnewline
\toprule 
Field Name & Field Type & Description \tabularnewline
\midrule
\endfirsthead

\caption[]{SQL NominalLighting Table Contents} \tabularnewline
\toprule 
Field Name & Field Type & Description \tabularnewline
\midrule
\endhead

NominalLightingIndex & INTEGER & The internal statement number \tabularnewline
ObjectName & TEXT & The LIGHTS object name \tabularnewline
ZoneIndex & INTEGER & Connects the NominalLighting table to the Zones table \tabularnewline
ScheduleIndex & INTEGER & Lighting schedule number (see Schedule table) \tabularnewline
DesignLevel & REAL & Nominal design level, in Watts \tabularnewline
FractionReturnAir & REAL & User-specified return air fraction \tabularnewline
FractionRadiant & REAL & User-specified radiant fraction \tabularnewline
FractionReplaceable & REAL & Defines the daylighting control for the LIGHTS object \tabularnewline
EndUseSubcategory & TEXT & User-specified end-use subcategory \tabularnewline
\bottomrule
\end{longtable}

Please see the LIGHTS object in the Group-Internal Gains section of the Input-Output Reference for more information.

\subsubsection{NominalElectricEquipment Table}

An overview of the NominalElectricEquipment SQL table is shown below.

{\scriptsize
\begin{longtable}[c]{>{\raggedright}p{1.5in}>{\raggedright}p{1.5in}>{\raggedright}p{2.99in}}
\caption{SQL NominalElectricEquipment Table Contents \label{table:table-17.-sql-nominalelectricequipment-table}} \tabularnewline
\toprule 
Field Name & Field Type & Description \tabularnewline
\midrule
\endfirsthead

\caption[]{SQL NominalElectricEquipment Table Contents} \tabularnewline
\toprule 
Field Name & Field Type & Description \tabularnewline
\midrule
\endhead

Nominal\-Electric\-Equipment\-Index & INTEGER & The internal statement number \tabularnewline
ObjectName & TEXT & The Electric Equipment object name \tabularnewline
ZoneIndex & INTEGER & Connects the NominalElectricEquipment table to the Zones table \tabularnewline
ScheduleIndex & INTEGER & Electric equipment schedule number (see Schedule table) \tabularnewline
DesignLevel & REAL & Nominal design level, in Watts \tabularnewline
FractionLatent & REAL & User-specified latent heat fraction \tabularnewline
FractionRadiant & REAL & User-specified radiant heat fraction \tabularnewline
FractionLost & REAL & User-specified lost heat fraction \tabularnewline
FractionConvected & REAL & User-specified convicted heat fraction \tabularnewline
EndUseSubcategory & TEXT & User-specified end-use subcategory \tabularnewline
\bottomrule
\end{longtable}}

Please see the Electric Equipment object in the Group-Internal Gains section of the Input-Output Reference for more information.

\subsubsection{NominalGasEquipment Table}

An overview of the NominalGasEquipment SQL table is shown below.

{\scriptsize
\begin{longtable}[c]{>{\raggedright}p{1.5in}>{\raggedright}p{1.5in}>{\raggedright}p{3.0in}}
\caption{SQL NominalGasEquipment Table Contents \label{table:table-18.-sql-nominalgasequipment-table}} \tabularnewline
\toprule 
Field Name & Field Type & Description \tabularnewline
\midrule
\endfirsthead

\caption[]{SQL NominalGasEquipment Table Contents} \tabularnewline
\toprule 
Field Name & Field Type & Description \tabularnewline
\midrule
\endhead

Nominal\-Gas\-Equipment\-Index & INTEGER & The internal statement number \tabularnewline
ObjectName & TEXT & The Gas Equipment object name \tabularnewline
ZoneIndex & INTEGER & Connects the NominalGasEquipment table to the Zones table \tabularnewline
ScheduleIndex & INTEGER & Gas equipment schedule number (see Schedule table) \tabularnewline
DesignLevel & REAL & Nominal design level, in Watts \tabularnewline
FractionLatent & REAL & User-specified latent heat fraction \tabularnewline
FractionRadiant & REAL & User-specified radiant heat fraction \tabularnewline
FractionLost & REAL & User-specified lost heat fraction \tabularnewline
FractionConvected & REAL & User-specified convicted heat fraction \tabularnewline
EndUseSubcategory & TEXT & User-specified end-use subcategory \tabularnewline
\bottomrule
\end{longtable}}

Please see the Gas Equipment object in the Group-Internal Gains section of the Input-Output Reference for more information.

\subsubsection{NominalSteamEquipment Table}

An overview of the NominalSteamEquipment SQL table is shown below.

{\scriptsize
\begin{longtable}[c]{>{\raggedright}p{1.5in}>{\raggedright}p{1.5in}>{\raggedright}p{2.99in}}
\caption{SQL NominalSteamEquipment Table Contents \label{table:table-19.-sql-nominalsteamequipment-table}} \tabularnewline
\toprule 
Field Name & Field Type & Description \tabularnewline
\midrule
\endfirsthead

\caption[]{SQL NominalSteamEquipment Table Contents} \tabularnewline
\toprule 
Field Name & Field Type & Description \tabularnewline
\midrule
\endhead

Nominal\-Steam\-Equipment\-Index & INTEGER & The internal statement number \tabularnewline
ObjectName & TEXT & The Steam Equipment object name \tabularnewline
ZoneIndex & INTEGER & Connects the NominalSteamEquipment table to the Zones table \tabularnewline
ScheduleIndex & INTEGER & Steam equipment schedule number (see Schedule table) \tabularnewline
DesignLevel & REAL & Nominal design level, in Watts \tabularnewline
FractionLatent & REAL & User-specified latent heat fraction \tabularnewline
FractionRadiant & REAL & User-specified radiant heat fraction \tabularnewline
FractionLost & REAL & User-specified lost heat fraction \tabularnewline
FractionConvected & REAL & User-specified convicted heat fraction \tabularnewline
EndUseSubcategory & TEXT & User-specified end-use subcategory \tabularnewline
\bottomrule
\end{longtable}}

Please see the Steam Equipment object in the Group-Internal Gains section of the Input-Output Reference for more information.

\subsubsection{NominalHotWaterEquipment Table}

An overview of the NominalHotWaterEquipment SQL table is shown below.

{\scriptsize
\begin{longtable}[c]{>{\raggedright}p{1.5in}>{\raggedright}p{1.5in}>{\raggedright}p{2.99in}}
\caption{SQL NominalHotWaterEquipment Table Contents \label{table:table-20.-sql-nominalhotwaterequipment-table}} \tabularnewline
\toprule 
Field Name & Field Type & Description \tabularnewline
\midrule
\endfirsthead

\caption[]{SQL NominalHotWaterEquipment Table Contents} \tabularnewline
\toprule 
Field Name & Field Type & Description \tabularnewline
\midrule
\endhead

Nominal\-Hot\-Water\-Equipment\-Index & INTEGER & The internal statement number \tabularnewline
ObjectName & TEXT & The Hot Water Equipment object name \tabularnewline
ZoneIndex & INTEGER & Connects the NominalHotWaterEquipment table to the Zones table \tabularnewline
ScheduleIndex & INTEGER & Hot water equipment schedule number (see Schedule table) \tabularnewline
DesignLevel & REAL & Nominal design level, in Watts \tabularnewline
FractionLatent & REAL & User-specified latent heat fraction \tabularnewline
FractionRadiant & REAL & User-specified radiant heat fraction \tabularnewline
FractionLost & REAL & User-specified lost heat fraction \tabularnewline
FractionConvected & REAL & User-specified convicted heat fraction \tabularnewline
EndUseSubcategory & TEXT & User-specified end-use subcategory \tabularnewline
\bottomrule
\end{longtable}}

Please see the Hot Water Equipment object in the Group-Internal Gains section of the Input-Output Reference for more information.

\subsubsection{NominalOtherEquipment Table}

An overview of the NominalOtherEquipment SQL table is shown below.

{\scriptsize
\begin{longtable}[c]{>{\raggedright}p{1.5in}>{\raggedright}p{1.5in}>{\raggedright}p{2.99in}}
\caption{SQL NominalOtherEquipment Table Contents \label{table:table-21.-sql-nominalotherequipment-table}} \tabularnewline
\toprule 
Field Name & Field Type & Description \tabularnewline
\midrule
\endfirsthead

\caption[]{SQL NominalOtherEquipment Table Contents} \tabularnewline
\toprule 
Field Name & Field Type & Description \tabularnewline
\midrule
\endhead

Nominal\-Other\-Equipment\-Index & INTEGER & The internal statement number \tabularnewline
ObjectName & TEXT & The Other Equipment object name \tabularnewline
ZoneIndex & INTEGER & Connects the NominalOtherEquipment table to the Zones table \tabularnewline
ScheduleIndex & INTEGER & Other equipment schedule number (see Schedule table) \tabularnewline
DesignLevel & REAL & Nominal design level, in Watts \tabularnewline
FractionLatent & REAL & User-specified latent heat fraction \tabularnewline
FractionRadiant & REAL & User-specified radiant heat fraction \tabularnewline
FractionLost & REAL & User-specified lost heat fraction \tabularnewline
FractionConvected & REAL & User-specified convicted heat fraction \tabularnewline
EndUseSubcategory & TEXT & User-specified end-use subcategory \tabularnewline
\bottomrule
\end{longtable}}

Please see the Other Equipment object in the Group-Internal Gains section of the Input-Output Reference for more information.

\subsubsection{NominalBaseboardHeaters Table}

An overview of the NominalBaseboardHeaters SQL table is shown below.

\begin{longtable}[c]{>{\raggedright}p{1.5in}>{\raggedright}p{1.5in}>{\raggedright}p{2.99in}}
\caption{SQL NominalBaseboardHeaters Table Contents \label{table:table-22.-sql-nominalbaseboardheaters-table}} \tabularnewline
\toprule 
Field Name & Field Type & Description \tabularnewline
\midrule
\endfirsthead

\caption[]{SQL NominalBaseboardHeaters Table Contents} \tabularnewline
\toprule 
Field Name & Field Type & Description \tabularnewline
\midrule
\endhead

Nominal\-Baseboard\-Heater\-Index & INTEGER & The internal statement number \tabularnewline
ObjectName & TEXT & The Baseboard Heat object name \tabularnewline
ZoneIndex & INTEGER & Connects the NominalBaseboardHeat table to the Zones table \tabularnewline
ScheduleIndex & INTEGER & Baseboard heat schedule number (see Schedules table) \tabularnewline
Capat\-Low\-Temperature & REAL & Capacity at low temperature, in Watts \tabularnewline
LowTemperature & REAL & Low temperature capacity setpoint \tabularnewline
Capat\-High\-Temperature & REAL & Capacity at high temperature, in Watts \tabularnewline
HighTemperature & REAL & High temperature capacity setpoint \tabularnewline
FractionRadiant & REAL & User-specified radiant heat fraction \tabularnewline
FractionConvected & REAL & User-specified convicted heat fraction \tabularnewline
EndUseSubcategory & TEXT & User-specified end-use subcategory \tabularnewline
\bottomrule
\end{longtable}

Please see the Baseboard Heat object in the Group-Internal Gains section of the Input-Output Reference for more information.

\subsubsection{NominalInfiltration Table}

An overview of the NominalInfiltration SQL table is shown below.

\begin{longtable}[c]{>{\raggedright}p{1.5in}>{\raggedright}p{1.5in}>{\raggedright}p{3.0in}}
\caption{SQL NominalInfiltration Table Contents \label{table:table-23.-sql-nominalinfiltration-table}} \tabularnewline
\toprule 
Field Name & Field Type & Description \tabularnewline
\midrule
\endfirsthead

\caption[]{SQL NominalInfiltration Table Contents} \tabularnewline
\toprule 
Field Name & Field Type & Description \tabularnewline
\midrule
\endhead

Nominal\-Infiltration\-Index & INTEGER PRIMARY KEY & The internal statement number \tabularnewline
ObjectName & TEXT & The Infiltration object name \tabularnewline
ZoneIndex & INTEGER & Connects the NominalInfiltration table to the Zones table \tabularnewline
ScheduleIndex & INTEGER & Infiltration schedule number (see Schedule table) \tabularnewline
DesignLevel & REAL & Nominal design level, in m3/s \tabularnewline
\bottomrule
\end{longtable}

Please see the Infiltration object in the Group-Airflow section of the Input-Output Reference for more information.

\subsubsection{NominalVentilation Table}

An overview of the NominalVentilation SQL table is shown below.

\begin{longtable}[c]{>{\raggedright}p{1.5in}>{\raggedright}p{1.5in}>{\raggedright}p{3.0in}}
\caption{SQL NominalVentilation Table Contents \label{table:table-24.-sql-nominalventilation-table}} \tabularnewline
\toprule 
Field Name & Field Type & Description \tabularnewline
\midrule
\endfirsthead

\caption[]{SQL NominalVentilation Table Contents} \tabularnewline
\toprule 
Field Name & Field Type & Description \tabularnewline
\midrule
\endhead

Nominal\-Ventilation\-Index & INTEGER PRIMARY KEY & The internal statement number \tabularnewline
ObjectName & TEXT & The Ventilation object name \tabularnewline
ZoneIndex & INTEGER & Connects the NominalVentilation table to the Zones table \tabularnewline
ScheduleIndex & INTEGER & Ventilation schedule number (see Schedule table) \tabularnewline
DesignLevel & REAL & Nominal design level, in m3/s \tabularnewline
\bottomrule
\end{longtable}

Please see the Ventilation object in the Group-Airflow section of the Input-Output Reference for more information.

\subsubsection{Surfaces Table}

An overview of the Surfaces SQL table is shown below.

\begin{longtable}[c]{>{\raggedright}p{1.5in}>{\raggedright}p{1.5in}>{\raggedright}p{3.0in}}
\caption{SQL Surfaces Table Contents \label{table:table-25.-sql-surfaces-table-contents}} \tabularnewline
\toprule 
Field Name & Field Type & Description \tabularnewline
\midrule
\endfirsthead

\caption[]{SQL Surfaces Table Contents} \tabularnewline
\toprule 
Field Name & Field Type & Description \tabularnewline
\midrule
\endhead

SurfaceIndex & INTEGER PRIMARY KEY & Surface number (used for cross-referencing) \tabularnewline
SurfaceName & TEXT & Surface name \tabularnewline
ConstructionIndex & INTEGER & Construction Index \tabularnewline
ClassName & TEXT & Surface class name (e.g., shading, wall) \tabularnewline
Area & REAL & Surface area (excluding cutouts) \tabularnewline
GrossArea & REAL & Surface area (including cutouts) \tabularnewline
Perimeter & REAL & Surface perimeter, in meters \tabularnewline
Azimuth & REAL & As news angle, in degrees \tabularnewline
Height & REAL & Surface height, in meters \tabularnewline
Reveal & REAL & Reveal depth, in meters \tabularnewline
Shape & INTEGER & Shape index \tabularnewline
Sides & INTEGER & Number of sides \tabularnewline
Tilt & REAL & Tilt angle, in degrees \tabularnewline
Width & REAL & Surface width, in meters \tabularnewline
HeatTransferSurf & INTEGER & Flag indicating whether the surface is a heat transfer surface \tabularnewline
BaseSurfaceIndex & INTEGER & Based surface index \tabularnewline
ZoneIndex & INTEGER & Zone index \tabularnewline
ExtBoundCond & INTEGER & External boundary condition index. For interzone surface, this is the adjacent surface number. For an internal/adiabatic surface this is the current surface number. 0=external environment, -1=ground, -2=other side coefficients, -3=other side conditions model. \tabularnewline
ExtSolar & INTEGER & Flag indicating whether the surface is exposed to solar \tabularnewline
ExtWind & INTEGER & Flag indicating whether the surface is exposed to wind \tabularnewline
\bottomrule
\end{longtable}

Please see the Surface(s) object in the Group-Thermal Zone Description/Geometry section of the Input-Output Reference for more information.

\subsubsection{Constructions Table}

An overview of the Constructions SQL table is shown below.

\begin{longtable}[c]{>{\raggedright}p{1.5in}>{\raggedright}p{1.5in}>{\raggedright}p{3.0in}}
\caption{SQL Constructions Table Contents \label{table:table-26.-sql-constructions-table-contents}} \tabularnewline
\toprule 
Field Name & Field Type & Description \tabularnewline
\midrule
\endfirsthead

\caption[]{SQL Constructions Table Contents} \tabularnewline
\toprule 
Field Name & Field Type & Description \tabularnewline
\midrule
\endhead

Construction\-Index & INTEGER PRIMARY KEY & Construction Index \tabularnewline
Name & TEXT & Construction name \tabularnewline
TotalLayers & INTEGER & Total number of layers \tabularnewline
TotalSolidLayers & INTEGER & Total number of solid layers \tabularnewline
TotalGlassLayers & INTEGER & Total number of glass layers \tabularnewline
InsideAbsorpVis & REAL & The visible absorptance of the inside layer (see Materials table) \tabularnewline
OutsideAbsorpVis & REAL & The visible absorptance of the outside layer (see Materials table) \tabularnewline
InsideAbsorpSolar & REAL & The solar absorptance of the inside layer (see Materials table) \tabularnewline
OutsideAbsorpSolar & REAL & The solar absorptance of the outside layer (see Materials table) \tabularnewline
InsideAbsorp\-Thermal & REAL & The thermal absorptance of the inside layer (see Materials table) \tabularnewline
OutsideAbsorp\-Thermal & REAL & The thermal absorptance of the outside layer (see Materials table) \tabularnewline
OutsideRoughness & INTEGER & The roughness of the outside layer \tabularnewline
TypeIsWindow & INTEGER & Flag indicating whether the construction is a window or glass door \tabularnewline
Uvalue & REAL & Nominal U-value for the construction \tabularnewline
\bottomrule
\end{longtable}

Please see the Construction object in the Group-Surface Construction Elements section of the Input-Output Reference for more information.

\subsubsection{ConstructionLayers Table}

An overview of the ConstructionLayers SQL table is shown below.

\begin{longtable}[c]{>{\raggedright}p{1.52in}>{\raggedright}p{1.5in}>{\raggedright}p{2.98in}}
\caption{SQL ConstructionLayers Table Contents \label{table:table-27.-sql-constructionlayers-table}} \tabularnewline
\toprule 
Field Name & Field Type & Description \tabularnewline
\midrule
\endfirsthead

\caption[]{SQL ConstructionLayers Table Contents} \tabularnewline
\toprule 
Field Name & Field Type & Description \tabularnewline
\midrule
\endhead

Construction\-Layers\-Index & INTEGER PRIMARY KEY & Construction Layers Index \tabularnewline
Construction\-Index & INTEGER & Construction Index (see Constructions table) \tabularnewline
LayerIndex & INTEGER & Layer number (layer 1 is the outside layer) \tabularnewline
MaterialIndex & INTEGER & Material index (see Materials table) \tabularnewline
\bottomrule
\end{longtable}

Please see the Construction object in the Group-Surface Construction Elements section of the Input-Output Reference for more information.

\subsubsection{Materials Table}

An overview of the Materials SQL table is shown below.

\begin{longtable}[c]{>{\raggedright}p{1.5in}>{\raggedright}p{1.5in}>{\raggedright}p{3.0in}}
\caption{SQL Materials Table Contents \label{table:table-28.-sql-materials-table-contents}} \tabularnewline
\toprule 
Field Name & Field Type & Description \tabularnewline
\midrule
\endfirsthead

\caption[]{SQL Materials Table Contents} \tabularnewline
\toprule 
Field Name & Field Type & Description \tabularnewline
\midrule
\endhead

MaterialIndex & INTEGER PRIMARY KEY & Material Index (links the Materials table with the ConstructionLayers and Constructions tables) \tabularnewline
Name & TEXT & Material name \tabularnewline
MaterialType & INTEGER & Material type \tabularnewline
Roughness & INTEGER & Roughness index \tabularnewline
Conductivity & REAL & Conductivity, in W/(m-K) \tabularnewline
Density & REAL & Density, in kg/m3 \tabularnewline
Porosity & REAL & Porosity \tabularnewline
Resistance & REAL & Resistance \tabularnewline
Ronly & INTEGER & Flag that indicates the material definition is of type Material:Regular-R \tabularnewline
SpecHeat & REAL & Specific heat \tabularnewline
ThermGrad\-Coef & REAL & Thermal gradient coefficient \tabularnewline
Thickness & REAL & Thickness, in meters \tabularnewline
VaporDiffus & REAL & Vapor diffusivity \tabularnewline
\bottomrule
\end{longtable}

Please see the Materials object in the Group-Surface Construction Elements section of the Input-Output Reference for more information.

\subsubsection{RoomAirModels Table}

An overview of the RoomAirModels SQL table is shown below. Please see the Group-RoomAir Models section of the Input-Output Reference for more information.

\begin{longtable}[c]{@{}lll@{}}
\caption{SQL RoomAirModels Table Contents \label{table:table-29.-sql-roomairmodels-table-contents}} \tabularnewline
\toprule 
Field Name & Field Type & Description \tabularnewline
\midrule
\endfirsthead

\caption[]{SQL RoomAirModels Table Contents} \tabularnewline
\toprule 
Field Name & Field Type & Description \tabularnewline
\midrule
\endhead

ZoneIndex & INTEGER PRIMARY KEY & Zone index \tabularnewline
AirModelName & TEXT & Air model name \tabularnewline
AirModelType & INTEGER & Air model index \tabularnewline
TempCouple\-Scheme & INTEGER & Temperature coupling index \tabularnewline
SimAirModel & INTEGER & Simulation air model index \tabularnewline
\bottomrule
\end{longtable}

\subsubsection{ComponentSizes Table}

An overview of the ComponentSizes SQL table is shown below.

\begin{longtable}[c]{@{}lll@{}}
\caption{SQL ComponentSizes Table Contents \label{table:table-30.-sql-componentsizes-table-contents}} \tabularnewline
\toprule 
Field Name & Field Type & Description \tabularnewline
\midrule
\endfirsthead

\caption[]{SQL ComponentSizes Table Contents} \tabularnewline
\toprule 
Field Name & Field Type & Description \tabularnewline
\midrule
\endhead

Component\-Sizes\-Index & INTEGER PRIMARY KEY & Component Sizes Index \tabularnewline
CompType & TEXT & Component type \tabularnewline
CompName & TEXT & Component name \tabularnewline
Description & TEXT & Component description \tabularnewline
Value & REAL & Sizing value \tabularnewline
Units & TEXT & Sizing units \tabularnewline
\bottomrule
\end{longtable}

Please see the Sizing object in the Group-Design Objects section of the Input-Output Reference for more information.

\subsubsection{SystemSizes Table}

An overview of the SystemSizes SQL table is shown below. Please see the System Sizing object in the Group-Design Objects section of the Input-Output Reference for more information.

\begin{longtable}[c]{@{}lll@{}}
\caption{SQL SystemSizes Table Contents \label{table:table-31.-sql-systemsizes-table-contents}} \tabularnewline
\toprule 
Field Name & Field Type & Description \tabularnewline
\midrule
\endfirsthead

\caption[]{SQL SystemSizes Table Contents} \tabularnewline
\toprule 
Field Name & Field Type & Description \tabularnewline
\midrule
\endhead

SystemSizes\-Index & INTEGER PRIMARY KEY & System Sizes Index \tabularnewline
SystemName & TEXT & System name \tabularnewline
Description & TEXT & System description \tabularnewline
Value & REAL & Sizing value \tabularnewline
Units & TEXT & Sizing units \tabularnewline
\bottomrule
\end{longtable}

\subsubsection{ZoneSizes Table}

An overview of the ZoneSizes SQL table is shown below.

\begin{longtable}[c]{>{\raggedright}p{1.5in}>{\raggedright}p{1.5in}>{\raggedright}p{3in}}
\caption{SQL ZoneSizes Table Contents \label{table:table-32.-sql-zonesizes-table-contents}} \tabularnewline
\toprule 
Field Name & Field Type & Description \tabularnewline
\midrule
\endfirsthead

\caption[]{SQL ZoneSizes Table Contents} \tabularnewline
\toprule 
Field Name & Field Type & Description \tabularnewline
\midrule
\endhead

ZoneSizes\-Index & INTEGER PRIMARY KEY & Zone Sizes Index \tabularnewline
ZoneName & TEXT & Zone name \tabularnewline
LoadType & TEXT & Load type \tabularnewline
DesLoad & REAL & Design load \tabularnewline
CalcDesFlow & REAL & Calculated design flow \tabularnewline
UserDesFlow & REAL & User-specified design flow \tabularnewline
DesDayName & TEXT & Design day name \tabularnewline
PeakHrMin & TEXT & Time of the peak temperature \tabularnewline
PeakTemp & REAL & Peak temperature \tabularnewline
PeakHumRat & REAL & Peak humidity ratio \tabularnewline
CalcOutside\-Air\-Flow & REAL & Calculated outside air flow rate, in m3/s \tabularnewline
\bottomrule
\end{longtable}

Please see the Zone Sizing object in the Group-Design Objects section of the Input-Output Reference for more information.

\subsubsection{ZoneGroups Table}

An overview of the ZoneGroups SQL table is shown below.

\begin{longtable}[c]{>{\raggedright}p{1.5in}>{\raggedright}p{1.5in}>{\raggedright}p{3.0in}}
\caption{SQL ZoneGroups Table Contents \label{table:table-33.-sql-zonegroups-table-contents}} \tabularnewline
\toprule 
Field Name & Field Type & Description \tabularnewline
\midrule
\endfirsthead

\caption[]{SQL ZoneGroups Table Contents} \tabularnewline
\toprule 
Field Name & Field Type & Description \tabularnewline
\midrule
\endhead

ZoneGroup\-Index & INTEGER PRIMARY KEY & Zone group index \tabularnewline
ZoneGroupName & TEXT & Zone list name \tabularnewline
ZoneListIndex & INTEGER & The ZoneListIndex links this record to the respective zone list record (see ZoneLists table below) \tabularnewline
ZoneList\-Multiplier & INTEGER & Zone list multiplier \tabularnewline
\bottomrule
\end{longtable}

Please see the Zone Group object in the Group-Thermal Zone Description/Geometry section of the Input-Output Reference for more information.

\subsubsection{ZoneLists Table}

An overview of the ZoneLists SQL table is shown below.

\begin{longtable}[c]{@{}lll@{}}
\caption{SQL ZoneLists Table Contents \label{table:table-34.-sql-zonelists-table-contents}} \tabularnewline
\toprule 
Field Name & Field Type & Description \tabularnewline
\midrule
\endfirsthead

\caption[]{SQL ZoneLists Table Contents} \tabularnewline
\toprule 
Field Name & Field Type & Description \tabularnewline
\midrule
\endhead

ZoneListIndex & INTEGER PRIMARY KEY & Zone list index \tabularnewline
Name & TEXT & Zone list name \tabularnewline
\bottomrule
\end{longtable}

Please see the Zone List object in the Group-Thermal Zone Description/Geometry section of the Input-Output Reference for more information.

\subsubsection{ZoneInfoZoneLists Table}

An overview of the ZoneInfoZoneLists SQL table is shown below.

\begin{longtable}[c]{@{}lll@{}}
\caption{SQL ZoneLists Table Contents \label{table:sql-zonelists-table-contents}} \tabularnewline
\toprule 
Field Name & Field Type & Description \tabularnewline
\midrule
\endfirsthead

\caption[]{SQL ZoneLists Table Contents} \tabularnewline
\toprule 
Field Name & Field Type & Description \tabularnewline
\midrule
\endhead

ZoneListIndex & INTEGER & Zone list index \tabularnewline
ZoneIndex & INTEGER & Zone index \tabularnewline
\bottomrule
\end{longtable}

This is a join table to represent the many-to-many relationship of zones and zone lists.  ZoneListIndex and ZoneIndex together are the Primary Key for this table.

\subsection{Miscellaneous Tables}

\subsubsection{Schedules Table}

An overview of the Schedules SQL table is shown below.

\begin{longtable}[c]{@{}lll@{}}
\caption{SQL Schedules Table Contents \label{table:table-35.-sql-schedules-table-contents}} \tabularnewline
\toprule 
Field Name & Field Type & Description \tabularnewline
\midrule
\endfirsthead

\caption[]{SQL Schedules Table Contents} \tabularnewline
\toprule 
Field Name & Field Type & Description \tabularnewline
\midrule
\endhead

ScheduleIndex & INTEGER PRIMARY KEY & Schedule index \tabularnewline
ScheduleName & TEXT & Schedule name \tabularnewline
ScheduleType & TEXT & Schedule Type \tabularnewline
ScheduleMinimum & REAL & ScheduleMinimum \tabularnewline
ScheduleMaximum & REAL & ScheduleMaximum \tabularnewline
\bottomrule
\end{longtable}

Please see the Group-Schedules section of the Input-Output Reference for more information.

\subsubsection{Simulations Table}

An overview of the Simulations SQL table is shown below.  Currently there will only be one record in the table, because the SQLite database is relevant to only one simulation. In the future this might change if multiple simulations are aggregated into a larger database.

\begin{longtable}[c]{>{\raggedright}p{1.5in}>{\raggedright}p{1.5in}>{\raggedright}p{3.0in}}
\caption{SQL Simulations Table Contents \label{table:table-36.-sql-simulations-table-contents}} \tabularnewline
\toprule 
Field Name & Field Type & Description \tabularnewline
\midrule
\endfirsthead

\caption[]{SQL Simulations Table Contents} \tabularnewline
\toprule 
Field Name & Field Type & Description \tabularnewline
\midrule
\endhead

SimulationIndex & INTEGER PRIMARY KEY & Simulation index, currently there will be only one record in this table. \tabularnewline
EnergyPlus\-Version & TEXT & The version of EnergyPlus that was used to run the simulation. \tabularnewline
TimeStamp & TEXT & A text string containing the timestamp when the simulation was run. \tabularnewline
NumTimesteps\-Per\-Hour & INTEGER & The number of timesteps per hour used for the simulation.~ See the Timestep object for more information. \tabularnewline
Completed & BOOL & True if the simulation completed without crashing. \tabularnewline
Completed\-Successfully & BOOL & True if the simulation completed without any severe or fatal errors. \tabularnewline
\bottomrule
\end{longtable}

\subsubsection{EnvironmentPeriods Table}

An overview of the EnvironmentPeriods SQL table is shown below.

\begin{longtable}[c]{>{\raggedright}p{1.5in}>{\raggedright}p{1.5in}>{\raggedright}p{3.0in}}
\caption{SQL EnvironmentPeriods Table Contents \label{table:table-37.-sql-environmentperiods-table}} \tabularnewline
\toprule 
Field Name & Field Type & Description \tabularnewline
\midrule
\endfirsthead

\caption[]{SQL EnvironmentPeriods Table Contents} \tabularnewline
\toprule 
Field Name & Field Type & Description \tabularnewline
\midrule
\endhead

Environment\-Period\-Index & INTEGER PRIMARY KEY & Environment period index \tabularnewline
SimulationIndex & INTEGER & A foreign key to a record in the Simulations table. \tabularnewline
EnvironmentName & TEXT & A text string identifying the environment. \tabularnewline
EnvironmentType & INTEGER & An enumeration of the environment type.~ (1 = Design Day, 2 =~ Design Run Period, 3 = Weather Run Period)~ See the various SizingPeriod objects and the RunPeriod object for details. \tabularnewline
\bottomrule
\end{longtable}

\subsubsection{TabularData Table}

An overview of the TabularData SQL table is shown below. The TabularData table is only output when the “SimpleAndTabular” choice is made in the Output:SQLite object. The TabularData table contains all of the tabular report information that is typically output to one of the formats controlled by the OutputControl:Table:Style object.  This is a generic way of dumping all of the existing tabular reports to the SQLite database.  This table has many references to the companion Strings table. The Strings table simply associates strings to an integer value for space efficiency of the database. Tabular data is easier to query from the TabularDataWithStrings table, which is implemented as a database view that joins this table with the Strings table.

\begin{longtable}[c]{>{\raggedright}p{1.5in}>{\raggedright}p{1.5in}>{\raggedright}p{3.0in}}
\caption{SQL TabularData Table Contents \label{table:table-38.-sql-tabulardata-table-contents}} \tabularnewline
\toprule 
Field Name & Field Type & Description \tabularnewline
\midrule
\endfirsthead

\caption[]{SQL TabularData Table Contents} \tabularnewline
\toprule 
Field Name & Field Type & Description \tabularnewline
\midrule
\endhead

TabularDataIndex & INTEGER PRIMARY KEY & Tabular Data Index \tabularnewline
ReportNameIndex & INTEGER & A foreign key to an entry in the Strings table, which indicates the name of the report the record belongs to. \tabularnewline
ReportFor\-String\-Index & INTEGER & A foreign key to an entry in the Strings table, which indicates the “For” text that is associated with the record.~ An example is the Space Loads report where the “For” text identifies which zone the record pertains to. \tabularnewline
TableNameIndex & INTEGER & A foreign key to an entry in the Strings table, which indicates the name of the table the record belongs to.~ This is typically the text that immediately precedes the tables in html format. \tabularnewline
Simulation\-Index & INTEGER & A foreign key to the Simulations table. \tabularnewline
RowNameIndex & INTEGER & A foreign key to an entry in the Strings table, which indicates the name of the row the record belongs to. \tabularnewline
ColumnName\-Index & INTEGER & A foreign key to an entry in the Strings table, which indicates the name of the column the record belongs to. \tabularnewline
UnitsIndex & INTEGER & A foreign key to an entry in the Strings table, which indicates the units associated with the record. \tabularnewline
Simulation\-Index & INTEGER & A foreign key to the Simulations table. \tabularnewline
RowId & INTEGER & The index of the row associated with the record.~ Data in the first row of a tabular report would have the Id 1. \tabularnewline
ColumnId & INTEGER & The index of the column associated with the record.~ Data in the first column of a tabular report would have the Id 1. \tabularnewline
Value & REAL & The value of the record.~ Most data is numeric, thus the type for this column is REAL.~ The SQLite engine will first try to store the value as a REAL, however if this fails the value will be stored as text. \tabularnewline
\bottomrule
\end{longtable}

\subsubsection{Strings Table}

An overview of the Strings SQL table is shown below. The Strings table is only output when the “SimpleAndTabular” choice is made in the Output:SQLite object.  The Strings table simply associates strings with integers to save space in the database.

\begin{longtable}[c]{>{\raggedright}p{1.5in}>{\raggedright}p{1.5in}>{\raggedright}p{3in}}
\caption{SQL Strings Table Contents \label{table:table-39.-sql-strings-table-contents}} \tabularnewline
\toprule 
Field Name & Field Type & Description \tabularnewline
\midrule
\endfirsthead

\caption[]{SQL Strings Table Contents} \tabularnewline
\toprule 
Field Name & Field Type & Description \tabularnewline
\midrule
\endhead

StringIndex & INTEGER PRIMARY KEY & The integer primary key that uniquely identifies the string record. \tabularnewline
StringType\-Index & INTEGER & A foreign key to the StringTypes table. \tabularnewline
Value & TEXT & The record’s string value. \tabularnewline
\bottomrule
\end{longtable}

\subsubsection{StringTypes Table}

An overview of the StringTypes SQL table is shown below. The StringTypes table is only output when the “SimpleAndTabular” choice is made in the Output:SQLite object.

\begin{longtable}[c]{>{\raggedright}p{1.5in}>{\raggedright}p{1.5in}>{\raggedright}p{3in}}
\caption{SQL StringTypes Table Contents \label{table:table-40.-sql-stringtypes-table-contents}} \tabularnewline
\toprule 
Field Name & Field Type & Description \tabularnewline
\midrule
\endfirsthead

\caption[]{SQL StringTypes Table Contents} \tabularnewline
\toprule 
Field Name & Field Type & Description \tabularnewline
\midrule
\endhead

StringType\-Index & INTEGER PRIMARY KEY & The integer primary key. \tabularnewline
Value & TEXT & The record’s string value.~ Currently there are 6 possible values; ReportName, ReportForString, TableName, RowName, ColumnName, and Units.~ That is, entries in the Strings table are one of these types. \tabularnewline
\bottomrule
\end{longtable}

\subsubsection{TabularDataWithStrings Table}

An overview of the TabularDataWithStrings SQL table is shown below.  This table is implemented as a view of other tables and is not actually a table in the database.  This view is similar to the TabularData table except text strings are directly available in place of foreign keys to the Strings table.   The TabularDataWithStrings view is only created when the “SimpleAndTabular” choice is made in the Output:SQLite object.

An example of a SQL query to get a table out of the AnnualBuildingUtilityPerformanceSummary report is the following.

\begin{lstlisting}
select * FROM TabularDataWithStrings WHERE
  ReportName = `AnnualBuildingUtilityPerformanceSummary' and
  TableName = `Site and Source Energy';
\end{lstlisting}

\begin{longtable}[c]{>{\raggedright}p{1.5in}>{\raggedright}p{1.5in}>{\raggedright}p{3in}}
\caption{SQL TabularDataWithStrings Table Contents \label{table:table-41.-sql-tabulardatawithstrings-table}} \tabularnewline
\toprule 
Field Name & Field Type & Description \tabularnewline
\midrule
\endfirsthead

\caption[]{SQL TabularDataWithStrings Table Contents} \tabularnewline
\toprule 
Field Name & Field Type & Description \tabularnewline
\midrule
\endhead

TabularData\-Index & INTEGER PRIMARY KEY & The index of the record \tabularnewline
Value & REAL & The value of the record \tabularnewline
ReportName & TEXT & The name of the report the record belongs to. \tabularnewline
ReportFor\-String & TEXT & The “For” string. \tabularnewline
TableName & TEXT & The name of the table the record belongs to. \tabularnewline
RowName & TEXT & The name of the row associated with the record. \tabularnewline
ColumnName & TEXT & The name of the column associated with the record. \tabularnewline
Units & TEXT & The units associated with the record. \tabularnewline
\bottomrule
\end{longtable}

\subsubsection{Errors Table}\label{errors-table}

The Errors SQL table reports errors and warnings for the simulation.~ The content of this table includes most of the content of the error file. However, it does not contain purely informational messages (e.g.~not warnings or errors) and some of the error summary statistics which may be readily computed. Users should be aware that errors experienced while processing input cause EnergyPlus to exit before the SQLite output database has been initialized. This is because the Output:SQLite object must be parsed from the input to request the database. Therefore, failures in input processing are indicated by the absence of a SQLite output file. The regular error or audit files must be used to identify the erroneous input objects in this case. Because each error or warning message is expected to be unique these messages are stored directly in the Errors table and not in the Strings table.

\begin{longtable}[c]{>{\raggedright}p{1.5in}>{\raggedright}p{1.5in}>{\raggedright}p{3in}}
\caption{SQL Errors Table Contents \label{table:table-42.-sql-errors-table-contents}} \tabularnewline
\toprule 
Field Name & Field Type & Description \tabularnewline
\midrule
\endfirsthead

\caption[]{SQL Errors Table Contents} \tabularnewline
\toprule 
Field Name & Field Type & Description \tabularnewline
\midrule
\endhead

ErrorIndex & INTEGER PRIMARY KEY & The index of the error or warning message. \tabularnewline
SimulationIndex & INTEGER FOREIGN KEY & The simulation the error or warning pertains to. \tabularnewline
ErrorType & INTEGER & Type of error or warning 0=Warning, 1=Severe, 2=Fatal. \tabularnewline
ErrorMessage & TEXT & The text of the error message. \tabularnewline
Count & INTEGER & Number of times that the error was repeated.~ This will be 1 for normal warnings or errors, 0 for warning and error messages, and the number of times the error or warning is repeated. \tabularnewline
\bottomrule
\end{longtable}

\subsection{How to Access the SQLite Data}\label{how-to-access-the-sqlite-data}

The SQL database can be accessed in a number of ways, including via the command line, through ODBC, or through as SQLite's API interface. SQLite uses the industry standard SQL 92 language.

\subsubsection{Command Line}\label{command-line}

One of the simplest ways to access the data in the SQL database is by way of the SQL command line tool (i.e., sqlite3). A brief description of how to use sqlite3 for each computing platform is given below.

\subsubsection{Windows XP and Windows Vista}\label{windows-xp-and-windows-vista}

While Windows does not ship with sqlite3 installed, the sqlite3 binary can be downloaded from the SQLite webpage (www.sqlite.org/download.html). After downloading the precompiled binary, install it in the EnergyPlus directory.

Once the sqlite3 executable is installed, access the program from the command line by typing ``sqlite3'' at the DOS prompt.

\subsubsection{Linux}\label{linux}

The sqlite3 command line tool comes preinstalled on a number of more recent Linux releases. To see if sqlite3 is available (and which version is installed), type ``sqlite3 --version'' from the command line. If sqlite3 is not installed, the sqlite3 binary, as well as source code, can be downloaded from the SQLite webpage (\url{http://www.sqlite.org/download.html}) and installed in the directory of your choice.

\subsubsection{Macintosh OS X}\label{macintosh-os-x}

The sqlite3 program comes standard on MacOS X 10.5. From the command line, type ``sqlite3 --version'' to see which version of sqlite3 is installed. In order to access the database created by EnergyPlus, version 3 or later is required.

\subsubsection{Accessing the Data from the Command Line}\label{accessing-the-data-from-the-command-line}

Once it has been confirmed that SQLite3 is installed on your machine, the SQL database can be accessed by typing:

\begin{lstlisting}
sqlite3 <database name>
\end{lstlisting}

at the command line, where \textless{}database name \textgreater{} is the name of the SQL database (e.g., sqlite3 eplusout.sql).

The sqlite.org \href{http://www.sqlite.org/sqlite.html}{website} gives examples of how sqlite3 can be used to access and output data in various formats.

\subsubsection{ODBC}\label{odbc}

ODBC allows access to the SQL database from a variety of programs, including Microsoft Excel, Microsoft Access, and FileMaker. How to install and use ODBC drivers is outside the scope of this document, and more information can be found at the following websites:

Macintosh ODBC drivers:

\url{http://www.actualtechnologies.com/}

Windows and Linux ODBC information and drivers:

\url{http://www.sqlite.org/cvstrac/wiki?p=SqliteOdbc}

\url{http://www.ch-werner.de/sqliteodbc/}

\subsubsection{API}\label{api}

Sqlite3 includes a rich C++ API (detailed on the SQLite website \href{http://www.sqlite.org/cintro.html}{www.sqlite.org/cintro.html}), and wrappers for the API interface are available in a variety of programming languages, including Fortran, TCL, and Ruby (see \href{http://www.sqlite.org/cvstrac/wiki?p\%20=\%20SqliteWrappers}{www.sqlite.org/cvstrac/wiki?p = SqliteWrappers} for more information).
