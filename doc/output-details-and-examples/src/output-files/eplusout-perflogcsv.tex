\section{eplusout\_perflog.csv}

An output file ending with \_perflog.csv is created whenever the PerformancePrecisionTradeoffs object is present. This log file consists of echoing the settings in the PerformancePrecisionTradeoffs object, 
some overall building energy results, the runtime of the simulation, oscillation outputs, and the number of warning and severe messages. The \_perflog.csv
file is appended to instead of replaced each time a simulation is run to facilitate evaluation of the performance objects options and their impact on the runtime and other factors so  
the user can determine which combination is best for their particular input file. It includes a timestamp. Since it is a CSV (comma separated value) file, it is easily opened in spreadsheet software. Please ensure that file is 
not open in the spreadsheet while EnergyPlus is running so that the file can be appended to. If the spreadsheet program is open with the file, it will be locked and not able to be appended. The fields in the
file are shown below.


\begin{itemize}
\item
  Program
\item
  Version
\item
  TimeStamp
\item
  Use Coil Direct Solution
\item
  Zone Radiant Exchange Algorithm
\item
  Override Mode
\item
  Number of Timesteps per Hour
\item
  Minimum Number of Warmup Days
\item
  SuppressAllBeginEnvironmentResets
\item
  MaxZoneTempDiff
\item
  Electricity ABUPS Total [J]
\item
  Natural Gas ABUPS Total [J]
\item
  Additional Fuel ABUPS Total [J]
\item
  District Cooling ABUPS Total [J]
\item
  District Heating ABUPS Total [J]
\item
  Water ABUPS Total [m3]
\item
  Values Gathered Over [hours]
\item
  Facility Any Zone Oscillating Temperatures Time [hours]
\item
  Facility Any Zone Oscillating Temperatures During Occupancy Time [hours]
\item
  Facility Any Zone Oscillating Temperatures in Deadband Time [hours]
\item
  Run Time [seconds]
\item
  Run Time [string]
\item
  Number of Warnings
\item
  Number of Severe
\end{itemize}

When using this file it is good to set PerformancePrecisionTradeoffs to default values the first time to establish a baseline then change one input at a time and rerunning to understand the impact of that input. The total runtime decrease 
and the impact on energy and the ocscillation variables should be considered prior to committing to a specific PerformancePrecisionTradeoffs scheme.

