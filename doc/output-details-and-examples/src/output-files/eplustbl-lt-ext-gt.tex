\section{eplustbl.\textless{}ext\textgreater{}}\label{eplustbl.ext}

The eplustbl file contains the tabular output results that are created when using the following objects:

\begin{itemize}
\item
  Output:Table:SummaryReports
\item
  Output:Table:TimeBins
\item
  Output:Table:Monthly
\item
  UtilityCost:Tariff
\item
  ComponentCost:Line Item
\end{itemize}

The format and the extension for the file depends on the setting of the ColumnSeparator field of the Output:Table:Style object. The choices of HTML, tab, fixed, comma, and XML result in eplustbl.htm, eplustbl.tab, eplustbl.txt, eplustbl.csv, eplustbl.xml respectively. The HTML version of the report also includes a table of contents that allows easier navigation through the file.

By default the energy units reported in all of the eplustbl files are in Joules (J) but the UnitConversion field of the Output:Table:Style object allows for the values to be reported in MJ, GJ or in kWh. In addition, the Output:Table:Style object can specify for the tables to be in IP units for all fields.

\subsection{Output:Table:SummaryReports}\label{outputtablesummaryreports}

Several predefined reports are available from the Output:Table:SummaryReports object including the following. (spaces are inserted in names for readability; keys are included in the Input Data Dictionary or by just removing the spaces):

\begin{itemize}
\tightlist
\item
  All Summary
\end{itemize}

Lists all following applicable tables with ``Summary'' in the name.

\begin{itemize}
\tightlist
\item
  All Monthly
\end{itemize}

Lists all following applicable tables with ``Monthly'' in the name.

\begin{itemize}
\tightlist
\item
  All Summary And Monthly
\end{itemize}

Lists all following applicable tables with both ``Summary'' and ``Monthly'' in the name. This does not include the Zone Component Load Summary report.

\begin{itemize}
\tightlist
\item
  All Summary and Sizing Period
\end{itemize}

Lists the All Summary tables as well as the Zone Component Load report (currently the only Sizing Period report). Please note that the Zone Component Load report does increase the run time because it repeats sizing periods.

\begin{itemize}
\tightlist
\item
  All Summary Monthly and Sizing Period
\end{itemize}

Lists the All Summary tables, the Monthly tables as well as the Zone Component Load report (currently the only Sizing Period report). Please note that the Zone Component Load report does increase the run time because it repeats sizing periods.

\begin{itemize}
\item
  Annual Building Utility Performance Summary
\item
  Input Verification and Results Summary
\item
  Demand End Use Components Summary
\item
  Source Energy End Use Components Summary
\item
  Climatic Data Summary
\item
  Equipment Summary
\item
  Envelope Summary
\item
  Surface Shadowing Summary
\item
  Shading Summary
\item
  Lighting Summary
\item
  HVAC Sizing Summary
\item
  System Summary
\item
  Component Sizing Summary
\item
  Outdoor Air Summary
\item
  Object Count Summary
\item
  Component Cost Economics Summary
\item
  Energy Meters
\item
  Sensible Heat Gain Summary
\item
  Standard 62.1 Summary
\item
  Zone Component Load Summary
\item
  Zone Cooling Summary Monthly
\item
  Zone Heating Summary Monthly
\item
  Zone Electric Summary Monthly
\item
  Space Gains Monthly
\item
  Peak Space Gains Monthly
\item
  Space Gain Components At Cooling Peak Monthly
\item
  Energy Consumption Electricity Natural Gas Monthly
\item
  Energy Consumption Electricity Generated Propane Monthly
\item
  Energy Consumption Diesel FuelOil Monthly
\item
  Energy Consumption District Heating Cooling Monthly
\item
  Energy Consumption Coal Gasoline Monthly
\item
  End Use Energy Consumption Electricity Monthly
\item
  End Use Energy Consumption NaturalGas Monthly
\item
  End Use Energy Consumption Diesel Monthly
\item
  End Use Energy Consumption FuelOil Monthly
\item
  End Use Energy Consumption Coal Monthly
\item
  End Use Energy Consumption Propane Monthly
\item
  End Use Energy Consumption Gasoline Monthly
\item
  Peak Energy End Use Electricity Part1 Monthly
\item
  Peak Energy End Use Electricity Part2 Monthly
\item
  Electric Components Of Peak Demand Monthly
\item
  Peak Energy End Use NaturalGas Monthly
\item
  Peak Energy End Use Diesel Monthly
\item
  Peak Energy End Use FuelOil Monthly
\item
  Peak Energy End Use Coal Monthly
\item
  Peak Energy End Use Propane Monthly
\item
  Peak Energy End Use Gasoline Monthly
\item
  Setpoints Not Met With Temperatures Monthly
\item
  Comfort Report Simple55 Monthly
\item
  Unglazed Transpired Solar Collector Summary Monthly
\item
  Occupant Comfort Data Summary Monthly
\item
  Chiller Report Monthly
\item
  Tower Report Monthly
\item
  Boiler Report Monthly
\item
  DX Report Monthly
\item
  Window Report Monthly
\item
  Window Energy Report Monthly
\item
  Window Zone Summary Monthly
\item
  Window Energy Zone Summary Monthly
\item
  Average Outdoor Conditions Monthly
\item
  Outdoor Conditions Maximum DryBulb Monthly
\item
  Outdoor Conditions Minimum DryBulb Monthly
\item
  Outdoor Conditions Maximum WetBulb Monthly
\item
  Outdoor Conditions Maximum DewPoint Monthly
\item
  Outdoor Ground Conditions Monthly
\item
  WindowAC Report Monthly
\item
  Water Heater Report Monthly
\item
  Generator Report Monthly
\item
  Daylighting Report Monthly
\item
  Coil Report Monthly
\item
  PlantLoop Demand Report Monthly
\item
  Fan Report Monthly
\item
  Pump Report Monthly
\item
  CondLoop Demand Report Monthly
\item
  Zone Temperature Oscillation Report Monthly
\item
  AirLoop System Energy And Water Use Monthly
\item
  AirLoop System Component Loads Monthly
\item
  AirLoop System Component Energy Use Monthly
\item
  Mechanical Ventilation Loads Monthly
\end{itemize}

Each of these reports is made up of several sub-tables of information. Examples of some of the tables are shown below. To enable all of the reports the single All Summary may be specified.

\subsection{Annual Building Utility Performance Summary}\label{annual-building-utility-performance-summary}

The Annual Building Utility Performance Summary report provides an overview of energy consumption in the building for different end uses. The following is an example this report (some columns may be truncated due to page size). The key used to obtain this report is AnnualBuildingUtilityPerformanceSummary.

In the \emph{Comfort and Setpoint Not Met Summary} sub-table, facility hours represents the total number of hours that any single zone did not meet the comfort or setpoint criteria. It is not weighted by number of zones or number of occupants.

The values in the \emph{End Uses} sub-table are from report meters. To determine which components are attached to each end-use meter, consult the meter details output file (*.mtd).

Report: Annual Building Utility Performance Summary

For: Entire Facility

Timestamp: 2009-02-10 12:39:35

Values gathered over 8760.00 hours

Site and Source Energy

\begin{longtable}[c]{p{1.5in}p{1.5in}p{1.5in}p{1.5in}}
\toprule 
~ & Total Energy (GJ) & Energy Per Total Building Area (MJ/m2) & Energy Per Conditioned Building Area (MJ/m2) \tabularnewline \midrule
\endhead
Total Site Energy & 194.80 & 210.09 & 210.09 \tabularnewline
Net Site Energy & 194.80 & 210.09 & 210.09 \tabularnewline
Total Source Energy & 532.25 & 574.04 & 574.04 \tabularnewline
Net Source Energy & 532.25 & 574.04 & 574.04 \tabularnewline
\bottomrule
\end{longtable}

\subsection{XML Tabular Output}\label{xml-tabular-output}

The tables discussed in this section can also be output in an XML format that may be easier for some other programs to extract specific results. An excerpt of the XML file format is shown below. This report presents the same output results as the other tabular report formats but using XML tags.

\textless{}?xml version = ``1.0''?\textgreater{}

\textless{}EnergyPlusTabularReports\textgreater{}

~ \textless{}BuildingName\textgreater{}NONE\textless{}/BuildingName\textgreater{}

~ \textless{}EnvironmentName\textgreater{}Chicago Ohare Intl Ap IL USA TMY3 WMO\# = 725300\textless{}/EnvironmentName\textgreater{}

~ \textless{}WeatherFileLocationTitle\textgreater{}Chicago Ohare Intl Ap IL USA TMY3 WMO\# = 725300\textless{}/WeatherFileLocationTitle\textgreater{}

~ \textless{}ProgramVersion\textgreater{}EnergyPlus, Version 8.0, YMD = 2013.02.18 15:50\textless{}/ProgramVersion\textgreater{}

~ \textless{}SimulationTimestamp\textgreater{}

~~~ \textless{}Date\textgreater{}

~~~~~ 2013-02-18

~~~ \textless{}/Date\textgreater{}

~~~ \textless{}Time\textgreater{}

~~~~~ 15:50:33

~~~ \textless{}/Time\textgreater{}

~ \textless{}/SimulationTimestamp\textgreater{}

\textless{}AnnualBuildingUtilityPerformanceSummary\textgreater{}

~ \textless{}for\textgreater{}Entire Facility\textless{}/for\textgreater{}

~ \textless{}SiteAndSourceEnergy\textgreater{}

~~~ \textless{}name\textgreater{}TotalSiteEnergy\textless{}/name\textgreater{}

~~~ \textless{}TotalEnergy units = ``kBtu''\textgreater{}200671.97\textless{}/TotalEnergy\textgreater{}

~~~ \textless{}EnergyPerTotalBuildingArea units = ``kBtu/ft2''\textgreater{}143.32\textless{}/EnergyPerTotalBuildingArea\textgreater{}

~~~ \textless{}EnergyPerConditionedBuildingArea units = ``kBtu/ft2''\textgreater{}143.32\textless{}/EnergyPerConditionedBuildingArea\textgreater{}

~ \textless{}/SiteAndSourceEnergy\textgreater{}

~ \textless{}SiteAndSourceEnergy\textgreater{}

~~~ \textless{}name\textgreater{}NetSiteEnergy\textless{}/name\textgreater{}

~~~ \textless{}TotalEnergy units = ``kBtu''\textgreater{}200671.97\textless{}/TotalEnergy\textgreater{}

~~~ \textless{}EnergyPerTotalBuildingArea units = ``kBtu/ft2''\textgreater{}143.32\textless{}/EnergyPerTotalBuildingArea\textgreater{}

~~~ \textless{}EnergyPerConditionedBuildingArea units = ``kBtu/ft2''\textgreater{}143.32\textless{}/EnergyPerConditionedBuildingArea\textgreater{}

~ \textless{}/SiteAndSourceEnergy\textgreater{}

~ \textless{}SiteAndSourceEnergy\textgreater{}

~~~ \textless{}name\textgreater{}TotalSourceEnergy\textless{}/name\textgreater{}

~~~ \textless{}TotalEnergy units = ``kBtu''\textgreater{}442504.19\textless{}/TotalEnergy\textgreater{}

~~~ \textless{}EnergyPerTotalBuildingArea units = ``kBtu/ft2''\textgreater{}316.04\textless{}/EnergyPerTotalBuildingArea\textgreater{}

~~~ \textless{}EnergyPerConditionedBuildingArea units = ``kBtu/ft2''\textgreater{}316.04\textless{}/EnergyPerConditionedBuildingArea\textgreater{}

~ \textless{}/SiteAndSourceEnergy\textgreater{}

~ \textless{}SiteAndSourceEnergy\textgreater{}

~~~ \textless{}name\textgreater{}NetSourceEnergy\textless{}/name\textgreater{}

~~~ \textless{}TotalEnergy units = ``kBtu''\textgreater{}397810.98\textless{}/TotalEnergy\textgreater{}

~~~ \textless{}EnergyPerTotalBuildingArea units = ``kBtu/ft2''\textgreater{}284.12\textless{}/EnergyPerTotalBuildingArea\textgreater{}

~~~ \textless{}EnergyPerConditionedBuildingArea units = ``kBtu/ft2''\textgreater{}284.12\textless{}/EnergyPerConditionedBuildingArea\textgreater{}

~ \textless{}/SiteAndSourceEnergy\textgreater{}

\emph{An actual file will be much longer than this example but follows this format.}
