\section{G-Function Spreadsheet}\label{g-function-spreadsheet}

The spreadsheet (g-function\_library.xls) has comparison plots for different configurations/grouts for the Ground Heat Exchanger:Vertical. A user can see how the boreholes interact with each other in different configurations. It has the actual screen shot (GLHEPro) showing the conditions/inputs used to obtain the data. GLHEPro is not included with EnergyPlus and it or something similar should be used to generate similar values. Some of the values used in getting g-functions are not used in the EnergyPlus Ground loop model, but the spreadsheet screen shot gives an idea of what the inputs were in obtaining the g-functions.

The reference data set GLHERefData.idf contains sets of parameters for the Ground Heat Exchangers:

``This file contains sample input for the ground loop heat exchanger model. The response of the borehole/ground is found from the `G-function' that is defined in the input as series of `n' pairs of values (LNTTSn, GNFCn). It is important to note that the G-functions have to be calculated for specific GHE configurations and borehole resitance, length and borehole/ length ratio. That is, the parameters for the units vary with each design. The data in this file are intended as examples/samples and may not represent actual designs.

The sample data has been calculated for a number of configurations:

\begin{itemize}
\item
  1 x 2 boreholes
\item
  4 x 4 boreholes
\item
  8 x 8 boreholes
\end{itemize}

Data is given for both `standard' grout (k = 0.744 W/m.K) and `thermally enhanced' grout (k = 1.471 W/m.K). The flow rate per borehole is .1514 kg/s. The pipe given is 0.75in. Dia. SDR11 HDPE. The fluid is water. The borehole/length ratio is 0.06 (76.2m/4.572m {[}300ft/15ft{]})
