\chapter{Creating Input Files}\label{creating-input-files}

EnergyPlus has several options for the user to create input files. One of the goals of EnergyPlus was to make a simple, readable input file for the program. More of this background and explanation is contained in the Interface Developer's Guide. Understanding the Input Data Dictionary (IDD) is the key to reading the input file. We have set some conventions for commenting the IDD so that the units, minimum, maximum, and other information. This changes some of the information that is shown in the Interface Developer's guide. The Energy+.idd (delivered with the install program) contains the most current information. In addition to the four methods for creating inputs described below, several other items are described that may assist you in getting the results you want from EnergyPlus in a timely manner.

Four methods (with the installed program) are available to create input files:

\begin{enumerate}
\def\labelenumi{\arabic{enumi})}
\item
  IDFEditor - this is a very simple, ``intelligent'' editor that reads the IDD and IDFs and allows creation/revision of IDF files. It can be run from a shortcut in the main EnergyPlus directory (created as part of the install) or directly from EP-Launch.
\item
  BLAST Translator - if you already have BLAST and/or BLAST input files, this program will produce the bulk of a translation to EnergyPlus for you. It generates a complete IDF file but does not include specifics for Systems or Plants. (It does include the System and Plant schedules that were in the BLAST deck). Many of the sample files included with the install started out as BLAST input files.
\item
  DOE-2 Translator - if you already have DOE-2.1e input files, this program will produce the bulk of a translation to EnergyPlus for you. It generates a IMF (input macro file) that must be run through the EnergyPlus Macro (EPMacro) program before it can be used by EnergyPlus.
\item
  Hand editing - for simple changes to an existing file (such as one of the sample files), you can hand edit a file using your knowledge of the IDD, comments in the IDF file, and a text editor such as NOTEPAD\textsuperscript{TM} (Wordpad\textsuperscript{TM} for large files). For creating HVAC simulations - the HVACtemplate objects provide a quick way to start at HVAC simulation.
\end{enumerate}
