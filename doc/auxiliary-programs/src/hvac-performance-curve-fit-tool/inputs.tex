\section{Inputs}\label{inputs}

First step in curve generation is to select appropriate parameters from the dropdown menu. These inputs define the DX Coil Type, Curve Type, Independent Variable and the Units type. The choices available for each input parameters are described in the following table. Once these input parameters are selected the tool read in the values and automatically populates labels for each of the independent and dependent variables. The labels guide users to enter the data for each variable in the corresponding worksheet input range. Two sets of input data are required for curve generation: Rated, and Performance Data.

\begin{longtable}{p{2in}|p{4in}}

\toprule
Input Parameter & Input Description \tabularnewline
\midrule
\endfirsthead

\toprule
Input Parameter & Input Description \tabularnewline
\midrule
\endhead 
DX Coil Type & \emph{Cooling}: Applicable for DX cooling coil single speed; 
               \emph{Heating}: Applicable for DX heating coil single speed; 
               \emph{Other}: Applicable for any equipment that use the three curve types
\tabularnewline
Independent Variables & Temperature and Flow 
\tabularnewline
Curve Types & \emph{Biquadratic}: Capacity and EIR as a function of temperature;
              \emph{Cubic}: Capacity and EIR as a function of flow fraction or temperature;
              \emph{Quadratic}: Capacity and EIR as a function of flow fraction
\tabularnewline
Units & \emph{IP}: Temperature in degF, Capacity in kBtu/h, Power in kW, and Flow in CFM;
        \emph{SI}: Temperature in degC, Capacity in kW, Power in kW, and Flow in m/s
\tabularnewline
Curve Object Name & This input is optional. This string is appended to the default curve object name, or if left blank the default curve object name will be displayed. A curve object is named is created by concatenation as follows: (User-specified curve object name) + ``DXCoilType'' + one of: [``CAPFTEMP'', ``CAPFFF'', ``EIRFTEMP'', ``EIRFFF'']\tabularnewline
\bottomrule
\end{longtable}
