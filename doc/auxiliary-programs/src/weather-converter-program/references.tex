\section{References}\label{references-002}

ASHRAE. 1985. \emph{Weather Year for Energy Calculations}. Atlanta: American Society of Heating, Refrigerating and Air-Conditioning Engineers, Inc.

ASHRAE. 1997. \emph{WYEC2 Weather Year for Energy Calculations 2, Toolkit and Data}, Atlanta: ASHRAE.

ASHRAE. 2001. \emph{International Weather for Energy Calculations (IWEC Weather Files) Users Manual and CD-ROM}, Atlanta: ASHRAE

ASHRAE. 2004. \emph{ANSI/ASHRAE Standard 90.2-2004}, ``Energy-Efficient Design of Low-Rise Residential Buildings,'' September 2004. Atlanta: American Society of Heating, Refrigerating, and Air-conditioning Engineers.

ASHRAE. 2004. \emph{ANSI/ASHRAE/IESNA Standard 90.1-2004}, ``Energy-Efficient Design of Buildings Except Low-Rise Residential,'' September 2004. Atlanta: American Society of Heating, Refrigerating, and Air-conditioning Engineers.

ASHRAE, 2009. Handbook of Fundamentals, Atlanta: American Society of Heating, Refrigerating, and Air-conditioning Engineers.

Briggs, Robert S., Robert G. Lucas, and Z. Todd Taylor. 2002. ``Climate Classification for Building Energy Codes and Standards: Part 1 - Development Process'' in \emph{ASHRAE Transactions 2002}, 109, Pt 1. Atlanta: ASHRAE.

Briggs, Robert S., Robert G. Lucas, and Z. Todd Taylor. 2002. ``Climate Classification for Building Energy Codes and Standards: Part 2 - Zone Definitions, Maps and Comparisons'' in \emph{ASHRAE Transactions}, 109, Pt 1. Atlanta: ASHRAE.

Buhl, W.F. 1998. DOE-2 Weather Processor, DOE2.1E Documentation Update, Berkeley: Lawrence Berkeley National Laboratory.

COMIS Weather Program, \url{http://www.byggforsk.no/hybvent/COMISweather.htm}

China Meteorological Bureau, Climate Information Center, Climate Data Office and Tsinghua University, Department of Building Science and Technology. 2005. China Standard Weather Data for Analyzing Building Thermal Conditions, April 2005. Beijing: China Building Industry Publishing House, ISBN 7-112-07273-3 (13228). \url{http://www.china-building.com.cn}.

Commission of the European Community. 1985. \emph{Test Reference Years,} Weather data sets for computer simulations of solar energy systems and energy consumption in buildings, CEC, DG XII. Brussels, Belgium: Commission of the European Community.

Crawley, Drury B., Linda K. Lawrie, Curtis O. Pedersen, Richard J. Liesen, Daniel E. Fisher, Richard K. Strand, Russell D. Taylor, Frederick C. Winkelmann, W.F. Buhl, A. Ender Erdem, and Y. Joe Huang. 1999. ``EnergyPlus, A New-Generation Building Energy Simulation Program,'' in \emph{Proceedings of Building Simulation '99}, Kyoto, Japan. IBPSA.

Crawley, Drury B. 1998. ``Which Weather Data Should You Use for Energy Simulations of Commercial Buildings?,'' \emph{ASHRAE Transactions}, pp.~498-515, Vol. 104, Pt. 2. Atlanta: ASHRAE. \url{http://energyplus.gov/pdfs/bibliography/whichweatherdatashouldyouuseforenergysimulations.pdf}

Crawley, Drury B., Jon Hand, and Linda K. Lawrie, 1999. ``Improving the Weather Information Available to Simulation Programs'', in \emph{Proceedings of Building Simulation '99}, Kyoto, Japan. September 1999. IBPSA.

Energy Simulation Research Unit. 1999. \url{http://www.strath.ac.uk/Departments/ESRU}

Janak, M. 1997. ``Coupling Building Energy and Lighting Simulation,'' in \emph{Proceedings of Building Simulation 97}, September 1997, Volume II pp 313-319, Prague, Czech Republic, IBPSA.

Köppen, W. 1931. Grundriss der Klimakunde. Berlin: Walter de Gruyter \& Co.

Kusuda, T., ``Earth Temperatures Beneath Five Different Surfaces'', Institute for Applied Technology, NBS Report 10-373, 1971, NBS, Washington DC 20234.

Kusuda, T., Least Squares Technique for the Analysis of Periodic Temperature of the Earth's Surface Region, NBS Journal of Research, Vol. 71C, Jan-Mar. 1967, pp 43-50.

National Instruments Corporation. 1999. \emph{LabVIEW User Manual}. Austin, Texas: National Instruments Corporation.

McDonald, Iain, and Paul Strachan. 1998. ``Practical Application of Uncertainty Analysis'' in \emph{Proceedings of EPIC 98: Second International Conference on Energy Performance and Indoor Climate in Buildings}, Lyon, France, 19-21 November 1998.

National Climatic Data Center (NCDC). 1976. \emph{Test Reference Year (TRY)}, Tape Reference Manual, TD-9706, September 1976. Asheville, North Carolina: National Climatic Data Center, U.S. Department of Commerce.

NCDC. 1981. \emph{Typical Meteorological Year User's Manual, TD-9734, Hourly Solar Radiation -- Surface Meteorological Observations}, May 1981. Asheville, North Carolina: National Climatic Data Center, U.S. Department of Commerce.

NCDC. 1981. \emph{Meteorological Observations}, May 1981. Asheville, North Carolina: National Climatic Data Center, U.S. Department of Commerce.

NCDC. 1993. \emph{Solar and Meteorological Surface Observation Network, 1961-1990, Version 1.0}, September 1993. Asheville, North Carolina: National Climatic Data Center, U.S. Department of Commerce.

National Renewable Energy Laboratory (NREL). 1995. \emph{User's Manual for TMY2s (Typical Meteorological Years)}, NREL/SP-463-7668, and \emph{TMY2s, Typical Meteorological Years Derived from the 1961-1990 National Solar Radiation Data Base}, June 1995, CD-ROM. Golden, Colorado: National Renewable Energy Laboratory.

\url{http://rredc.nrel.gov/solar/pubs/tmy2/}

Numerical Logics. 1999. Canadian Weather for Energy Calculations, Users Manual and CD-ROM. Downsview, Ontario: Environment Canada.

Oliver, John E. 1991. ``The History, Status and Future of Climatic Classification,'' in Physical Geography 1991, Vol 12, No. 3, pp.~231-251.

Perez R, Ineichen P, Maxwell E, Seals R and Zelenka. A 1992. Dynamic Global-to-Direct Irradiance Conversion Models. \emph{ASHRAE Transactions-Research Series},354-369.

Perez R, Ineichen P, Seals R, Michalsky J and Stewart R. 1990. Modeling daylight.availability and irradiance components from direct and global irradiance. \emph{Solar.Energy}44, 271-289.

University of Illinois. 1998. \emph{BLAST User's Guide.} Building Systems Laboratory, University of Illinois. Urbana, Illinois: University of Illinois, Department of Industrial and Mechanical Engineering.

Ward. G. 1996. \emph{Radiance.} Berkeley: Lawrence Berkeley National Laboratory.

Winkelmann, F.C., W.F. Buhl, B. Birdsall, A. E. Erdem, and K. Ellington. 1994. \emph{DOE-2.1E Supplement}, DE-940-11218. Lawrence Berkeley Laboratory, Berkeley, California. Springfield, Virginia: NTIS.

Zhang, Q. Y., Y. J. Huang. 2002. ``Development of Typical Year Weather Files for Chinese Locations'', in ASHRAE Transactions, Volume 108, Part 2.
