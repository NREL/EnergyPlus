\section{Weather Data Availability}\label{weather-data-availability}

Typically, acquisition of weather data has been a user's burden. Though this will remain the case in many instances for EnergyPlus users, the EnergyPlus team has been successful in making a wealth of US, Canadian and International data available to our users. To summarize, the weather data for 2092 locations is available at the EnergyPlus web site: \href{http://www.energyplus.gov}{www.energyplus.gov}

The details are shown in Table~\ref{table:summary-of-downloadable-weather-data-by-type}. Summary of Downloadable Weather Data by Type. This data has been selected with the energy simulation user in mind. All the data (as well as the statistical reports - described later in this document) are downloadable for individual locations.

% table 1
\begin{longtable}[c]{>{\raggedright}p{3in}llll}

\caption{Comparison of E/E with ESP-r/DOE-2/BLAST Weather Data Formats \label{table:comparison-of-ee-with-esp-rdoe-2blast-weather}} \tabularnewline
\toprule 
Data Element & DOE-2 & BLAST & ESP-r & E/E \tabularnewline
\midrule
\endfirsthead

\caption[]{Comparison of E/E with ESP-r/DOE-2/BLAST Weather Data Formats} \tabularnewline
\toprule 
Data Element & DOE-2 & BLAST & ESP-r & E/E \tabularnewline
\midrule
\endhead

Location (name, latitude, longitude, elevation, time zone) & X & X & X & X \tabularnewline
Data source &  &  &  & X \tabularnewline
Commentary &  &  & X & X \tabularnewline
Design conditions &  &  &  & X \tabularnewline
Typical/extreme periods &  &  & X & X \tabularnewline
Data periods &  &  &  & X \tabularnewline
Holiday/Daylight Saving &  & X &  & X \tabularnewline
Solar Angles/Equation of Time Hours &  & X &  &  \tabularnewline
Degree Days &  & X &  & X \tabularnewline
Year & X & X & X & X \tabularnewline
Month & X & X & X & X \tabularnewline
Day & X & X & X & X \tabularnewline
Hour & X & X & X & X \tabularnewline
Minute &  &  &  & X \tabularnewline
Data source and uncertainty flags &  &  &  & X \tabularnewline
Dry bulb temperature & X & X & X & X \tabularnewline
Wet bulb temperature & X & X &  &  \tabularnewline
Dew point temperature & X &  &  & X \tabularnewline
Atmospheric station pressure & X & X &  & X \tabularnewline
Humidity ratio & X & X &  &  \tabularnewline
Relative humidity &  &  & X & X \tabularnewline
Enthalpy & X &  &  &  \tabularnewline
Density & X &  &  &  \tabularnewline
Wind Speed & X & X & X & X \tabularnewline
Wind Direction & X & X & X & X \tabularnewline
Infrared Sky Temperature &  & X &  & X \tabularnewline
Solar Radiation (global, normal, diffuse) & X & X & X & X \tabularnewline
Illuminance (global, normal, diffuse) &  &  &  & X \tabularnewline
Sky cover (cloud amount) & X &  &  & X \tabularnewline
Opaque sky cover &  &  &  & X \tabularnewline
Visibility &  &  &  & X \tabularnewline
Ceiling height &  &  &  & X \tabularnewline
Clearness (monthly) & X &  &  &  \tabularnewline
"Undisturbed" Ground temperatures (monthly) & X &  &  & X \tabularnewline
Present weather observation and codes (rain, snow) &  & X &  & X \tabularnewline
Precipitable water &  &  &  & X \tabularnewline
Aerosol optical depth &  &  &  & X \tabularnewline
Snow depth &  &  &  & X \tabularnewline
Days since last snowfall &  &  &  & X \tabularnewline
Albedo &  &  &  & X \tabularnewline
Liquid Precipitation Depth &  &  &  & X \tabularnewline
Liquid Precipitation Quantity &  &  &  & X \tabularnewline
\bottomrule
\end{longtable}
