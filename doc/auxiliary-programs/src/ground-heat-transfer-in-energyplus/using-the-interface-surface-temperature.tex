\section{Using the Interface Surface Temperature Results in EnergyPlus}\label{using-the-interface-surface-temperature-results-in-energyplus}

If the objects are placed in the normal EnergyPlus input IDF file using the ``GroundHeatTransfer:Basement:'' prefix, then the values resulting from the Basement preprocessor will be automatically included in the simulation at run time. The surfaces can reference these values using Outside Boundary Conditions of:

\begin{itemize}
\item
  GroundBasementPreprocessorAverageWall
\item
  GroundBasementPreprocessorAverageFloor
\item
  GroundBasementPreprocessorUpperWall
\item
  GroundBasementPreprocessorLowerWall
\end{itemize}

The objects that support this include:

\begin{itemize}
\item
  BuildingSurface:Detailed
\item
  Wall:Detailed
\item
  RoofCeiling:Detailed
\item
  Floor:Detailed
\end{itemize}

The monthly results from the basement program are used with the SurfaceProperty:OtherSideCoefficients object in EnergyPlus. The idd corresponding to that object is shown below. The two applicable fields are N2 and A2. N2 should have the value 1.0. It will be modified by a schedule that is named in A2, and that temperature will be used on the outside of the surface specified as having the OtherSideCoeffcients named in A1.

\begin{lstlisting}

SurfaceProperty:OtherSideCoefficients,
  \memo This object sets the other side conditions for a surface in a variety of ways.
  \min-fields 8
  A1, \field Name
  \required-field
  \reference OSCNames
  \reference OutFaceEnvNames
  N1, \field Combined Convective/Radiative Film Coefficient
  \required-field
  \type real
  \note if>0, this field becomes the exterior convective/radiative film coefficient
  \note and the other fields are used to calculate the outdoor air temperature
  \note then exterior surface temperature based on outdoor air and specified coefficient
  \note if< = 0, then remaining fields calculate the outside surface temperature
  \note following fields are used in the equation:
  \note SurfTemp = N7*TempZone + N4*OutdoorDry-bulb + N2*N3 + GroundTemp*N5 + WindSpeed*N6*OutdoorDry-bulb
  N2, \field Constant Temperature
  \units C
  \type real
  \default 0
  \note This parameter will be overwritten by the values from the Constant Temperature Schedule Name (below) if one is present
  N3, \field Constant Temperature Coefficient
  \note This coefficient is used even with a Schedule.  It should normally be 1.0 in that case
  \default 1
  N4, \field External Dry-Bulb Temperature Coefficient
  \type real
  \default 0
  N5, \field Ground Temperature Coefficient
  \type real
  \default 0
  N6, \field Wind Speed Coefficient
  \type real
  \default 0
  N7, \field Zone Air Temperature Coefficient
  \type real
  \default 0
  A2; \field Constant Temperature Schedule Name
  \note Name of schedule for values of constant temperature.
  \note Schedule values replace any value specified in the field Constant Temperature.
  \type object-list
  \object-list ScheduleNames
\end{lstlisting}

A sample idf for this object is shown below.

\begin{lstlisting}

SurfaceProperty:OtherSideCoefficients, OSCCoef:Zn005:Wall003, !- OSC Name
  !  Example input for second ground temperature
  0.0000000E+00,    !- OSC SurfFilmCoef
  0.000000    ,    !- OSC Temp
  1.000000    ,    !- OSC Temp Coef
  0.000000    ,    !- OSC dry-bulb
  0.000000    ,    !- OSC GrndTemp
  0.000000    ,    !- OSC WindSpdCoeff
  0.000000    ,    !- OSC ZoneAirTemp
  GroundTempSched;  !  Name of schedule defining additional ground temperature.
\end{lstlisting}

The OSC object can be repeated for as many outside temperatures as needed. A more detailed explanation of how to use this object is contained in the next section, and an example object is output by the program in the file EPObjects.txt.

\subsection{Multiple Ground Temperatures}\label{multiple-ground-temperatures}

These three objects show how the OtherSideCoefficients object can be used to provide extra ground temperatures for surfaces exposed to different ground temperatures.

Here is the surface description. Note that the OutsideFaceEnvironment is specified as OtherSideCoeff.

\begin{lstlisting}

BuildingSurface:Detailed,
  OSCTest:South Wall,      !- User Supplied Surface Name
  Wall,                    !- Surface Type
  Exterior,                !- Construction Name of the Surface
  OSCTest,                 !- Zone
  OtherSideCoefficients,   !- Outside Boundary Condition
  ExampleOSC,              !- Outside Boundary Condition Object
  NoSun  ,                 !- Sun Exposure
  NoWind  ,                !- Wind Exposure
  0.5,                     !- View Factor to Ground
  4,                       !- Number of Vertices
  0,0,0,                   !- Vertex 1 X,Y,Z-coordinates {m}
  6.096,0,0,               !- Vertex 2 X,Y,Z-coordinates {m}
  6.096,0,4.572,           !- Vertex 3 X,Y,Z-coordinates {m}
  0,0,4.572;               !- Vertex 4 X,Y,Z-coordinates {m}
\end{lstlisting}

The OtherSideCoefficients object has to supply the basic form of the environment. Note that the name corresponds to thee name in the Surface object. This object also supplies the name of a schedule that will provide the monthly ground temperature values.

\begin{lstlisting}

SurfaceProperty:OtherSideCoefficients,
  ExampleOSC,              !- OtherSideCoeff Name
  0,                       !- Combined convective/radiative film coefficient
  1,                       !- User selected Constant Temperature {C}
  1,                       !- Coefficient modifying the user selected constant temperature
  0,                       !- Coefficient modifying the external dry bulb temperature
  0,                       !- Coefficient modifying the ground temperature
  0,                       !- Coefficient modifying the wind speed term (s/m)
  0,                       !- Coefficient modifying the zone air temperature part of the equation
  GroundTempCompactSched;  !- Schedule Name for values of "const" temperature. Schedule values replace N2.
\end{lstlisting}

The schedule named in the last field of the OtherSideCoefficients object must be supplied. In compact schedule format it would appear as shown below. Again, objects for each of the surface temperatures are produced by the program and output in the file EPObjects.txt.

\begin{lstlisting}

Schedule:Compact,
  GroundTempCompactSched,  !- Name
  Temperature ,            !- ScheduleType
  Through: 1/31,           !- Complex Field \#1
  For:AllDays,             !- Complex Field \#2
  Until: 24:00,            !- Complex Field \#3
  16,                      !- Complex Field \#4
  Through: 2/28,           !- Complex Field \#5
  For:AllDays,             !- Complex Field \#6
  Until: 24:00,            !- Complex Field \#7
  17,                      !- Complex Field \#8
  Through: 3/31,           !- Complex Field \#9
  For:AllDays,             !- Complex Field \#10
  Until: 24:00,            !- Complex Field \#11
  18,                      !- Complex Field \#12
  Through: 4/30,           !- Complex Field \#13
  For:AllDays,             !- Complex Field \#14
  Until: 24:00,            !- Complex Field \#15
  19,                      !- Complex Field \#16
  Through: 5/31,           !- Complex Field \#17
  For:AllDays,             !- Complex Field \#18
  Until: 24:00,            !- Complex Field \#19
  20,                      !- Complex Field \#20
  Through: 6/30,           !- Complex Field \#21
  For:AllDays,             !- Complex Field \#22
  Until: 24:00,            !- Complex Field \#23
  20,                      !- Complex Field \#24
  Through: 7/31,           !- Complex Field \#25
  For:AllDays,             !- Complex Field \#26
  Until: 24:00,            !- Complex Field \#27
  20,                      !- Complex Field \#28
  Through: 8/31,           !- Complex Field \#29
  For:AllDays,             !- Complex Field \#30
  Until: 24:00,            !- Complex Field \#31
  19,                      !- Complex Field \#32
  Through: 9/30,           !- Complex Field \#33
  For:AllDays,             !- Complex Field \#34
  Until: 24:00,            !- Complex Field \#35
  18,                      !- Complex Field \#36
  Through: 10/31,          !- Complex Field \#37
  For:AllDays,             !- Complex Field \#38
  Until: 24:00,            !- Complex Field \#39
  17,                      !- Complex Field \#40
  Through: 11/30,          !- Complex Field \#41
  For:AllDays,             !- Complex Field \#42
  Until: 24:00,            !- Complex Field \#43
  16,                      !- Complex Field \#44
  Through: 12/31,          !- Complex Field \#45
  For:AllDays,             !- Complex Field \#46
  Until: 24:00,            !- Complex Field \#47
  16;                      !- Complex Field \#48
\end{lstlisting}
