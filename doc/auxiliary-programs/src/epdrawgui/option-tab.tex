\section{Option Tab}\label{option-tab}

The Option Tab, shown in Figure~\ref{fig:epdrawgui-options-tab}, contains an additional option to control some complex drawings as well as ways to select the DXF file viewer and view DXF files.

The ``View DXF File'' button is used to select a previously created DXF file and view it with the DXF file viewer. The DXF viewer is usually found automatically but if this function does not work, you may want to select the DXF file viewer manually using the Select DXF Viewer button.

The ``Select DXF Viewer'' button allows you to manually select the program used to display DXF files. Normally, it is not necessary to use this function since the DXF file viewer program is automatically detected but if the wrong file viewer is automatically detected or no file viewer is detected, this button can be used to select the viewer program.

For IDF files that contain surfaces with more than four sides, the options under ``Polygons with 5+ Sides'' can affect the way the drawing is shown. Polygons with \textgreater{}4 sides do not display with the DXF 3DFACE command used for surfaces of 3 and 4 sides which subsequently will display very nicely as a ``solid'' in many DXF viewers.

Thus there are four options which the user may choose to display \textgreater{}4 sided polygons.

\begin{itemize}
\tightlist
\item
  Attempt Triangulation
\end{itemize}

This option attempts simple triangulation for the polygon (\textgreater{}4 sides) surfaces. This triangulation will show in the wireframe views but will appear as a solid face in 3D views. This triangulation is only for drawing purposes and does not affect the simulations in any way. The triangle algorithm is not perfect and warnings do result when the software cannot triangulate a surface. If unable to triangulate simply, a warning error is generated to the .EPDerr file.

\begin{itemize}
\tightlist
\item
  Thick Polyline
\end{itemize}

With this option, the \textgreater{}4 sided polygon appears as a thicker line in all views of the building model. This option creates a `thick' line at the border of the polygon (\textgreater{}4 sides) surfaces. It will look like a hole in the drawing with a thicker edge. This thick border shows in wireframe as well as 3D views and can be confusing, due to overlap with other surfaces.

\begin{itemize}
\tightlist
\item
  Regular Polyline
\end{itemize}

With this option, the \textgreater{}4 sided polygon appears as a wire frame line in all views of the building model. This option creates a `regular' polyline for all polygon (\textgreater{}4 sides) surfaces. It will look like a hole in the drawing. Also, it will look the same in both wireframe and 3D views.

\begin{itemize}
\tightlist
\item
  Wireframe
\end{itemize}

This option creates a wireframe drawing (all lines) for all surfaces. All surfaces will appear as lines in both wireframe and 3D views.

Note that the EPDrawGUI program only processes building and shading surfaces. It does not process daylighting reference points though the similar option in the EnergyPlus program (Report, Surfaces, DXF;) does show the daylighting reference points (but not illuminance map points) in the DXF view.
