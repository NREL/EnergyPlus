\subsection{CoilSystem:Cooling:Water}\label{coilsystemcoolingwater}

The \hyperref[coilsystemcoolingwater]{CoilSystem:Cooling:Water} object is a coil system wrapper for water cooling coils. Valid water cooling coils are: \hyperref[coilcoolingwater]{Coil:Cooling:Water}, \hyperref[coilcoolingwaterdetailedgeometry]{Coil:Cooling:Water:DetailedGeometry} and \hyperref[coilsystemcoolingwaterheatexchangerassisted]{CoilSystem:Cooling:Water:HeatExchangerAssisted}. This coil system allows users to model the three water cooling coils in airloop main and outdoor air branches. Also this object is designed to model water-side economizers for free pre-cooling when the water-side of the coil is placed on the demand side a condenser loop. As a water-side economizer the \textit{CoilSystem:Cooling:Water} object is placed upstream of packaged DX systems or chilled water main cooling coils. The coil system as a water-side economizer provides free pre-cooling when the condition is favorable. Any remaining system cooling load not met by the water side economizer is provided either by a DX or chilled water cooling coil placed downstream of the water-side economizer coil. The \textit{CoilSystem:Cooling:Water} object does not require \textit{Controller:WaterCoil} and relies on a built-in controller. This coil system uses setpoint based control that varies coil entering water (fluid) mass flow rate to meet user specified temperature or humidity setpoint. The coil system as a water-side economizer can be placed on airloop main or outdoor air system branch. 

Figure~\ref{fig:water-side-economizer-coil-system-upstream-of-packaged-dx-system} below shows packaged DX system serving multiple zones and a coil system water cooling object placed upstream of the packaged system. The water-side of the coil system is connected to the demand side of a condenser or plant loop. This coil system configuration provides free pre-cooling when the condition is favorable to operate and there is cooling or dehumidification demand. In this configuration the packaged DX system can be \textit{AirloopHVAC:UnitarySystem} object.

\begin{figure}[hbtp] % fig 141
\centering
\includegraphics[width=0.9\textwidth, height=0.9\textheight, keepaspectratio=true]{media/image141.png}
\caption{Water Side Economizer Coil System Upstream of Packaged DX System \protect \label{fig:water-side-economizer-coil-system-upstream-of-packaged-dx-system}}
\end{figure}

Figure~\ref{fig:water-side-economizer-coil-system-in-outdoor-air-system} below shows a packaged DX system serving multiple zones and a coil system water cooling object placed in the outdoor air system. The water-side of the coil system is connected to the demand side of a condenser or plant loop.  This coil provides free cooling to the outdoor air stream when the condition is favorable to operate and there is pre-cooling demand.

\begin{figure}[hbtp] % fig 142
\centering
\includegraphics[width=0.9\textwidth, height=0.9\textheight, keepaspectratio=true]{media/image142.png}
\caption{Water Side Economizer Coil System In Outdoor Air System \protect \label{fig:water-side-economizer-coil-system-in-outdoor-air-system}}
\end{figure}


\subsubsection{Water Side Economizer Mode}\label{water-side-economizer-mode}

The coil system cooling object to operate the coil entering water (fluid) temperature must be less than the coil entering air temperature minus the user specified temperature offset value. The second requirement is that the coil system entering air temperature must be greater than the coil system air outlet node (control node) setpoint temperature, i.e., there has to be a cooling or dehumidification demand. Built in coil system controller strives to meet either temperature or humidity ratio setpoint at the coil system air outlet node by varying the cold water (fluid) mass flow rate.



\subsubsection{Inputs}\label{inputs}

\paragraph{Field: Name}\label{field-name-06}

This alpha field contains the identifying name for the coil system cooling water object. Any reference to this coil system by another object will use this name.

\paragraph{Field: Air Inlet Node Name}\label{field-air-inlet-node-name-01}

This alpha field contains the coil system cooling water object air inlet node name.

\paragraph{Field: Air Outlet Node Name}\label{field-air-outlet-node-name-01}

This alpha field contains the coil system cooling water object air outlet node name.

\paragraph{Field: Availability Schedule Name}

This alpha field contains the schedule name which contains information on the availability of the coil system cooling water object to operate. A schedule value equal to 0 denotes that the coil system must be off for that time period. A value greater than 0 denotes that the coil system is available to operate during that time period. This schedule may be used to completely disable the coil system as required. If this field is left blank, the schedule has a value of 1.

\paragraph{Field: Cooling Coil Object Type}\label{field-cooling-coil-object-type-01}

This alpha field contains the identifying type of cooling coil specified in the coil system cooling water object. Valid choices for this field are:
\begin{itemize}
\item Coil:Cooling:Water
\item Coil:Cooling:Water:DetailedGeometry
\item CoilSystem:Cooling:Water:HeatExchangerAssisted
\end{itemize}


\paragraph{Field: Cooling Coil Name}\label{field-cooling-coil-name-01}

This alpha field contains the identifying name given to the coil system water cooling coil.

\paragraph{Field: Dehumidification Control Type}

This alpha field contains the type of dehumidification control. The following options are valid for this field:

\textbf{None} - meet sensible load only, no active dehumidification control. The default is None.

\textbf{Multimode} - activate water coil and meet sensible load. If no sensible load exists, and Run on Latent Load = Yes, and a latent load exists, the coil will operate to meet the latent load. If the latent load cannot be met the heat exchanger will be activated. IF Run on Latent Load = No, the heat exchanger will always be active. This control mode either switches the coil mode or allows the heat exchanger to be turned on and off based on the zone dehumidification requirements. Valid only with cooling coil type CoilSystem:Cooling:Water:HeatExchangerAssisted.

\textbf{CoolReheat} - cool beyond the dry-bulb temperature set point as required to meet the high humidity setpoint. If cooling coil type = CoilSystem:Cooling:Water:HeatExchangerAssisted, then the heat exchanger is assumed to always transfer energy between the cooling coil’s inlet and outlet airstreams when the cooling coil is operating.


For the dehumidification control modes, the maximum humidity setpoint on the Sensor Node is used. This must be set using a ZoneControl:Humidistat object. When extra dehumidification is required, the system may not be able to meet the humidity setpoint if its full capacity is not adequate. If the dehumidification control type is specified as CoolReheat, then the system may require reheat coil type and name elsewhere. Although the reheat coil is required only when CoolReheat is selected, the optional reheat coil may be present for any of the allowed Dehumidification Control Types.

Valid humidity setpoint managers include:
\begin{itemize}
\item SetpointManager:SingleZone:Humidity:Maximum
\item SetpointManager:MultiZone:Humidity:Maximum
\item SetpointManager:MultiZone:MaximumHumidity:Average
\item SetpointManager:OutdoorAirPretreat
\end{itemize}

\paragraph{Field: Run On Sensible Load}\label{field-run-on-sensible-load-01}

This alpha field specifies if the coil system will operate to meet a sensible load calculated from the air flow rates through the coil system, coil system entering air temperature and coil outlet node (control node) temperature setpoint. There are two valid choices, Yes or No. If Yes, coil will run if there is a sensible load. If No, coil will not run if there is only a sensible load. The default is Yes.


\paragraph{Field: Run on Latent Load}\label{field-run-on-latent-load-01}

This alpha field specifies if the coil will operate to meet a latent load calculated from the air flow rate through the coil system, coil system entering air humidity ratio and coil system outlet node (control node) maximum humidity ratio setpoint. There are two valid choices, Yes or No. If Yes, the coil will run if there is a latent load. If both a sensible and latent load exist, the system will operate to maintain the temperature set point. When only a latent load exists, the system will operate to meet the maximum humidity ratio set point and requires the use of a heating coil and heating coil outlet node air temperature set point manager downstream of this cooling coil to maintain the temperature set point. If No, the coil will not run if there is only a latent load. The default is No.


\paragraph{Field: Minimum Air To Water Temperature Offset}\label{field-minimum-air-to-water-temperature-offset}

The coil system will turn ON as required when coil entering air temperature is above coil entering water temperature by more than the amount of this offset. To model a waterside economizer connected to condenser loop an increased offset as desired. Default is 0.


Following is an example input for a coil system cooling water.

\begin{lstlisting}

  CoilSystem:Cooling:Water,
    Unitary-Free-Cooling,    !- Name
    Mixed Air Node,          !- Air Inlet Node Name
    FreeClgCoil OutletNode,  !- Air Outlet Node Name
    ,                        !- Availability Schedule Name
    Coil:Cooling:Water,      !- Cooling Coil Object Type
    Free Cooling Coil,       !- Cooling Coil Name
    CoolReheat,              !- Dehumidification Control Type
    Yes,                     !- Run on Sensible Load
    Yes,                     !- Run on Latent Load
    3.0;                     !- Minimum Air To Water Temperature Offset
	
  CoilSystem:Cooling:Water,
    Unitary-Free-Cooling,    !- Name
    Mixed Air Node,          !- Air Inlet Node Name
    FreeClgCoil OutletNode,  !- Air Outlet Node Name
    ,                        !- Availability Schedule Name
    Coil:Cooling:Water:DetailedGeometry, !- Cooling Coil Object Type
    Free Cooling Coil,       !- Cooling Coil Name
    CoolReheat,              !- Dehumidification Control Type
    Yes,                     !- Run on Sensible Load
    Yes,                     !- Run on Latent Load
    3.0;                     !- Minimum Air To Water Temperature Offset

  CoilSystem:Cooling:Water,
    Unitary-Free-Cooling,    !- Name
    Mixed Air Node,          !- Air Inlet Node Name
    DXACFurnace InletNode,   !- Air Outlet Node Name
    ,                        !- Availability Schedule Name
    CoilSystem:Cooling:Water:HeatExchangerAssisted, !- Cooling Coil Object Type
    Free Cooling Coil HXA,   !- Cooling Coil Name
    CoolReheat,              !- Dehumidification Control Type
    Yes,                     !- Run on Sensible Load
    Yes,                     !- Run on Latent Load
    3.0;                     !- Minimum Air To Water Temperature Offset
	
\end{lstlisting}


\subsubsection{Outputs}\label{outputs}

Following are the list of possible output variables from this coil model:

\begin{itemize}
\item HVAC, Average, Coil System Water Part Load Ratio {[]}
\item HVAC, Average, Coil System Water Total Cooling Rate {[}W{]}
\item HVAC, Average, Coil System Water Sensible Cooling Rate {[}W{]}
\item HVAC, Average, Coil System Water Latent Cooling Rate {[}W{]}
\item HVAC, Average, Coil System Water Control Status {[]}
\end{itemize}


\paragraph{Coil System Water Part Load Ratio {[]}}\label{coil-system-water-part-load-ratio}

This output variable is the ratio of the sensible cooling load to the current full cooling capacity of the coil system. This variable reports the average load met as a fraction of the full coil capacity during the system timestep. If the ratio is 0.0, then there is no cooling load, else if the ratio is 1.0, then the load met is equal to the coil system full capacity.

\paragraph{Cooling Coil Total Cooling Rate {[}W{]}}

This output field is the total (sensible + latent) cooling rate of the coil system from the supply or outdoor air in Watts. This value is calculated using the enthalpy difference of the coil system outlet air and inlet air streams and the air mass flow rate through the coil system. This value is reported for each HVAC system timestep being simulated and is an average for the timestep.

\paragraph{Cooling Coil Sensible Cooling Rate {[}W{]}}

This output field reports the moist air sensible cooling rate of the coil system from the supply or outdoor air system. This value is calculated using the enthalpy difference of the coil system outlet air and inlet air streams at a constant humidity ratio, and the air mass flow rate through the coil system. This value is reported for each HVAC system timestep simulated and is an average for the timestep.

\paragraph{Cooling Coil Latent Cooling Rate {[}W{]}}

This output field is the latent cooling (dehumidification) rate of the coil system in Watts. This value is calculated as the difference between the total cooling rate and the sensible cooling rate provided by the coil system. This value is reported for each HVAC system timestep being simulated and is an averaged for the timestep.

\paragraph{Coil System Water Control Status {[]}}\label{coil-system-water-control-status}

This output field indicates whether the coil system is favorable to operate or not. Control status value of 1 indicates that the condition is favorable for the coil system to operate. Control status value of 0 indicates the condition is not favorable the coil system to operate. The control status is determined from the coil entering air temperature, coil entering water temperatures and user specified temperature offset. If the coil entering air temperature is above the coil entering water temperatures by more than the specified temperature offset, then the control status is set to 1, else it is set to 0. This value is reported for each HVAC system timestep being simulated, and the control status is an average for the timestep.

