%% LyX 2.3.1-1 created this file.  For more info, see http://www.lyx.org/.
%% Do not edit unless you really know what you are doing.
\documentclass[english]{article}
\usepackage[T1]{fontenc}
\usepackage[letterpaper]{geometry}
\geometry{verbose,tmargin=0.5in,bmargin=0.5in,lmargin=0.5in,rmargin=0.5in,headheight=0.5in,headsep=0.5in,footskip=0.5in}
\usepackage{babel}
\usepackage{textcomp}
\usepackage{graphicx}
\usepackage[unicode=true,pdfusetitle,
 bookmarks=true,bookmarksnumbered=false,bookmarksopen=false,
 breaklinks=false,pdfborder={0 0 1},backref=false,colorlinks=false]
 {hyperref}

\makeatletter
%%%%%%%%%%%%%%%%%%%%%%%%%%%%%% Textclass specific LaTeX commands.
\newenvironment{lyxcode}
	{\par\begin{list}{}{
		\setlength{\rightmargin}{\leftmargin}
		\setlength{\listparindent}{0pt}% needed for AMS classes
		\raggedright
		\setlength{\itemsep}{0pt}
		\setlength{\parsep}{0pt}
		\normalfont\ttfamily}%
	 \item[]}
	{\end{list}}

\makeatother

\begin{document}
\title{EnergyPlus Essentials}
\date{January 21, 2019}

\maketitle
<\textcompwordmark <TO DO: GRAMMARLY, ADDITIONAL GRAPHICS, >\textcompwordmark >

\section{Introduction}

The following background on building energy modeling (BEM) will provide
a foundation before learning the essentials of EnergyPlus. 

\section*{What is BEM?}

According to the \href{https://www.bemlibrary.com/index.php/owners-managers/introduction/what-bem/}{BEM Library}:
\begin{quotation}
``Building Energy Modeling (BEM) is the practice of using computer-based
simulation software to perform a detailed analysis of a building\textquoteright s
energy use and energy-using systems. The simulation software works
by enacting a mathematical model that provides an approximate representation
of the building. 

``BEM includes whole-building simulation as well as detailed component
analysis utilizing specialized software tools that address specific
concerns, such as moisture transfer through building materials, daylighting,
indoor air quality, natural ventilation, and occupant comfort.

``BEM offers an alternative approach that encourages customized,
integrated design solutions, which offer deeper savings. Using BEM
to compare energy-efficiency options directs design decisions prior
to construction. It also guides existing building projects to optimize
operation or explore retrofit opportunities.''
\end{quotation}
Some other terms that are used to describe the same topic are:
\begin{itemize}
\item Building Simulation
\item Building Performance Simulation
\item Building Energy Simulation
\item Building Performance Modeling
\end{itemize}
Somewhat confusing, the term ``model'' can refer to both the specific
building being described and analyzed as well as referencing the software
for implementing a specific component of the building such as a wall
or chiller.

To learn more about building energy modeling consider reviewing the
following resources:
\begin{itemize}
\item \href{https://www.ashrae.org/technical-resources/ashrae-handbook/description-2017-ashrae-handbook-fundamentals}{2017 ASHRAE Handbook - Fundamentals }Chapter
19 Energy Estimating and Modeling Methods
\item \href{https://www.bemlibrary.com/}{BEM Libary}
\item \href{http://www.ibpsa.org/?page_id=695}{IBPSA}, \href{https://www.ibpsa.us/videos/all}{IBPSA-USA},
and \href{https://www.youtube.com/results?search_query=building+energy+modeling}{YouTube}
videos
\item And numerous \href{https://www.amazon.com/s/ref=nb_sb_noss_2?url=search-alias\%3Daps&field-keywords=building+energy+modeling}{books}
\end{itemize}
Building enery modeling is necessary to understand the overall energy
consumption in buildings because the energy flows in buildings can
be very complicated. The heat flow through walls is determined not
just by the area and the temperatures but the mass characteristics
of the walls may result in heat flowing out of a wall several hours
after it has flowed in. The constantly changing temperature on the
outside of the wall and the less frequent but often large step changes
in thermostat setpoints between day and night operation for the inside
of the building make the direction of the heat flow vary over time.
Solar enery being absorbed on the exterior wall and roof surfaces
as well as entering windows varies by the sun position for each period
of time over the day and over the year. Heat sources within the building
including people, office and other equipment, and lighting include
both heat that immediately impacts the air within (convective heat)
and heat that is absorbed by the sufaces and released slowly over
time (radiant heat). Lighting may be controlled by a sensor that turns
it on or off or changes its intensity based on the amount of natural
daylight that is entering the space through windows and skylights
making the heat generated from the lighting system change over time.
Air conditioning and heating systems come in a huge number of configurations
and each one can be used with many different controls configurations
based on the temperature or other conditions within each space in
the building and ultimately their operation accounts for a large portion
of the energy consumed in a building. For these reasons and more,
what might first appear as something that can be calcuated with just
a few formulas in a spreadsheet is instead a software program and,
in the case of EnergyPlus, with over 500,000 lines of code. 

\section*{Questions that BEM can answer}

The most common questions that BEM can answer are:
\begin{itemize}
\item If my building was made or operated differently, how would the energy
consumption change?
\item Does my building comply with a building energy code or standard?
\item What kind of rating or how many points can I get in a environmental
certification program?
\item Is my building operating as it was designed to?
\item What is a good target for energy consumption for my building?
\end{itemize}
These questions can take place at different times during the life-cycle
of a building from before schematic design all through the rest of
the design process, and into the operation of the building

\subsection*{The Design Process and BEM }

Building energy modeling can be used throughout the design process
for a new building or when considering updates to existing buildings.
It can also be used later in the life-cycle of a building related
to comparing actual operation of the building with predicted operation.
ASHRAE \href{https://www.techstreet.com/ashrae/standards/ashrae-209-2018?gateway_code=ashrae&product_id=2010483}{Standard 209}
titled ``Energy Simulation Aided Design for Buildings Except Low-Rise
Residential Buildings'' is a critical document to in how to apply
building energy modeling. It describes a methodology to apply building
energy modeling to the design and was created to define reliable and
consistent procedures that advance the use of timely energy modeling
to quanifty the impact of design decision s at the point in time which
they are made. The standard includes different \textquotedblleft Modeling
Cycles\textquotedblright{} for different stages of using building
energy modeling during the life-cycle of the building:
\begin{itemize}
\item Modeling Cycle \#1 -- Simple Box Modeling
\item Modeling Cycle \#2 -- Conceptual Design Modeling
\item Modeling Cycle \#3 -- Load Reduction Modeling
\item Modeling Cycle \#4 -- HVAC System Selection Modeling
\item Modeling Cycle \#5 -- Design Refinement
\item Modeling Cycle \#6 -- Design Integration and Optimization
\item Modeling Cycle \#7 -- Energy Simulated-Aided Value Engineering
\item Modeling Cycle \#8 -- As-Designed Energy Performance
\item Modeling Cycle \#9 -- Change Orders
\item Modeling Cycle \#10 -- As-Built Energy Performance
\item Modeling Cycle \#11 -- Postoccupancy Energy Comparison
\end{itemize}
In addition it has information about how to integrate climate and
site analysis, benchmarking, energy charrettes, the energy performance
goals of the owners project requirements in to the design process
when using building energy modeling.

\subsection*{EnergyPlus Capabilities}

According to the \href{https://energyplus.net/}{energyplus.net web site}
(as of January 2019):
\begin{quotation}
``EnergyPlus\texttrademark{} is a whole building energy simulation
program that engineers, architects, and researchers use to model both
energy consumption---for heating, cooling, ventilation, lighting
and plug and process loads---and water use in buildings. Some of
the notable features and capabilities of EnergyPlus include:
\end{quotation}
\begin{itemize}
\item Integrated, simultaneous solution of thermal zone conditions and HVAC
system response that does not assume that the HVAC system can meet
zone loads and can simulate un-conditioned and under-conditioned spaces.
\item Heat balance-based solution of radiant and convective effects that
produce surface temperatures thermal comfort and condensation calculations.
\item Sub-hourly, user-definable time steps for interaction between thermal
zones and the environment; with automatically varied time steps for
interactions between thermal zones and HVAC systems. These allow EnergyPlus
to model systems with fast dynamics while also trading off simulation
speed for precision.
\item Combined heat and mass transfer model that accounts for air movement
between zones.
\item Advanced fenestration models including controllable window blinds,
electrochromic glazings, and layer-by-layer heat balances that calculate
solar energy absorbed by window panes.
\item Illuminance and glare calculations for reporting visual comfort and
driving lighting controls.
\item Component-based HVAC that supports both standard and novel system
configurations.
\item A large number of built-in HVAC and lighting control strategies and
an extensible runtime scripting system for user-defined control.
\item Functional Mockup Interface import and export for co-simulation with
other engines.
\item Standard summary and detailed output reports as well as user definable
reports with selectable time-resolution from annual to sub-hourly,
all with energy source multipliers.''
\end{itemize}
In addition:
\begin{itemize}
\item ASCII text based weather, input, and output files that include hourly
or sub-hourly environmental conditions, and standard and user definable
reports, respectively 
\item Transient heat conduction through building elements such as walls,
roofs, floors, etc. using conduction transfer functions
\item Thermal comfort models based on activity, inside dry bulb, humidity,
etc.
\item Anisotropic sky model for improved calculation of diffuse solar on
tilted surfaces
\item Atmospheric pollution calculations that predict CO2, SOx, NOx, CO,
particulate matter, and hydrocarbon production for both on site and
remote energy conversion
\item System and plant output is allowed to directly impact the building
thermal response rather than calculating all loads first, then simulating
systems and plants which allows the designer to more accurately investigate
the effect of undersizing fans and equipment and what impact that
might have on the thermal comfort of occupants within the building 
\item EnergyPlus runs on Windows, MacOS, and Linux computers. 
\end{itemize}
Integration of Loads, Systems, and Plants: One of the strong points
of EnergyPlus is the integration of all aspects of the simulation---loads,
systems, and plants. System and plant output is allowed to directly
impact the building thermal response rather than calculating all loads
first, then simulating systems and plants. The simulation is coupled
allowing the designer to more accurately investigate the effect of
undersizing fans and equipment and what impact that might have on
the thermal comfort of occupants within the building. 

<need to balance overview with enough details to help people, I think
a list of what can be modeled might be more helpful>

\subsection*{Open Source}

\href{https://energyplus.net/}{EnergyPlus}is an \href{https://opensource.org/}{Open Source}
program so all the \href{https://github.com/NREL/EnergyPlus}{source code}
is available to inspect and modify. If you are interested in how calculations
are performed and the \href{https://energyplus.net/documentation}{Engineering Reference}
does not provide enough details to you, you can review the source
code itself. \href{https://github.com/NREL/EnergyPlus/wiki/BuildingEnergyPlus}{Instructions to build the code}
(compile the source code into an executable application) are available
in source code repository wiki if you see something that needs to
be enhanced or fixed. 

\subsection*{Brief History}

EnergyPlus has been under development since 1997 and was first released
in 2001. EnergyPlus has its roots in both the BLAST and DOE--2 programs.
BLAST (Building Loads Analysis and System Thermodynamics) and DOE--2
were both developed and released in the late 1970s and early 1980s
as energy and load simulation tools. BLAST was developed by \href{https://www.erdc.usace.army.mil/Locations/CERL/}{Construction Engineering Researrch Laboratory (CERL)}
and \href{https://illinois.edu/}{University of Illinois} which DOE-2
was developed by \href{https://www.lbl.gov/}{Berkeley Lab} and many
others. Their intended audience is a design engineer or architect
that wishes to size appropriate HVAC equipment, develop retrofit studies
for life cycling cost analyses, optimize energy performance, etc.
Born out of concerns driven by the energy crisis of the early 1970s
and recognition that building energy consumption is a major component
of the American energy usage statistics, the two programs attempted
to solve the same problem from two slightly different perspectives.
Both programs had their merits and shortcomings, their supporters
and detractors, and solid user bases both nationally and internationally.
Originally written in Fortran, in 2014 it was converted to C++. It
was developed as a simulation engine and many graphical user interfaces
utilize it.

\subsection*{Documentation and Other Resources}

The EnergyPlus documentation is currently included the installation
in the ``Documenation'' folder as PDFs. It is also available as
\href{https://energyplus.net/documentation}{PDFs online} and as \href{https://bigladdersoftware.com/epx/docs/}{browsable HTML}.
The documentation includes:
\begin{itemize}
\item Input and Output Reference: Contains a thorough description of the
various input and output files related to EnergyPlus, the format of
these files, and how the files interact and interrelate.
\item Output Details, Examples and Data Sets: Contains details on output
from EnergyPlus and specific examples. It also addresses the reference
data sets that are included.
\item Auxiliary Programs:~Contains information for the auxiliary programs
that are part of the EnergyPlus package. For example, this document
contains the user manual for the Weather Converter program, descriptions
on using Ground Heat Transfer auxiliary programs with EnergyPlus,
HVAC Template descriptions, and other assorted topics.
\item The Engineering Reference: Provides more in-depth knowledge into the
theoretical basis behind the various calculations contained in the
program including algorithm descriptions.
\item EMS Application Guide: Contains information useful to use the advanced
feature of EnergyPlus: Energy Management System tweaks. The Erl language
is described and examples for use are given.
\item External Interface Application Guide: Contains information specific
to using the external interface feature of EnergyPlus to connect other
simulation systems.
\item Plant Application Guide: Details the methods for simulating real chilled
and hot water plant systems within EnergyPlus.
\item Using EnergyPlus for Compliance Guide: Contains information specific
to using EnergyPlus in Compliance and Standard Rating systems.
\item External Interface(s) Application Guide: Contains information about
external interfaces (through the Building Controls Virtual Test Bed
link) to EnergyPlus.
\item Tips \& Tricks for Using EnergyPlus: Contains short tips and tricks
for using various parts of EnergyPlus. <MAYBE THIS SHOULD BE ELIMINATED
AND PORTIONS INCLUDED IN THIS DOCUMENT>
\end{itemize}
To learn more about how to use EnergyPlus, the example files are a
great resource. The example files are in the directory ExampleFiles
and 

\section{The EnergyPlus Ecosystem }

\subsection*{Current Interfaces }

EnergyPlus is often used directly using the text file input (IDF or
epJSON) and various output file formats along with the utilities that
come with the installation package. More information on that can be
found in the Using EnergyPlus section. In addition, EnergyPlus is
often the simulation engine for graphical user interfaces. To see
comprehensive list, see the \href{https://www.buildingenergysoftwaretools.com/}{BEST (Building Energy Software Tools) Directory}
that is operated by \href{https://www.ibpsa.us/}{IBPSA-USA}. At the
time of writing, many people use EnergyPlus inside an existing graphical
user interface such as \href{https://www.openstudio.net/}{OpenStudio},
\href{https://www.designbuilder.com/}{DesignBuilder}, \href{https://d-alchemy.com/products/simergy}{Simergy},
\href{https://bigladdersoftware.com/projects/euclid/}{Euclid}, \href{https://www.ladybug.tools/}{Ladybug},
\href{https://www.autodesk.com/products/insight/overview}{Insight},
\href{https://www.bentley.com/en/products/brands/aecosim}{AECOsim},
and \href{https://bigladdersoftware.com/projects/euclid/}{Euclid}.
Other interfaces are also available. 

\subsection*{Approaches to Implement Measures }

In the terminology used within the building energy modeling industry
``measures,'' sometimes called energy conservation measures (ECMs)
or energy efficiency measures (EEMs), are when alternatives configurations
of a building are considered and simulated and compared with the original
building model. Measures include added wall or roof insulation, lower
internal lighting power, higher efficiency heating and cooling equipment,
and many others. When using EnergyPlus within a graphical user interface,
that interface often provides a way to implement various measures.
When not working within a graphically user interface, users have several
options for implementing measures:

For a specific building input file, a copy of the input file can be
made and then modified to reflect the measure. This is an easy approach
initially but since the original building model might change, it means
making the change in the file that reflects the measure. When many
measures are being considered, this duplicative editing can be inefficient
and prone to errors. In addition, the measure cannot be easily applied
to a different building energy model without again editing another
set of files duplicating the original effort. Due to this, it is very
common for some type of scripting to be used so that measures can
be applied to the original building model and other models of other
buildings reliably and quickly. Scripting in this approach means to
apply some type of programming language to the task of modifying files.

EnergyPlus includes, with the installation package, two methods of
implementing measures EP-Macro and the ParametricPreprocessor. EP-Macro
is similar to the C langauge pre-processor and allows for portions
of the file to be included or excluded, for portions of other files
to be included when desired, and for specific entries that are normally
fixed values to change programmatically. EP-Macro is documented int
he AuxiliaryPrograms documentation. A file with lines starting ``\#\#''
or with the ``imf'' file extension are likely to be a file using
EP-Macro files. 

The EnergyPlus ParametricPreprocessor uses special input objects in
EnergyPlus to set values for any field in any other input object for
a series of simple options <OBJECTS HAVE NOT BEEN DEFINED YET>. This
is briefly described in the AuxiliaryPrograms documentation and more
details are present in the InputOutputReference under the section
on ``Parametric Objects.'' The ParametricPreprocessor approach is
very simple but limited in the flexibility that it affords. It is
suitable for implementing measures related to internal loads, constructions,
and simple efficiency changes but is probably not the appropriate
tool for more complicated measures.

Another approach to writing measures and using measures written by
others is to use \href{https://www.openstudio.net/}{OpenStudio} and
its \href{http://nrel.github.io/OpenStudio-user-documentation/}{documentation}.
OpenStudio is closely aligned with EnergyPlus but has a different
data model and file format. It is an application programming interface
for EnergyPlus and some graphical user interfaces are using it to
work with EnergyPlus. It has two categories of measures: native OpenStudio
measures and EnergyPlus measures. Both types of measures are usually
written in the Ruby programming language. Many measures are already
available for OpenStudio and can be found in the \href{https://bcl.nrel.gov/}{Building Component Library.}It
is an open source project. 

Using the Python scripting language, \href{https://github.com/santoshphilip/eppy}{Eppy}
allows access to any of the input objects in EnergyPlus as if they
were native Python objects. It is very flexible and an especially
good choice if you are already familiar with Python. It is also an
open source project.

Another option based on the Ruby programming language is \href{https://bigladdersoftware.com/projects/modelkit/}{Modelkit}
which similar to EP-Macro consists of embedding scripting within normal
EnergyPlus input files but since it is using Ruby has a great deal
more flexility.

A \href{https://www.ashrae.org/File\%20Library/Conferences/Specialty\%20Conferences/2018\%20Building\%20Performance\%20Analysis\%20Conference\%20and\%20SimBuild/Papers/C043.pdf}{paper}
that describes these approaches to scripting and more was made during
the 2018 ASHRAE/IBPSA-USA Building Performance Conference and SimBuild.

\subsection*{Parametric Analysis Tools }

When just implementing a simple measure or even a series of measures
is not enough, a parametric analysis tool may be appropriate. These
tools allow the exploration of throughout the range of variables (such
as the thickness of insulation in the roof or the efficiency of a
boiler) to see the impacts of optimization. While any of the approaches
to implement measures described previously may be used for parametric
analysis, a few specific tools have been developed. \href{http://nrel.github.io/OpenStudio-user-documentation/reference/parametric_studies/}{OpenStudio Parametric Analysis}
tool works with OpenStudio measures and allows the simulation of a
large number of options using cloud computing services and is part
of the \href{https://www.openstudio.net/}{OpenStudio} suite. JEPlus
is a simulation manager for parametrics that includes optimization
and leverages both the EP-Macro style of implementing measures as
well as Eppy for implementing measures in the Python programming language.
\href{https://simulationresearch.lbl.gov/GO/summary.html}{GenOpt}
is a Java based generic optimization program that can use EnergyPlus.
\href{https://beopt.nrel.gov/}{BEopt} is a building optimization
tool focused on residential applications.

\subsection*{Weather Files \label{subsec:Weather-Files}}

Weather files are available for many locations throughout the world.
Finding the right file that represents the weather in your specific
location can sometimes be a challenge. Often the closest weather location
is the best one to choose but sometimes a site that is further away
may actually have the most similar weather. This is especially the
case in terrain that varies in elevation or when near large bodies
of water. Many weather files are available from both public and private
sources. The \href{https://energyplus.net/weather}{EnergyPlus weather file }
web site has many weather files with \href{https://energyplus.net/weather/sources}{sources described}.
That web site also has a \href{https://energyplus.net/weather/simulation}{page with links}
to many other sites that provide weather files.

\subsection*{Co-simulation and Linked Software}

<\textcompwordmark <NEED CONTENT>\textcompwordmark >

\subsection*{Interoperability }

<\textcompwordmark <NEED CONTENT>\textcompwordmark >

\subsection*{Example Files }

Many hundreds of example files come with EnergyPlus are in the ExampleFiles
folder from the installation. The \textbackslash ExampleFiles\textbackslash ExampleFiles.html
lists each one and includes the name, a description (scroll all the
way to the right) and lots of information such as the floor area and
whether certain types of input objects are included in the file. Often
searching through this file is a good way to find the proper example
file to learn about a feature. Another method is the \textbackslash ExampleFiles\textbackslash ExampleFiles-ObjectsLink.html
file which lists every type of input object that EnergyPlus uses and
then the first three files that use that input object. It is possible
that many other files also use a particular input object so if the
first three files do not help, a text search of files in the ExampleFiles
folder may find some more. 

\subsection*{Getting Help \label{subsec:Getting-Help}}

Several resources are available for getting help when using EnergyPlus:
\begin{itemize}
\item \href{https://unmethours.com/questions/}{UnmetHours}
\item \href{http://energyplus.helpserve.com/}{EnergyPlus Helpdesk}
\item \href{https://groups.yahoo.com/neo/groups/EnergyPlus_Support/info}{EnergyPlus\_support mailing list}
\item \href{https://buildingenergysoftwaretools.com/?capabilities=Support+Services&keys=EnergyPlus}{Several organizations provide paid support}
\end{itemize}
Please do not post questions as issues on the EnergyPlus Github website.
Of course, if you are using a graphical user interface with EnergyPlus,
the vendor will provide direct support.

After reviewing this document and other pertinant documents that come
with EnergyPlus like the InputOutputReference, if additional training
is required, several sources are available:
\begin{itemize}
\item \href{https://www.youtube.com/results?search_query=energyplus}{YouTube}
\item University \href{https://energyplus.net/support}{course} teaching
materials
\item Several \href{https://www.buildingenergysoftwaretools.com/?capabilities=Training+Services&keys=EnergyPlus}{organizations}
provide paid training
\end{itemize}
In addition, if you are using a graphical user interface, the vendor
probably also provides training.

\section{Using EnergyPlus }

\subsection*{Installing EnergyPlus}

Please see the EnergyPlus QuickStart Guide for instructions on how
to install EnergyPlus for your system.

\subsection*{Running EnergyPlus}

EnergyPlus is a simulation program designed for modeling buildings
with all their associated heating, ventilating, and air conditioning
equipment. EnergyPlus is a simulation engine: it was designed to be
an element within a system of programs that would include a graphical
user interface to describe the building. However, it can be run stand
alone without such an interface. There are several option for performing
a simulation with EnergyPlus:
\begin{itemize}
\item Graphical user interface
\item Command line
\item EP-Launch
\end{itemize}
In each case, a building model will be simulated in combination with
a weather file from the appropriate building location. 

\subsubsection*{Graphical User Interface}

When running an EnergyPlus simulation within a graphical user interface,
the exact method will vary depending on the specific program being
used. You should read the documentation for that software to understand
how to perfom a simulation. 

\subsubsection*{Command Line}

EnergyPlus can be used as a command line tool within a Terminal window
in Linux or MacOS or with the CMD prompt or PowerShell window under
Windows. Basic usage using the command line approach is will documented
in the QuickStart Guide. To learn more about the command line mode,
you can type:
\begin{verbatim}
energyplus --help 
\end{verbatim}
when in the Energyplus folder. This will give the following display
of options:
\begin{verbatim}
EnergyPlus, Version 9.0.1-bb7ca4f0da
Usage: energyplus [options] [input-file]
Options:
  -a, --annual                 Force annual simulation
  -c, --convert                Output IDF->epJSON or epJSON->IDF, dependent on
                               input file type
  -d, --output-directory ARG   Output directory path (default: current
                               directory)
  -D, --design-day             Force design-day-only simulation
  -h, --help                   Display help information
  -i, --idd ARG                Input data dictionary path (default: Energy+.idd
                               in executable directory)
  -m, --epmacro                Run EPMacro prior to simulation
  -p, --output-prefix ARG      Prefix for output file names (default: eplus)
  -r, --readvars               Run ReadVarsESO after simulation
  -s, --output-suffix ARG      Suffix style for output file names (default: L)
                                  L: Legacy (e.g., eplustbl.csv)
                                  C: Capital (e.g., eplusTable.csv)
                                  D: Dash (e.g., eplus-table.csv)
  -v, --version                Display version information
  -w, --weather ARG            Weather file path (default: in.epw in current
                               directory)
  -x, --expandobjects          Run ExpandObjects prior to simulation
Example: energyplus -w weather.epw -r input.idf
\end{verbatim}
EnergyPlus can be run by specifying a number of options followed by
the path to the input file. The file itself is usually in IDF (Input
Data File) format or epJSON format, but it may also be in IMF (Input
Macro File) format to be run with EPMacro using the -{}-epmacro option.

Each option has a short form (a single-character preceded by a single
dash, e.g., \textquotedbl -h\textquotedbl ) and a long form (a more
descriptive string of characters preceded by double dashes, e.g.,
\textquotedbl -{}-help\textquotedbl ).

The options generally fall into four categories:
\begin{itemize}
\item Basic informational switches: 
\begin{itemize}
\item help 
\item version 
\end{itemize}
\item Input/output control flags: 
\begin{itemize}
\item idd 
\item weather 
\item output-directory 
\item output-prefix 
\item output-suffix 
\end{itemize}
\item Pre- and post-processing switches: 
\begin{itemize}
\item epmacro 
\item expandobjects 
\item readvars 
\end{itemize}
\item Input override switches: 
\begin{itemize}
\item annual 
\item design-day
\end{itemize}
\end{itemize}
Several of these options are commonly used including the weather,
output-prefix, expandobjects, and readvars options.

\paragraph*{Examples}

Using a custom IDD file:
\begin{verbatim*}
energyplus -i custom.idd -w weather.epw input.idf
\end{verbatim*}
Pre-processing using EPMacro and ExpandObjects:
\begin{verbatim*}
energyplus -w weather.epw -m -x input.imf
\end{verbatim*}
Forcing design-day only simulations:
\begin{verbatim*}
energyplus -D input.idf
\end{verbatim*}
Giving all output files the prefix being the same as the input file
(building.idf) and placing them in a directory called output:
\begin{verbatim*}
energyplus -w weather -p building -d output building.idf
\end{verbatim*}

\paragraph*{Legacy Mode}

The command line interface is a new feature as of EnergyPlus 8.3.
Prior to version 8.3, the EnergyPlus executable took no command line
arguments, and instead expected the IDD (Input Data Dictionary) file
and the IDF files to be located in the current working directory and
named Energy+.idd and in.idf respectively. If a weather file was required
by the simulation, then an in.epw file was also required in the same
directory. This behavior is still respected if no arguments are passed
on the command line.

\subsubsection*{EP-Launch}

For users that want a simple way of selecting files and running EnergyPlus,
EP-Launch provides this and more. In addition, EP-Launch can help
open a text editor for the input and output files, open a spreadsheet
for the postprocessor results files, a web browser for the tabular
results file, and start up a viewer for the selected drawing file.
There are two different versions of EP-Launch currently part of the
EnergyPlus system. 

The main screen of EP-Launch 2 is shown below:

\includegraphics{eplaunch2}

It is a Windows program only. EP-Launch 2 is included in the EnergyPlus
installation package when installing on Windows so no additional steps
are needed to run it. It is located in the main ``root'' folder
of EnergyPlus, usually a folder named EnergyPlusVx-x-x, where the
x's are the version number. 

In 2018, EP-Launch 3 was developed and its main screen is shown below:

\includegraphics{eplaunch3}

EP-Launch 3 is not part of the EnergyPlus installation package and
needs to be installed separately. It is also open source and is available
from \href{https://github.com/NREL/EP-Launch}{GitHub} and it is documented
on \href{https://ep-launch.readthedocs.io/en/}{readthedocs} or in
the docs folder on GitHub. EP-Launch works on Windows, MacOS, and
Linux systems and is written in Python.

While both EP-Launch 2 and EP-Launch 3 do many of the same functions
the interface is quite different. For now, EP-Launch 2 allows groups
of files to be run together and has access to some utilities that
the newer version does not. EP-Launch works across multiple platforms
and is a built from the ground up to be flexible and extensible so
that individuals can make their own workflows that run what ever programs
they need to run.

\subsection*{IDF and JSON syntax}

EnergyPlus has two different input file formats that can be used to
describe the building and system that is simulated. The file extensions
for the two formats are IDF and epJSON. For both input files, the
numeric inputs are in SI units (International System of Units often
called metric units). 

\subsubsection*{IDF}

The legacy file format is a text based format that describes each
input object in series. The name of the type of input object starts
the input object and each value for each field follows separated by
commas. The end of the input object is indicated by a semi-colon.
Comments are indicated by a exclamation point ``!'' and anything
after this is ignored by EnergyPlus. Commonly, an input object is
spread over many lines in the file with one value for each field per
line. The names of each field are not required but are usually shown
after the value and the comma or semicolon as a special comment using
``!-'' as an indicator. The input objects can be in any order. A
small piece of an IDF file 
\begin{lyxcode}
Building,

~~Simple~One~Zone,~~~!-~Name

~~0,~~~~~~~~~~~~~~~~~!-~North~Axis~\{deg\}

~~Suburbs,~~~~~~~~~~~!-~Terrain

~~0.04,~~~~~~~~~~~~~~!-~Loads~Convergence~Tolerance~Value

~~0.004,~~~~~~~~~~~~~!-~Temperature~Convergence~Tolerance~Value~\{deltaC\}

~~MinimalShadowing,~~!-~Solar~Distribution

~~30,~~~~~~~~~~~~~~~~!-~Maximum~Number~of~Warmup~Days

~~6;~~~~~~~~~~~~~~~~~!-~Minimum~Number~of~Warmup~Days
\end{lyxcode}
The details of this example input object are not important but the
use of commas, exclamation points, and the closing semi-colon are
important. The IDF format is currently the most commonly used format
throughout the EnergyPlus ecosystem of utilities and GUIs. The list
of possible input objects and fields is documented in the Energy+.idd
file.

A variation on the IDF file format is the IMF file format which includes
macros that can be used for parametric analysis or file management
called EP-Macros. To learn more about macros see the Input Macros
chapter of the AuxiliaryPrograms document.

\subsubsection*{epJSON}

A new file format based on the industry standard \href{https://www.json.org/}{JSON}
format most often used to transmit extra data to web applications.
It is a text based file format. The JSON format has wide industy usage
and is supported in just about every modern programming langauge.
It is a field-value style format using brackets and colons to indicate
the heirarchy and commas to separate each field and value pair. The
input objects must appear grouped by the type of input object. The
list of possible input objects and fields is documented in the Energy+.schema.epJSON
file which uses \href{http://json-schema.org/}{json-schema}. The
same input object shown above in IDF format is shown below in epJSON
format:
\begin{lyxcode}
\{

~~~~\textquotedbl Building\textquotedbl :~\{

~~~~~~~~\textquotedbl Simple~One~Zone:~\{

~~~~~~~~~~~~\textquotedbl idf\_max\_extensible\_fields\textquotedbl :~0,

~~~~~~~~~~~~\textquotedbl idf\_max\_fields\textquotedbl :~8,

~~~~~~~~~~~~\textquotedbl idf\_order\textquotedbl :~3,

~~~~~~~~~~~~\textquotedbl loads\_convergence\_tolerance\_value\textquotedbl :~0.04,

~~~~~~~~~~~~\textquotedbl maximum\_number\_of\_warmup\_days\textquotedbl :~30,

~~~~~~~~~~~~\textquotedbl minimum\_number\_of\_warmup\_days\textquotedbl :~6,

~~~~~~~~~~~~\textquotedbl north\_axis\textquotedbl :~0,

~~~~~~~~~~~~\textquotedbl solar\_distribution\textquotedbl :~\textquotedbl MinimalShadowing\textquotedbl ,

~~~~~~~~~~~~\textquotedbl temperature\_convergence\_tolerance\_value\textquotedbl :~0.004,

~~~~~~~~~~~~\textquotedbl terrain\textquotedbl :~\textquotedbl Suburbs\textquotedbl{}

~~~~~~~~\}

~~~~\}

\}
\end{lyxcode}
While the IDF and epJSON file formats are quite different they contain
the same information and either may be used. In general, if producing
EnergyPlus input files using a programming langauge, the epJSON format
might make more sense while, at this point, if producing IDF files
using a GUI, they are likely to use the IDF format. EnergyPlus, when
used on the command line, can convert from IDF to epJSON and from
epJSON to IDF using the -c or -{}-convert option.

\subsection*{Creating and Editing Input Files}

Since both the IDF and epJSON file formats are text formats, a simple
text editor may be used to edit them. Even if not regularly used,
a good text editor is an important application to have when working
with EnergyPlus. There are many different \href{https://en.wikipedia.org/wiki/Comparison_of_text_editors}{text editors}
and a few have special features related to the IDF format such as
syntax highlighting including \href{https://github.com/bigladder/atom-language-energyplus}{Atom},
\href{https://github.com/jmarrec/notepad}{Notepad++}, and \href{http://energyplus.helpserve.com/knowledgebase/article/View/102/47/ultraedit-syntax-highlighting-file---v80}{UltraEdit}.

Another editing choice for IDF file is the IDF Editor which comes
with EnergyPlus in the \textbackslash PreProcess\textbackslash IDFEditor
directory. It is a Windows only program and is was not designed to
run on Linux or MacOS. It is specially designed for editing IDF files
and includes many features to simplify the process. It performs unit
conversions so either SI (metric) or IP (inch-pound) units can be
used. The main screen of the IDF editor is shown below. Full details
of the IDF Editor can be found in the Auxiliary Programs document
under the ``Creating Input Files'' section.

\includegraphics[width=6in]{idfeditor}

A cross platform program called \href{https://bitbucket.org/mattdoiron/idfplus/src/default/}{IDF+}
is currently being developed independantly and will have many of the
same features as the IDF Editor when complete.

\subsection*{Run-Check-Edit Repeat \label{subsec:Run-Check-Edit-Repeat}}

For most building energy modeling projects, whether assisting in early
design, refining a design, selecting a control method, or calibrating
an existing building, the use of EnergyPlus will be part of a repeating
cycle. The cycle will probably be in the form of:
\begin{itemize}
\item Running an EnergyPlus input file
\item Checking error and other output files
\item Fixing the input file
\item Repeat
\end{itemize}
Don't expect that an initial model is ever correct, it is probably
not. Initially, errors are likely to exist. The .ERR file should be
the first file checked each time EnergyPlus is run. The .ERR file
has several levels of messages: 
\begin{itemize}
\item Warning
\item Severe
\item Fatal 
\end{itemize}
A fatal error means that EnergyPlus has stopped during the simulation
and the input file needs to be fixed before the simulation can be
run to completion. Fatal errors should be the first thing fixed. Some
Fatal messages reference previous Severe messages so in that case
those should be fixed. Since the entire simulation was not performed,
it is likely that once the fatal errors are fixed that new Severe
and Warning messages will be shown. After all Fatal messages are eliminated
you should work on Severe messages. They should be fixed also. Finally,
Warning messages should be reviewed. Often Warning messages are informative
and point out unusual configurations, conditions, or choices. If what
is being described by the Warning message is as intended by you, then
the Warning message can be ignored. More often, the Warning message
points out something that is not as intended and should be fixed or
addressed. Since the .ERR file is a text file, you can usually keep
it open in a text editor program. Many (but not all) text editor programs
will detect that the .ERR file has been updated after each EnergyPlus
simulation and let you load the most recent version.

The next files to be examined are ones that show output results from
the simulation. Either the tabular output file (usually an HTML file
see Output:Table:SummaryReports and OutputControl:Table:Style) or
CSV file (see Output:Variable and Output:Meter) should be examined
depending on what you want to look at. Upon examination of the output
results, it is very likely that an aspect of the building and its
systems is not behaving as expected. With an input file representing
many thousands of assumptions, some assumptions made by you or as
a default of EnergyPlus is likely to be incorrect. Revising the EnergyPlus
input file to address this may cause new issues to be shown in the
.ERR file so it should \emph{always} be examined after each change.

To speed the process of running the simulations, you may want to only
run a design day (see SizingPeriod:DesignDay) or a subset of the year
(see RunPeriod). This approach speeds up the simulation time itself
and if used, please remember to recheck the .ERR file when running
an annual simulation for the first time.

\subsection*{Common Input Objects}

\subsection*{Key Concepts}

The following sections highlight some key concepts in EnergyPlus

\subsubsection*{Everything Included}

One principal that EnergyPlus uses is that (almost) everything is
specified in the input file. This means that instead of referencing
an external library for materials, schedules, equipment performance,
etc., the input objects that fully describe those items should be
included directly in the input file. The DataSets folder distributed
with EnergyPlus contains these kind of details and to use them, the
input objects should be copied into the input file that you are developing.
This approach does make the file include more specification than you
might be used to, and typically results in a large input file, but
you will have the assurance of knowing that all the inputs related
to your building are in the input file you have developed.

\subsubsection*{Wall Thickness}

Exterior and interior walls in real buildings have thickness as specified
on building plans by detailed cross sections. For EnergyPlus, the
Construction input object is made up of a list of names for the Material
input objects that make up the wall or roof or floor. Each material
is shown with a thickness along with the conductivity, density, specific
heat and other factors. These thicknesses should match the the thicknesses
shown in the detailed cross sections. But when it comes to specifying
the walls themselves in three dimensional space, the walls should
be entered assuming zero thickness. Once each surface has been placed,
changing the material thickness will have no impact on zone volume,
ceiling height, floor area, shading, or daylighting. For most modern
buildings the choice of inside vs outside vs centerline should have
little impact on results so many modelers just pick one a let the
volumes be off a little. Using centerlines throughout the model splits
the difference. Or some modelers use outer edges for exterior walls
and then use centerlines for interior walls. If you are modeling a
very thick wall, such as an old stone building, then you also have
thermal mass considerations. If you use the outside edges there will
be too much mass, inside will be too little. Again, centerline will
split the difference and will be very close to the correct amount
of thermal mass (possibly losing some corner mass).

\subsubsection*{Zones Are Not the Same as Rooms}

A zone, sometimes called a ``thermal zone'' is a theoretical construct
that usually describes a group of rooms that can be treated as a single
thermal entity. One way to thi

\subsubsection*{One Construction Per Surface}

Only one type of construction can be associated with each surface
so if the top half of a wall is made up of a different construction
than the bottom half of the wall, the top half and the bottom half
each need to be represented by separate surfaces.

\subsubsection*{Always Plan Ahead}

Some preliminary steps will facilitate the construction of your input
file. EnergyPlus requires some information in specified, externally
available formats; other information may require some lead time to
obtain. The following checklist should be completed before you start
to construct your input file.
\begin{itemize}
\item Obtain location and design climate information for the city in which
your building is located. If possible, use one of the weather files
available for your weather period run.
\item Obtain sufficient building construction information to allow specification
of overall building geometry and surface constructions (including
exterior walls, interior walls, partitions, floors, ceilings, roofs,
windows and doors).
\item Obtain sufficient building use information to allow specification
of the lighting and other equipment (e.g. electric, gas, etc.) and
the number of people in each area of the building.
\item Obtain sufficient building thermostatic control information to allow
specification of the temperature control strategy for each area of
the building.
\item Obtain sufficient HVAC operation information to allow specification
and scheduling of the fan systems.
\item Obtain sufficient central plant information to allow specification
and scheduling of the boilers, chillers and other plant equipment.
\end{itemize}

\subsection*{What Are All These Folders?}

The installation of EnergyPlus includes a lot of different files in
different folders:

\includegraphics{energyplusfolder}

Many of these folders include valuable resources for using and learning
EnergyPlus. The main folder includes the EnergyPlus executable which
can be used on the command line and EP-Launch 2, a program that makes
it easier to use EnergyPlus and the Energy+.IDD that describes each
possible EnergyPlus input object and the default, minimum, maximum,
and options for each field within each input object. The Documentation
folder includes this document as well as the InputOutputReference,
EngineeringReference, AuxiliaryPrograms, OutputDetailsAndExamples
which are very imporant to understand. If you haven't looked through
the documentation yet, take a few minutes and get familiar with them.
The DataSets and MacroDataSets folders include files containing libraries
of input objects that may be useful in constructing your own input
files. The ASHRAE\_2005\_HOF\_Materials.idf and WindowConstructs.idf
files, for example, will help with defining walls and windows. The
ExampleFiles folder includes a huge number of example files that are
indexed in the two HYML files in that folder or can be searched through
using most text editors. The Preprocess and PostProcess folders include
many utilities that can be used directly or as part of EP-Launch that
can aid in the setting up input files or or extracting or converting
results. The WeatherData folder includes a small sample of the many
weather files that are available. For other weather files, please
see the previous section on \ref{subsec:Weather-Files}. 

\subsection*{What Are All These Output Files?}

When running EnergyPlus using EP-Launch or from the command line,
depending on the options selected, many different output files may
be generated. The file extensions and file suffixes (added to the
original file name prior to the file extension are shown below:
\begin{itemize}
\item AUDIT -- input file echo with input processor errors and warnings
\item BND -- HVAC system node and component connection details
\item DBG -- output from the debug command
\item DXF -- drawing file in AutoCAD DXF format
\item EDD -- Energy Management System details
\item EIO -- additional EnergyPlus results
\item END - a single line synopsis of the simulation
\item EPMIDF -- clean idf file after EP-Macro processing
\item EPMDET -- EP-Macro detailed output with errors and warnings
\item ERR -- list of errors and warnings
\item ESO -- raw report variable output
\item MTD -- list of meter component variables
\item MTR -- raw report meter output
\item RDD -- list of output variables available from the run
\item MDD -- list of output meters available from the run 
\item SHD -- output related to shading
\item SLN -- output from \textquotedblleft report, surfaces, lines\textquotedblright{}
\item SQL - sqlite3 oiutput database format
\item SSZ -- system sizing details in comma, tab or space delimited format
\item ZSZ -- zone sizing details in comma, tab or space delimited format
\item CSV, TAB, or TXT -- tabulated results in comma, tab or space delimited
format (generated by the ReadVarsESO postprocessor) 
\item METER.CSV, METER.TAB, or METER.CSV File -- tabulated meter report
in comma, tab or space delimited format (generated by the ReadVarsESO
postprocessor)
\item MAP -- daylighting illuminance map
\item DFS - daylighting factors report
\item Screen.CSV - window screen transmittance map report
\item TABLE.HTML, TABLE.TXT, TABLE.TAB, TABLE.CSV, TABLE.XML -- tabulated
report of bin and monthly data in comma, tab or space delimited or
HTML format or XML format
\item RVAUDIT - output from the ReadVarsESO post processing program
\item SVG - HVAC Diagram related to the arrangement of HVAC components
\item SCI - surface cost information repor
\item WRL -- drawing file in VRML (Virtual Reality Markup Language) format
\item Delight IN - DElight input generated from EnergyPlus processed input
\item Delight OUT -- Detailed DElight output
\item Delight ELDMP -- DElight reference point illuminance per time step
\item Delight DFDMP -- DElight warning and error messages
\item EXPIDF -- Expanded IDF when using HVACTemplate input objects
\item Group Error -- combined error files for a group run
\item VCpErr -- Transition program error file
\item Proc.CSV -- Simple statistiscs generated from CSVProc
\end{itemize}
Almost of these output files are documented in the Output Files chapter
of the OutputDetailsAndExamples document.

Don't be intimindated by the long list of files, you can do a lot
in EnergyPlus with just the IDF input file, the TABLE.HTML file and
the ERR file. The building description, the detecting and solving
of errors, and the most common primary outputs are found between these
three files. Starting with these three files and branching out to
others as needed is a good strategy for using EnergyPlus.

\subsection*{Versions and Updating}

When using the IDF input file format with EnergyPlus, each release
(which are generally twice per year in March and September) is likely
to have small changes to the file format. Included with EnergyPlus
are a number of ways to update files so that they are compatible with
the release. Each method ultimately uses the TransitionVx-x-x-ToVx-x-x.exe
files that are located in the Preprocess\textbackslash IDFVersionUpdater
folder. The ways to update your IDF files are:
\begin{itemize}
\item IDFVersionUpdater program (shown below) is included in the installation
and works on multiple platforms. It is located in the Preprocess\textbackslash IDFVersionUpdater
folder. It can convert from EnergyPlus 7.2 to the most recent version
and even older versions can be converted if the proper files are downloaded
from the \href{http://energyplus.helpserve.com/Knowledgebase/List/Index/46/converting-older-version-files}{helpdesk}.
It can also update a group of files. It is documented in the Chapter
titled ``Using Older Version Input Files - Tranision'' in the AuxiliaryPrograms
document.
\end{itemize}
\includegraphics{IDFVersionUpdater}
\begin{itemize}
\item EP-Launch 2 - The windows only program that comes with the EnergyPlus
installation can update a single file from the just previous version
of EnergyPlus only by using the File..Transition command.
\item EP-Launch 3 - The program for Windows, Linux, and MacOS can update
a single file across multiple versions using the Transition workflow.
\item Command line Transition - This allows updating files using the command
line such as the Terminal for MacOS and Linux or the CMD or PowerShell
for Windows. It is documented in the Chapter titled ``Using Older
Version Input Files - Tranision'' in the AuxiliaryPrograms document.
\end{itemize}

\subsection*{Errors and How to Fix Them}

As described in the \nameref{subsec:Run-Check-Edit-Repeat} section
dealing with errors described in the ERR file are part of creating
files with EnergyPlus. Resovling erros is something that both new
and very experienced users have to do. Most of the, the error message
itself, if carefully reviewed will point to the problem. Some error
messages will also reference earlier messages that should also be
checked. Careful review of the ERR file and the input file will often
reveal solutions to the most common errors. 

Also see the \nameref{subsec:Getting-Help} section

<IT WOULD BE GOOD TO ADD A LIST OF COMMON ERRORS AND SOLUTIONS HERE> 

<MAYBE ADD LINK TO OTHER RESOURCES THAT CAN BE USED TO RESOLVE ERRORS
- UNMETHOURS, MAILING LISTS>

<MAYBE SOMETHING FROM GUIDELINES FOR ENERGY SIMULATION PDF>

\subsection*{Modeling Simply}

<MAYBE SOMETHING FROM GUIDELINES FOR ENERGY SIMULATION PDF>

\subsection*{Quality Control}

<MAYBE SOMETHING FROM GUIDELINES FOR ENERGY SIMULATION PDF>

\subsection*{Choosing a Baseline}

<The importance of deciding on the baseline model that is used as
a basis of comparison for various energy efficiency measures that
are used during the design process>

\subsection*{Presenting Results to Others}

<MAYBE SOMETHING FROM PROJECT STASIO>

\section{Simulation Parameters}

EnergyPlus includes a group of input objects used to set general parameters
related to how the simulation in performed. Some of these input objects
are controlling different options that are allowed within EnergyPlus
such as the selection of algorithms to use or parameters related to
how an algorithm is used. For a new modeler, these objects should
be included with their default field values. Later when additional
control is necessary to model a specific type of measure, the field
values can be re-evaluated. These input objects should appear in your
file and appear in almost all of the example files:
\begin{itemize}
\item Timestep - the number of timesteps each hour and usually set to 6.
\item HeatBalanceAlgorithm - selects the algorithm used for simulating heat
and moisture transfer through the buildings surfaces and usually set
to ConductionTransferFunction.
\item SurfaceConvectionAlgorithm:Inside - selects the algorithm used for
inside face of the building surfaces and is usuallly set to TARP.
\item SurfaceConvectionAlgorithm:Outside - selects the algorithm use for
the outside face of the building surfaces between interior and exterior
conditions and is usually set to DOE-2.
\end{itemize}
Another input object allows you to control how you want the simulation
to be performed. The input objects should appear in your file and
appear in almost all of the example files:
\begin{itemize}
\item SimulationControl - controls if the simulation is run for the weather
file period and if sizing calculations are performed. You should become
familiar with this object since you may find it one that you frequently
change during the Run-Check-Edit cycle.
\end{itemize}
Finally two other input objects should appear in your input file and
are included in almost all example files:
\begin{itemize}
\item Version - indicates what version of EnergyPlus is being used.
\item Building - includes fields for the name of the building, and the angle
of the entire building compared to true north, as well as parameters
related to the simulation that, in general, should be allowed to default.
\end{itemize}
These objects and more are further explained in the InputOutputReference
under the heading ``Group-Simulation Parameters.''

\section{Climate and Location}

Many of the fields in the group of input objects related to location,
climate and the weather file are ones that will be set once for each
specific project.
\begin{itemize}
\item Site:Location - describes the name, latitude, longitude and other
parameters related to the location of the building. When using a weather
file, the values from the weather file will be used instead.
\item SizingPeriod:DesignDay - the high and low temperature and humidities
describing a design day that is used for sizing equipment. Two (or
more) instances of this object are frequently in a file, one for heating
and one for cooling.
\item RunPeriod - the start and stop dates of the simulation and often set
to the full year. When debugging a file, a shorter period of time
can be used to speed up the simulation portion of the Run-Check-Edit
cycle.
\item RunPeriodControl:SpecialDays - allows specfication of holidays and
a good example can be seen in 5ZoneCostEst.idf. 
\item RunPeriodControl:DaylightSavingTime - allows the specification of
the start and ending period for daylight savings time. This will impact
when schedules operate but please note that output reporting timesteps
ignore this and are always shown in standard time and a good example
can be seen in 5ZoneCostEst.idf.
\item Site:GroundTemperature:BuildingSurface - for the most common ground
temperature algorith, specifies the average temperature for each month
of the year.
\item Site:WaterMainsTemperature - input for the water temperatures supplied
to the building from underground water mains and should be specified
whenever water heaters are described. A good example can be seen in
5ZoneVAV-ChilledWaterStorage-Mixed.idf. If not specified, some default
assumptions are used for the water temperature supplied to the building.
\end{itemize}
Other input objects in this group can help performing sizing using
the weather file, override the sky temperature, impact the variation
of outdoor conditions with building height (especially important for
tall buildings), work with ground temperatures and ground heat transfer,
override the precipitation in weather files, specify the irrigatation
for a green roof, and some advanced properties related to the light
spectrum for window performance. These objects and more are further
explained in the InputOutputReference under the heading ``Group-Location
and Climate.''

\section{Schedules}

Many aspects of building operation are characterized by timing whether
it is the hours that a building is occupied or when the control systems
are in various modes. Due to this, specifying when something occurs
using the Schedule input objects becomes one of the most common things
to do. It is important to coordinate schedules properly. The operation
of office equipment in a space usually corresponds to occupancy of
that space as does the thermostat set points and fan operation. Because
schedules are such a key input for so many features of a building,
a great deal of flexibility exists in EnergyPlus to specfiy them.
\begin{itemize}
\item Schedule:Compact - The most commonly used method of specifying schedules
and uses ``Through'' and ``For'' to reduce the amount of input
required.
\item ScheduleTypeLimits - Every schedule object includes a field that helps
validate the limiting values for the schedule and this input object
describes the upper and lower limit
\item Schedule:Constant - If the value of the schedule is the same every
hour of the year, this object is the easiest way to specify that value.
\item Schedule:File - At times, data is available from a building being
monitored or for factors that change throughout the year. This input
object allows a column of data from an external file to be referenced
as the values of the schedule. A variation of this object allows input
specifically for shading.
\end{itemize}
Other input objects in this input group allow specification of schedule
values to be in different formats. These objects and more are further
explained in the InputOutputReference under the heading ``Group-Schedules.''

\section{Surface Construction Elements}

Specifying the physical properties of the building envelope is something
every building model includes. The input objects in this group allow
the specification of the different layers that make up exterior and
interior walls, roofs, floors, windows, and skylights as well as the
order of the materials in these surfaces. A large number of input
objects appear in this group since there are many special features
that need to be modeled for certain energy efficiency measures. The
following is a list only the most commonly used input objects.
\begin{itemize}
\item Material - the most common input object to describe the materials
used in opaque constructions in walls, roofs, and floors and includes
input for the thickness, conductivity, density and specific heat as
well as absorptances. See examples in ASHRAE\_2005\_HOF\_Materials.idf
located in the DataSets folder.
\item Material:NoMass - used when the material only has thermal resistance
and little thermal mass such as insulation. It should not be used
to describe materials that do have signficant thermal mass.
\item Material:AirGap - used to describe when walls or roofs have an air
gaps. Note it cannot be used for windows.
\item WindowMaterial:Glazing - describes the material used in the glass
(or other transparent material) portion of the fenestration (windows
and skylights). See WindowGlassMaterials.idf in the DataSets folder
for examples.
\item WindowMaterial:Gas - the type of gas used between layers of glass
in windows and skylights have a signficant impact on the heat transfer
performance. See WindowGasMaterials.idf in the DataSets folder for
examples.
\item Construction - a list of materials (any from the list above plus many
more) in order from the outside to the inside making up the wall,
roof, floor, window or skylight. Every input file will have several
of these input objects. Examples of constructions for wall, roofs
and floors can be found in ASHRAE\_2005\_HOF\_Materials.idf located
in the DataSets folder while examples for windows and skylights can
be found in WindowConstructs.idf in the same folder.
\item WindowMaterial:SimpleGlazingSystem - the best way to describe a window
is with a construction that references WindowMaterial:Glazing and
WindowMaterial:Gas input objects but if all you have is the U-Factor
and Solar Heat Gain Coefficient they can be specified in this input
object.
\end{itemize}
A large variety of input objects in this group are not as commonly
used but are key to modeling specific types of walls and windows so
if what you are trying to model does not fall into the neat categories
for the objects described so far, there is still a good chance that
EnergyPlus has an input object that will work. These other input objects
include ones for wals and roof that can be used when modeling combined
heat and moisture transfer, modeling material which undergo a phase
change to store heat in the wall or when the material properties change
with temperature, when the material allows infrared radiation to flow
through it, when modeling green (vegetated) roofs, for simplified
C or F-factor modeling, or when the wall includes resistance or hydronic
tubing to provide heat. The other input object to describe windows
and skylights include input objects that can be used to describe thermochromic
and electochomic glazing, mixtures of gases between layers of glass,
vacuum glazing, movable portions of the window assembly such as shades
and blinds and screens, alternative ways of specifying fenestration
such as equivalent layers or refraction extinction method or ASHWAT
model or from a WINDOW program export/data file or specifying wavelength-by-wavelength
properties. 

The input objects described in this section are further explained
in the InputOutputReference under the heading ``Group-Surface Construction
Elements.''

\section{Thermal Zones and Surfaces}

Central
\begin{itemize}
\item GlobalGeometryRules
\item Zone
\item BuildingSurface:Detailed
\item FenestrationSurface:Detailed
\end{itemize}
Shading related
\begin{itemize}
\item Shading:Site:Detailed
\item Shading:Building:Detailed
\item Shading:Zone:Detailed
\end{itemize}
And some varient
\begin{itemize}
\item Wall:Detailed
\item RoofCeiling:Detailed
\item Floor:Detailed
\item Wall:Exterior
\item Wall:Adiabatic
\item Wall:Underground
\item Wall:Interzone
\item Roof
\item Ceiling:Adiabatic
\item Ceiling:Interzone
\item Floor:GroundContact
\item Floor:Adiabatic
\item Floor:Interzone
\item Window
\item Door
\item GlazedDoor
\item Window:Interzone
\item Door:Interzone
\item GlazedDoor:Interzone
\item Shading:Site
\item Shading:Building
\item Shading:Overhang
\item Shading:Overhang:Projection
\item Shading:Fin
\item Shading:Fin:Projection
\item ShadingProperty:Reflectance
\end{itemize}
Other object
\begin{itemize}
\item WindowShadingControl
\item WindowProperty:FrameAndDivider (why here?)
\item WindowProperty:AirflowControl (why here?)
\item WindowProperty:StormWindow (why here?)
\item InternalMass
\end{itemize}
Rarely used:
\begin{itemize}
\item ZoneList
\item ZoneGroup
\item GeometryTransform
\end{itemize}

\section{Internal Gains }

Most things inside a building produce heat and that includes people,
appliances, office equipment, and lighting. The combination of all
these items that produce heat within a building are called internal
gains and represent a significant contribution, sometimes the largest
contribution, to the cooling requirements for a building. In addition,
they offset the amount of heat from the HVAC system that is needed
at a given time. Typically, the peak value is entered in these input
objects for this group such as the maximum number of people, the total
power of equipment, or the total lighting power and then a schedule
is used to modify that value each hour of the year. It is just as
critical that the schedule values are realistic for your building
as is the peak value. For almost all buildings, it is rare that the
peak occupancy occurs for more than a few hours per year if at all
and this is especially the case for retail stores, theaters, and sports
complexes. Even office buildings when counting vacations and people
out of the building for meetings will rarely have peak occupancy. 

The most common internal gain input objects are: 
\begin{itemize}
\item People -- specifies not only the sensible, latent and radiant heat
from people but also includes ways of reporting the comfort of occupants
using a variety of thermal comfort models 
\item Lights -- describes the heat related to lighting systems 
\item ElectricEquipment -- describes the heat related to electrical appliances,
office equipment, and other heat sources that are powered by electricity 
\end{itemize}
Less common internal gains input objects include: 
\begin{itemize}
\item GasEquipment 
\item HotWaterEquipment 
\item SteamEquipment 
\item OtherEquipment 
\item ElectricEquipment:ITE:AirCooled 
\item SwimmingPool:Indoor 
\end{itemize}
Other 
\begin{itemize}
\item ComfortViewFactorAngles -- allows the specification of how different
surfaces impact the thermal comfort calculations for the occupants 
\item ZoneBaseboard:OutdoorTemperatureControlled -- specifies how the outside
air temperature impacts the control of zone baseboard systems. 
\end{itemize}
The Internal Gains group also contains input objects related to zone
contaminant sources and sinks. The input objects include modeling
components that impact contaminant concentrations which are scheduled,
pressure driven, use a cut off model, assume a decaying source, surface
diffusion, or using a deposition velocity model.

The input objects described in this section are further explained
in the InputOutputReference under the heading ``Group-Internal Gains.''

\section{Daylighting }

Reducing the amount of powered lighting that is used when sufficient
natural daylight illuminates the interior building through windows
and skylights is called daylighting. It is a very common energy efficiency
measure in buildings and is often required for new building designs
depending on the energy code that applies to the building location.
The most common input objects related to daylighting are: 
\begin{itemize}
\item Daylighting:Controls -- specifies the algorithm used for daylighting,
the dimming of lights is continuous or stepped, and how glare calculations
are performed. 
\item Daylighting:ReferencePoint -- specifies the location of the sensors
for the daylighting control system 
\end{itemize}
The input output reference includes not only a description of these
objects but also extra guidance on how they should be applied. 

Three different devices can be used with day 
\begin{itemize}
\item DaylightingDevice:Tubular 
\item DaylightingDevice:Shelf 
\item DaylightingDevice:LightWell 
\end{itemize}
An object called Daylighting:DELight:ComplexFenestration is used with
one of the two control methods specified in the Daylighting:Controls
input object when used in conjunction with complex fenestration systems
such as prismatic and holographic glass. 

Some flexibility is given to provide extra output related to daylighting
and includes 
\begin{itemize}
\item Output:DaylightFactors -- creates a special report on the 
\item Output:IlluminanceMap -- allows the generation of maps of illuminance
values within each interior zone that uses daylighting controls. The
exact file format can be set using the OutputControl:IlluminanceMap:Style
input object 
\end{itemize}
More details of these output options can be found in the OutputDetailsAndExamples
document. The input objects described in this section are further
explained in the InputOutputReference under the heading ``Group-Daylighting.''

\section{<DEPRECATED>Envelope and Geometry }

One input object that must appear in every file is the GlobalGeometryRules
input object

\subsection*{Coordinate System}

<maybe take this from existing documentation>

An input called Building North Axis in the Building input object,
simplifies building geometry specification by designating how the
building is oriented relative to true North. Buildings frequently
do not line up with true north. For convenience, one may enter surfaces
in a \textquotedblleft regular\textquotedblright{} coordinate system
and then shift them via the use of the North Axis. The value is specified
in degrees from \textquotedblleft true north\textquotedblright{} (clockwise
is positive). 

\subsection*{Surfaces and Zones}

A building \textquotedblleft surface\textquotedblright{} is the fundamental
element in the building model. In the general sense, there are two
types of \textquotedblleft surfaces\textquotedblright{} in EnergyPlus.
These are: 
\begin{itemize}
\item heat transfer surfaces
\item heat storage surfaces 
\end{itemize}
A good rule to follow for EnergyPlus modeling of surfaces is: \textquotedblleft Always
define a surface as a heat storage surface unless it must be defined
as a heat transfer surface.\textquotedblright{} Any surface, which
is expected to separate spaces of significantly different temperatures,
must be defined as a heat transfer surface. Thus, exterior surfaces,
such as outside walls, roofs and floors, are heat transfer surfaces.
Interior surfaces (partitions) are heat storage surfaces if they separate
spaces maintained at the same temperature and heat transfer surfaces
if they separate spaces maintained at different temperatures. A discussion
of how to define heat transfer and heat storage surfaces will occur
in later steps. In order to correctly \textquotedblleft zone\textquotedblright{}
the building it is necessary only to distinguish between the two. 

\subsection*{Zoning}

A \textquotedblleft zone\textquotedblright{} is a thermal, not a geometric,
concept. A \textquotedblleft zone\textquotedblright{} is an air volume
at a uniform temperature plus all the heat transfer and heat storage
surfaces bounding or inside of that air volume. EnergyPlus calculates
the energy required to maintain each zone at a specified temperature
for each hour of the day. Since EnergyPlus performs a zone heat balance,
the first step in preparing a building description is to break the
building into zones. The objective of this exercise is to define as
few zones as possible without significantly compromising the integrity
of the simulation. 

Although defining building zones is somewhat of an art, a few general
rules will keep the new simulation user out of trouble. <\textcompwordmark <Consider
the following figure, which shows the floor plan of an Adult Education
Center. >\textcompwordmark >

The question is, \textquotedblleft How many thermal zones should be
used to model this building?\textquotedblright{} The inexperienced
building modeler may be tempted to define each room in the building
as a zone, but the thermal zone is defined as a volume of air at a
uniform temperature. The general rule then is to use the number of
fan systems (and radiant systems) not the number of rooms to determine
the number of zones in the building. The minimum number of zones in
a general simulation model will usually be equal to the number of
systems serving the building. The collection of heat transfer and
heat storage surfaces defined within each zone will include all surfaces
bounding or inside of the space conditioned by the system. 

\subsection*{Materials and Constructions}

When you don't know the specific materials and constructions for the
builidng being modeled, you may want to review the ASHRAE\_2005\_HOF\_Materials.idf
file contained in the DataSets folder. The file contains typical materials
and constructionsand is based on the 2005 version of the ASHRAE Handbook
- Fundamentals.

When the building was zoned, our objective was to define as few zones
as possible. Now we would like to extend this objective to include
defining as few surfaces as possible without significantly compromising
the integrity of the simulation. We reduce the number and complexity
of surfaces in our input file by defining equivalent surfaces. 

Before dealing with equivalent surfaces, it is appropriate to take
the concept of a thermal zone one step further. EnergyPlus performs
heat balances on individual zone surfaces and on the zone air. For
purposes of the heat transfer calculations, a geometrically correct
rendering of the zone surfaces is not required. The surfaces do not
even have to be connected. As long as the program knows to which thermal
zone (mass of air) each surface transfers heat, it will calculate
all heat balances correctly. For example, all heat storage surfaces
of the same construction within a zone may be defined as a single
rectangular surface. The size of this equivalent surface will equal
the sum of all the areas of all the heat storage surfaces in the zone.
A few simple rules will further explain what we mean by equivalent
surfaces and how these surfaces may be used. Remember that these are
guidelines for optional simplification of input. Each simplification
must be evaluated to determine if it would significantly impact certain
shading, interior solar gains, or daylighting features. The goal is
to seek an adequate level of detail to capture the key features of
the building envelope without spending excess time describing and
computing results for details that are insignificant. 
\begin{enumerate}
\item Define all roofs and floors as rectangles regardless of the shape
of the zone. Each zone may have one rectangular roof and one rectangular
floor of a given construction. 
\item Define all heat storage surfaces of the same construction within a
zone as a single surface. The size of the single surface is obtained
by summing the individual surface areas exposed to the zone. Thus,
if a partition is completely within a zone (both sides of the partition
are exposed to the zone), the area of each side must be added to the
area of the equivalent surface. On the other hand, if the partition
separates two zones, the area of only one side should be added to
the equivalent surface. 
\item Combine all windows on a given exterior surface into a single window.
Usually each exterior surface should have only one window of each
type. Overhangs or other shading devices may require that more windows
be specified or combined together. By using the WindowMaterial:Glazing
con- struction for your glass door, they will be correctly modeled
in EnergyPlus with sunlight transferring into the zone. 
\end{enumerate}

\subsection*{Heat Transfer to the Ground}

\section{HVAC }

\subsection*{Ideal Loads}

\subsection*{HVAC Templates}

HVAC Template input objects are available. These are intended to allow
for several \textquotedblleft usual\textquotedblright{} HVAC types
to be expanded into EnergyPlus HVAC inputs with minimal user entries.
These are described in the \textquotedblleft Input/Output Reference\textquotedblright{}
document under the Group \textquotedblleft HVACTemplates\textquotedblright{}
and the expansion process is described in the Auxiliary Programs document
under \textquotedblleft ExpandObjects\textquotedblright .

\subsection*{Zone Equipment}

\subsection*{Air Loops}

\subsection*{Coils}

\subsection*{Plant Loops}

\subsection*{Plant Equipment}

\subsection*{Controls}

\subsection*{EMS}

\section{Economics and Emissions}

\subsection*{Tariffs}

\subsection*{Life Cycle Costs}

\subsection*{Emissions and TDV}

\section{Specific Tips }

\subsection*{Data Sets}

EnergyPlus uses snippets of IDF files to create the library of data
that may be useful for you. Two folders are created upon installation:
DataSets -- which contains IDF snippets and MacroDataSets -- which
also contain IDF snippets but are in a form such that they can be
easily used with the EPMacro program. 

\subsection*{Other Useful Utility Programs}

The EnergyPlus install includes a variety of tools to help with various
aspectis of converting data or displaying information to help with
using EnergyPlus
\begin{itemize}
\item Coefficient Curve Generation - The CoeffConv and CoeffConv utility
programs can be used to convert DOE-2 temperature depen- dent curves
(Fahrenheit) to EnergyPlus temperature curves (Centigrade/Celsius).
These programs are described in the Auxiliary Programs document.
\item HVAC-Diagram - Another post processing program is the HVAC-Diagram
application. It reads one of the EnergyPlus output files (eplusout.bnd)
and produces a Scalable Vector Graphics (SVG) file. More information
on the HVAC Diagram program is found in the Auxiliary Programs document.
\item convertESOMTR - This simple post processing program can be used seamlessly
with EP-Launch to provide IP (inch- pound) unit output files rather
than SI units. This program is described more fully in the Auxiliary
Programs document.
\end{itemize}

\end{document}
