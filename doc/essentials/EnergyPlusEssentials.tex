%% LyX 2.3.1-1 created this file.  For more info, see http://www.lyx.org/.
%% Do not edit unless you really know what you are doing.
\documentclass[english]{article}
\usepackage[T1]{fontenc}
\usepackage{babel}
\usepackage{textcomp}
\usepackage[unicode=true,pdfusetitle,
 bookmarks=true,bookmarksnumbered=false,bookmarksopen=false,
 breaklinks=false,pdfborder={0 0 1},backref=false,colorlinks=false]
 {hyperref}
\begin{document}
\title{EnergyPlus Essentials}
\date{December 17, 2018}
\maketitle

\section{Introduction}

\section*{What is BEM?}

According to the \href{https://www.bemlibrary.com/index.php/owners-managers/introduction/what-bem/}{BEM Library}:
\begin{quotation}
``Building Energy Modeling (BEM) is the practice of using computer-based
simulation software to perform a detailed analysis of a building\textquoteright s
energy use and energy-using systems. The simulation software works
by enacting a mathematical model that provides an approximate representation
of the building. 

``BEM includes whole-building simulation as well as detailed component
analysis utilizing specialized software tools that address specific
concerns, such as moisture transfer through building materials, daylighting,
indoor air quality, natural ventilation, and occupant comfort.

``BEM offers an alternative approach that encourages customized,
integrated design solutions, which offer deeper savings. Using BEM
to compare energy-efficiency options directs design decisions prior
to construction. It also guides existing building projects to optimize
operation or explore retrofit opportunities.''
\end{quotation}
Some other terms that are used to describe the same topic are:
\begin{itemize}
\item Building Simulation
\item Building Performance Simulation
\item Building Energy Simulation
\item Building Performance Modeling
\end{itemize}
Somewhat confusing, the term ``model'' can refer to both the specific
building being described and analyzed as well as referencing the software
for implementing a specific component of the building such as a wall
or chiller.

To learn more about building energy modeling consider reviewing the
following resources:
\begin{itemize}
\item \href{https://www.ashrae.org/technical-resources/ashrae-handbook/description-2017-ashrae-handbook-fundamentals}{2017 ASHRAE Handbook - Fundamentals }Chapter
19 Energy Estimating and Modeling Methods
\item \href{https://www.bemlibrary.com/}{BEM Libary}
\item \href{http://www.ibpsa.org/?page_id=695}{IBPSA}, \href{https://www.ibpsa.us/videos/all}{IBPSA-USA},
and \href{https://www.youtube.com/results?search_query=building+energy+modeling}{YouTube}
videos
\item And numerous \href{https://www.amazon.com/s/ref=nb_sb_noss_2?url=search-alias\%3Daps&field-keywords=building+energy+modeling}{books}
\end{itemize}

\section*{Questions that BEM can answer}

The most common questions that BEM can answer are:
\begin{itemize}
\item If my building was made or operated differently, how would the energy
consumption change?
\item Does my building comply with a building energy code or standard?
\item What kind of rating or how many points can I get in a environmental
certification program?
\item Is my building operating as it was designed to?
\item What is a good target for energy consumption for my building?
\end{itemize}
These questions can take place at different times during the life-cycle
of a building from before schematic design all through the rest of
the design process, and into the operation of the building

\subsection*{The Design Process and BEM }

Building energy modeling can be used throughout the design process
for a new building or when considering updates to existing buildings.
It can also be used later in the life-cycle of a building related
to comparing actual operation of the building with predicted operation.
ASHRAE \href{https://www.techstreet.com/ashrae/standards/ashrae-209-2018?gateway_code=ashrae&product_id=2010483}{Standard 209}
titled ``Energy Simulation Aided Design for Buildings Except Low-Rise
Residential Buildings'' is a critical document to in how to apply
building energy modeling. It describes a methodology to apply building
energy modeling to the design and was created to define reliable and
consistent procedures that advance the use of timely energy modeling
to quanifty the impact of design decision s at the point in time which
they are made. The standard includes different \textquotedblleft Modeling
Cycles\textquotedblright{} for different stages of using building
energy modeling during the life-cycle of the building:
\begin{itemize}
\item Modeling Cycle \#1 -- Simple Box Modeling
\item Modeling Cycle \#2 -- Conceptual Design Modeling
\item Modeling Cycle \#3 -- Load Reduction Modeling
\item Modeling Cycle \#4 -- HVAC System Selection Modeling
\item Modeling Cycle \#5 -- Design Refinement
\item Modeling Cycle \#6 -- Design Integration and Optimization
\item Modeling Cycle \#7 -- Energy Simulated-Aided Value Engineering
\item Modeling Cycle \#8 -- As-Designed Energy Performance
\item Modeling Cycle \#9 -- Change Orders
\item Modeling Cycle \#10 -- As-Built Energy Performance
\item Modeling Cycle \#11 -- Postoccupancy Energy Comparison
\end{itemize}
In addition it has information about how to integrate climate and
site analysis, benchmarking, energy charrettes, the energy performance
goals of the owners project requirements in to the design process
when using building energy modeling.

\subsection*{EnergyPlus Capabilities}

According to the \href{https://energyplus.net/}{energyplus.net web site}
(as of December 2018):
\begin{quotation}
``EnergyPlus\texttrademark{} is a whole building energy simulation
program that engineers, architects, and researchers use to model both
energy consumption---for heating, cooling, ventilation, lighting
and plug and process loads---and water use in buildings. Some of
the notable features and capabilities of EnergyPlus include:
\end{quotation}
\begin{itemize}
\item Integrated, simultaneous solution of thermal zone conditions and HVAC
system response that does not assume that the HVAC system can meet
zone loads and can simulate un-conditioned and under-conditioned spaces.
\item Heat balance-based solution of radiant and convective effects that
produce surface temperatures thermal comfort and condensation calculations.
\item Sub-hourly, user-definable time steps for interaction between thermal
zones and the environment; with automatically varied time steps for
interactions between thermal zones and HVAC systems. These allow EnergyPlus
to model systems with fast dynamics while also trading off simulation
speed for precision.
\item Combined heat and mass transfer model that accounts for air movement
between zones.
\item Advanced fenestration models including controllable window blinds,
electrochromic glazings, and layer-by-layer heat balances that calculate
solar energy absorbed by window panes.
\item Illuminance and glare calculations for reporting visual comfort and
driving lighting controls.
\item Component-based HVAC that supports both standard and novel system
configurations.
\item A large number of built-in HVAC and lighting control strategies and
an extensible runtime scripting system for user-defined control.
\item Functional Mockup Interface import and export for co-simulation with
other engines.
\item Standard summary and detailed output reports as well as user definable
reports with selectable time-resolution from annual to sub-hourly,
all with energy source multipliers.''
\end{itemize}
EnergyPlus runs on Windows, MacOS, and Linux computers. 

<need to balance overview with enough details to help people, I think
a list of what can be modeled might be more helpful>

\subsection*{Open Source}

\href{https://energyplus.net/}{EnergyPlus}is an \href{https://opensource.org/}{Open Source}
program so all the \href{https://github.com/NREL/EnergyPlus}{source code}
is available to inspect and modify. If you are interested in how calculations
are performed and the \href{https://energyplus.net/documentation}{Engineering Reference}
does not provide enough details to you, you can review the source
code itself. \href{https://github.com/NREL/EnergyPlus/wiki/BuildingEnergyPlus}{Instructions to build the code}
(compile the source code into an executable application) are available
in source code repository wiki if you see something that needs to
be enhanced or fixed. 

\subsection*{Brief History}

EnergyPlus has been under development since 1997 and was first released
in 2001. EnergyPlus has its roots in both the BLAST and DOE--2 programs.
BLAST (Building Loads Analysis and System Thermodynamics) and DOE--2
were both developed and released in the late 1970s and early 1980s
as energy and load simulation tools. BLAST was developed by \href{https://www.erdc.usace.army.mil/Locations/CERL/}{Construction Engineering Researrch Laboratory (CERL)}
and \href{https://illinois.edu/}{University of Illinois} which DOE-2
was developed by \href{https://www.lbl.gov/}{Berkeley Lab} and many
others. Their intended audience is a design engineer or architect
that wishes to size appropriate HVAC equipment, develop retrofit studies
for life cycling cost analyses, optimize energy performance, etc.
Born out of concerns driven by the energy crisis of the early 1970s
and recognition that building energy consumption is a major component
of the American energy usage statistics, the two programs attempted
to solve the same problem from two slightly different perspectives.
Both programs had their merits and shortcomings, their supporters
and detractors, and solid user bases both nationally and internationally.
Originally written in Fortran, in 2014 it was converted to C++. It
was developed as a simulation engine and many graphical user interfaces
utilize it.

\subsection*{Documentation and Other Resources}

The EnergyPlus documentation is currently included the installation
in the ``Documenation'' folder as PDFs. It is also available as
\href{https://energyplus.net/documentation}{PDFs online} and as \href{https://bigladdersoftware.com/epx/docs/}{browsable HTML}.
The documentation includes:
\begin{itemize}
\item Input and Output Reference: Contains a thorough description of the
various input and output files related to EnergyPlus, the format of
these files, and how the files interact and interrelate.
\item Output Details, Examples and Data Sets: Contains details on output
from EnergyPlus and specific examples. It also addresses the reference
data sets that are included.
\item Auxiliary Programs:~Contains information for the auxiliary programs
that are part of the EnergyPlus package. For example, this document
contains the user manual for the Weather Converter program, descriptions
on using Ground Heat Transfer auxiliary programs with EnergyPlus,
Compact HVAC descriptions, the Transition program/package and other
assorted documents.
\item The Engineering Reference: Provides more in-depth knowledge into the
theoretical basis behind the various calculations contained in the
program including algorithm descriptions.
\item EMS Application Guide: Contains information useful to use the advanced
feature of EnergyPlus: Energy Management System tweaks. The Erl language
is described and examples for use are given.
\item External Interface Application Guide: Contains information specific
to using the external interface feature of EnergyPlus to connect other
simulation systems.
\item Plant Application Guide: Details the methods for simulating real chilled
and hot water plant systems within EnergyPlus.
\item Using EnergyPlus for Compliance Guide: Contains information specific
to using EnergyPlus in Compliance and Standard Rating systems.
\item External Interface(s) Application Guide: Contains information about
external interfaces (through the Building Controls Virtual Test Bed
link) to EnergyPlus.
\item Tips \& Tricks for Using EnergyPlus: Contains short tips and tricks
for using various parts of EnergyPlus. <MAYBE THIS SHOULD BE ELIMINATED
AND PORTIONS INCLUDED IN THIS DOCUMENT>
\end{itemize}
To learn more about how to use EnergyPlus, the example files are a
great resource. The example files are in the directory ExampleFiles
and 

\section{The EnergyPlus Ecosystem }

\subsection*{Current Interfaces }

EnergyPlus is often used directly using the text file input (IDF or
epJSON) and various output file formats along with the utilities that
come with the installation package. More information on that can be
found in the Using EnergyPlus section. In addition, EnergyPlus is
often the simulation engine for graphical user interfaces. To see
comprehensive list, see the \href{https://www.buildingenergysoftwaretools.com/}{BEST (Building Energy Software Tools) Directory}
that is operated by \href{https://www.ibpsa.us/}{IBPSA-USA}. At the
time of writing, many people use EnergyPlus inside an existing graphical
user interface such as \href{https://www.openstudio.net/}{OpenStudio},
\href{https://www.designbuilder.com/}{DesignBuilder}, \href{https://d-alchemy.com/products/simergy}{Simergy},
\href{https://bigladdersoftware.com/projects/euclid/}{Euclid}, \href{https://www.ladybug.tools/}{Ladybug},
\href{https://www.autodesk.com/products/insight/overview}{Insight},
and \href{https://www.bentley.com/en/products/brands/aecosim}{AECOsim}.
Other interfaces are also available. 

\subsection*{Approaches to Implement Measures }

In the terminology used within the building energy modeling industry
``measures,'' sometimes called energy conservation measures (ECMs)
or energy efficiency measures (EEMs), are when alternatives configurations
of a building are considered and simulated and compared with the original
building model. Measures include added wall or roof insulation, lower
internal lighting power, higher efficiency heating and cooling equipment,
and many others. When using EnergyPlus within a graphical user interface,
that interface often provides a way to implement various measures.
When not working within a graphically user interface, users have several
options for implementing measures:

For a specific building input file, a copy of the input file can be
made and then modified to reflect the measure. This is an easy approach
initially but since the original building model might change, it means
making the change in the file that reflects the measure. When many
measures are being considered, this duplicative editing can be inefficient
and prone to errors. In addition, the measure cannot be easily applied
to a different building energy model without again editing another
set of files duplicating the original effort. Due to this, it is very
common for some type of scripting to be used so that measures can
be applied to the original building model and other models of other
buildings reliably and quickly. Scripting in this approach means to
apply some type of programming language to the task of modifying files.

EnergyPlus includes, with the installation package, two methods of
implementing measures EP-Macro and the ParametricPreprocessor. EP-Macro
is similar to the C langauge pre-processor and allows for portions
of the file to be included or excluded, for portions of other files
to be included when desired, and for specific entries that are normally
fixed values to change programmatically. EP-Macro is documented int
he AuxiliaryPrograms documentation. A file with lines starting ``\#\#''
or with the ``imf'' file extension are likely to be a file using
EP-Macro files. 

The EnergyPlus ParametricPreprocessor uses special objects in EnergyPlus
to set values for any field in any other object for a series of simple
options <OBJECTS HAVE NOT BEEN DEFINED YET>. This is briefly described
in the AuxiliaryPrograms documentation and more details are present
in the InputOutputReference under the section on ``Parametric Objects.''
The ParametricPreprocessor approach is very simple but limited in
the flexibility that it affords. It is suitable for implementing measures
related to internal loads, constructions, and simple efficiency changes
but is probably not the appropriate tool for more complicated measures.

Another approach to writing measures and using measures written by
others is to use \href{https://www.openstudio.net/}{OpenStudio} and
its \href{http://nrel.github.io/OpenStudio-user-documentation/}{documentation}.
OpenStudio is closely aligned with EnergyPlus but has a different
data model and file format. It is an application programming interface
for EnergyPlus and some graphical user interfaces are using it to
work with EnergyPlus. It has two categories of measures: native OpenStudio
measures and EnergyPlus measures. Both types of measures are usually
written in the Ruby programming language. Many measures are already
available for OpenStudio and can be found in the \href{https://bcl.nrel.gov/}{Building Component Library.}It
is an open source project. 

Using the Python scripting language, \href{https://github.com/santoshphilip/eppy}{Eppy}
allows access to any of the input objects in EnergyPlus as if they
were native Python objects. It is very flexible and an especially
good choice if you are already familiar with Python. It is also an
open source project.

Another option based on the Ruby programming language is \href{https://bigladdersoftware.com/projects/modelkit/}{Modelkit}
which similar to EP-Macro consists of embedding scripting within normal
EnergyPlus input files but since it is using Ruby has a great deal
more flexility.

A \href{https://www.ashrae.org/File\%20Library/Conferences/Specialty\%20Conferences/2018\%20Building\%20Performance\%20Analysis\%20Conference\%20and\%20SimBuild/Papers/C043.pdf}{paper}
that describes these approaches to scripting and more was made during
the 2018 ASHRAE/IBPSA-USA Building Performance Conference and SimBuild.

\subsection*{Parametric Analysis Tools }

\subsection*{Weather Files }

\subsection*{Co-simulation and Linked Software }

\subsection*{Interoperability }

\subsection*{Example Files }

\subsection*{Getting Help}

\section{Using EnergyPlus }

\subsection*{Running EnergyPlus}

<quickstart guide>

\subsection*{Run-Check-Edit Repeat}

\subsection*{IDF and JSON syntax}

\subsection*{Common Objects}

\subsection*{Key Concepts}

\subsection*{Versions and Updating}

\subsection*{Errors and How to Fix Them}

\subsection*{Modeling Simply}

\subsection*{Quality Control}

\subsection*{Choosing a Baseline}

<The importance of deciding on the baseline model that is used as
a basis of comparison for various energy efficiency measures that
are used during the design process>

\subsection*{Presenting Results to Others}

\section{Introduction to Example Building Models}

\section{Envelope and Geometry }

\subsection*{Coordinate System}

<maybe take this from existing documentation>

\subsection*{Surfaces and Zones}

\subsection*{Zoning}

\subsection*{Materials and Constructions}

\subsection*{Heat Transfer to the Ground}

\section{Internal Loads }

\subsection*{Lights and Equipment}

\subsection*{Daylighting}

\subsection*{People and Comfort}

\subsection*{Infiltration and Airflow}

\section{HVAC }

\subsection*{Ideal Loads}

\subsection*{HVAC Templates}

\subsection*{Zone Equipment}

\subsection*{Air Loops}

\subsection*{Coils}

\subsection*{Plant Loops}

\subsection*{Plant Equipment}

\subsection*{Controls}

\subsection*{EMS}

\section{Economics and Emissions}

\subsection*{Tariffs}

\subsection*{Life Cycle Costs}

\subsection*{Emissions and TDV}

\section{Specific Tips }
\end{document}
