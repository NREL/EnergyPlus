%% LyX 2.3.1-1 created this file.  For more info, see http://www.lyx.org/.
%% Do not edit unless you really know what you are doing.
\documentclass[english]{article}
\usepackage[T1]{fontenc}
\usepackage{babel}
\usepackage[unicode=true,pdfusetitle,
 bookmarks=true,bookmarksnumbered=false,bookmarksopen=false,
 breaklinks=false,pdfborder={0 0 1},backref=false,colorlinks=false]
 {hyperref}
\begin{document}
\title{EnergyPlus Essentials}
\date{December 10, 2018}
\maketitle

\section{Introduction}

\section*{What is BEM?}

According to the \href{https://www.bemlibrary.com/index.php/owners-managers/introduction/what-bem/}{BEM Library}:
\begin{quotation}
Building Energy Modeling (BEM) is the practice of using computer-based
simulation software to perform a detailed analysis of a building\textquoteright s
energy use and energy-using systems. The simulation software works
by enacting a mathematical model that provides an approximate representation
of the building. 

BEM includes whole-building simulation as well as detailed component
analysis utilizing specialized software tools that address specific
concerns, such as moisture transfer through building materials, daylighting,
indoor air quality, natural ventilation, and occupant comfort.

BEM offers an alternative approach that encourages customized, integrated
design solutions, which offer deeper savings. Using BEM to compare
energy-efficiency options directs design decisions prior to construction.
It also guides existing building projects to optimize operation or
explore retrofit opportunities.
\end{quotation}
Some other terms that are used to describe the same topic are:
\begin{itemize}
\item Building Simulation
\item Building Performance Simulation
\item Building Energy Simulation
\item Building Performance Modeling
\end{itemize}
Somewhat confusing, the term ``model'' can refer to both the specific
building being described and analyzed as well as referencing the software
for implementing a specific component of the building such as a wall
or chiller.

<add link to ashrae handbook?>

\section*{Questions that BEM can answer}

The most common questions that BEM can answer are:
\begin{itemize}
\item If my building was made or operated differently, how would the energy
consumption change?
\item Does my building comply with a building energy code or standard?
\item What kind of rating or how many points can I get in a environmental
certification program?
\item Is my building operating as it was designed to?
\item What is a good target for energy consumption for my building?
\end{itemize}
These questions can take place at different times during the life-cycle
of a building from before schematic design all through the rest of
the design process, and into the operation of the building

\subsection*{The Design Process and BEM }

\subsection*{EnergyPlus Capabilities}

\subsection*{Open Source}

\href{https://energyplus.net/}{EnergyPlus}is an \href{https://opensource.org/}{Open Source}
program so all the \href{https://github.com/NREL/EnergyPlus}{source code}
is available to inspect and modify. If you are interested in how calculations
are performed and the \href{https://energyplus.net/documentation}{Engineering Reference}
does not provide enough details to you, you can review the source
code itself. \href{https://github.com/NREL/EnergyPlus/wiki/BuildingEnergyPlus}{Instructions to build the code}
(compile the source code into an executable application) are available
in source code repository wiki if you see something that needs to
be enhanced or fixed. 

\subsection*{Brief History}

\subsection*{Documentation and Other Resources}

\section{The EnergyPlus Ecosystem }

\subsection*{Current GUIs }

<Make the list for major GUIs and tools and than reference IPBSA.
Say \textquotedbl at time of writing..\textquotedbl . Say \textquotedbl many
people use E+ inside an existing GUI\textquotedbl . Minimize confusion
and upkeep.> 

\subsection*{Approaches to Implement Measures }

\subsection*{Parametric Analysis Tools }

\subsection*{Weather Files }

\subsection*{Co-simulation and Linked Software }

\subsection*{Interoperability }

\subsection*{Example Files }

\section{Using EnergyPlus }

\subsection*{Running EnergyPlus}

<quickstart guide>

\subsection*{Run-Check-Edit Repeat}

\subsection*{IDF and JSON syntax}

\subsection*{Common Objects}

\subsection*{Key Concepts}

\subsection*{Versions and Updating}

\subsection*{Errors and How to Fix Them}

\subsection*{Modeling Simply}

\subsection*{Quality Control}

\subsection*{Choosing a Baseline}

<The importance of deciding on the baseline model that is used as
a basis of comparison for various energy efficiency measures that
are used during the design process>

\subsection*{Presenting Results to Others}

\section{Introduction to Example Building Models}

\section{Envelope and Geometry }

\subsection*{Coordinate System}

\subsection*{Surfaces and Zones}

\subsection*{Zoning}

\subsection*{Materials and Constructions}

\subsection*{Heat Transfer to the Ground}

\section{Internal Loads }

\subsection*{Lights and Equipment}

\subsection*{Daylighting}

\subsection*{People and Comfort}

\subsection*{Infiltration and Airflow}

\section{HVAC }

\subsection*{Ideal Loads}

\subsection*{HVAC Templates}

\subsection*{Zone Equipment}

\subsection*{Air Loops}

\subsection*{Coils}

\subsection*{Plant Loops}

\subsection*{Plant Equipment}

\subsection*{Controls}

\subsection*{EMS}

\section{Economics and Emissions}

\subsection*{Tariffs}

\subsection*{Life Cycle Costs}

\subsection*{Emissions and TDV}

\section{Specific Tips }
\end{document}
