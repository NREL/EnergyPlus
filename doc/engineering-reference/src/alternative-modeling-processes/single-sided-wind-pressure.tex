\subsection{Single Sided Wind Pressure Coefficient Algorithm}

The approach in the single sided (SS) model is based on correlating the flow rate into
the room through one or more openings with local flow conditions at the openings. These
are in one fa\c{c}ade of a rectangular building of width WB, depth DB and height HB.
There is a vertical wind profile $U(z)$ approaching the building, which generates a
local flow $U_L$ parallel to the fa\c{c}ade (whose value depends on location, building
geometry and wind direction).

Consider first the 2-aperture case, where air flows in through one opening of area
$A_\textup{in}$ and exits through another of equal area. Let us denote the flow rate
of ambient air into the room by $Q$, and define a dimensionless flow rate $Q'$ by
\begin{equation}
Q'=Q/(A_\textup{in} U_\textup{ref}),
\end{equation}
where $U_\textup{ref}$ is a reference velocity taken as the undisturbed wind speed
$U(z_\textup{ref})$ at a reference height $z_\textup{ref} = 10 \textup{m}$. There is a
pressure difference $\Delta p(t)$ between the two opening locations, with time-averaged
mean value $\overline{\Delta p}$ and standard deviation $\sigma_{\Delta p}$.

There are three principal mechanisms determining the flow rate:

\begin{enumerate}[label=(\alph*)]
\item Mean pressure difference:  a non-zero mean value for the difference in pressure
at the two opening locations, $\overline{\Delta p}$, drives a flow that scales as
$\overline{\Delta p}^{1/2}$.\label{ss-mean-mech}
\item Unsteady pressure difference: turbulence drives a fluctuating pressure difference
between the two openings, resulting in a net addition of air from the outside. This is
characterized by the standard deviation of the pressure difference signal,
$\sigma_{\Delta p}$ and the associated flow rate adds to that arising from
$\overline{\Delta p}$. It includes periodic Strouhal forcing (`pumping') at relevant
wind angles. Note that this contribution is non-zero even when the mean pressure
difference is zero.\label{ss-unsteady-mech}
\item Shear layer mechanism: when there is small or zero mean pressure difference
between the two opening locations, the local flow parallel to the fa\c{c}ade causes a
mixing layer to grow over the length of the opening resulting in the incorporation of
ambient air. This scales with $U_L$.\label{ss-shear-mech}
\end{enumerate}

Our model therefore views the flow rate as comprising contributions from each of these
as appropriate, i.e. 
\begin{equation}
Q'=\left[a_p |\Delta c_p |+a_\sigma \sigma_{\Delta c_p} \right]^{1/2}+a_s U_L'
\end{equation}
where $a_p$, $a_\sigma$ and $a_s$ are constants, the mean and fluctuating pressure
difference terms have been written as pressure coefficients
\begin{equation}
\Delta c_p=\frac{\overline{\Delta p}}{\frac{1}{2} \rho U_\textup{ref}^2},
\end{equation}
\begin{equation}
\sigma_{\Delta c_p}=\frac{\sigma_{\Delta p}}{\frac{1}{2} \rho U_\textup{ref}^2},
\end{equation}
$\rho$ being the ambient air density, and
\begin{equation}
U_L'=\frac{U_L}{U_\textup{ref}}.
\end{equation}

The pressure-difference mechanisms \ref{ss-mean-mech} and \ref{ss-unsteady-mech} are
dominant for the 2-opening case and \ref{ss-shear-mech} is negligible, while, in
contrast, in the case of a single opening, shear is dominant. Hence we take $a_s=0$ for
2-opening cases and $a_p = a_\sigma = 0$ for one opening. Henceforth we focus on the
2-opening case, although a procedure is given at the end of the section for how to
approximate a greater number by just two.

Figure \ref{fig:ss-plan-view-two-openings} shows the geometry of this case in plan view.
The two openings are a distance $s$ apart in a fa\c{c}ade of width WB whose normal has
azimuthal angle $\varphi_n$. The wind azimuthal direction is $\varphi_w$, so that its
direction relative to the fa\c{c}ade is 
\begin{equation}
\theta=\varphi_w-\varphi_n.
\end{equation}
In these definitions, $\ang{0} \leq \varphi_n, \varphi_w \leq \ang{360}$, while
$|\theta| \leq \ang{180}$. Note that if $\theta$ initially falls outside the range
indicated it should be replaced by $\sgn(\theta) \left[|\theta| - \ang{360}\right]$.

According to the above discussion, the flow rate is modeled as comprising two terms,
arising, respectively, from the mean and fluctuating components of the pressure
difference between the two aperture locations:
\begin{equation}\label{eqn:ss-flow-two-terms}
Q'=\left(a_p |\Delta c_p |+a_\sigma \sigma_{\Delta c_p} \right)^{1/2}
\end{equation}

\begin{figure}[hbtp] % fig 1
\centering
\includegraphics[width=0.9\textwidth, height=0.9\textheight, keepaspectratio=true]{media/ss-figure1.png}
\caption{Plan view of building with 2-opening fa\c{c}ade. \protect \label{fig:ss-plan-view-two-openings}}
\end{figure}

There are two parts to deriving the detailed form of the model:
\begin{enumerate}
\item Express the mean and fluctuating pressures in terms of other physical quantities
that are more readily accessible. EnergyPlus does not have the information on variations
in pressure that is needed to compute $\Delta c_p$ and $\sigma_{\Delta c_p}$. Analysis
of wind tunnel pressure data (Linden et al., 2013) allows for the derivation of simple
expressions for these quantities in terms of the wind direction and opening separation
$s$.\label{ss-item-the-first}
\item Find the coefficients $a_p$ and $a_\sigma$. This is achieved here by analysis of
ventilation rate data (Linden et al., 2013), and using linear regression on the
data.\label{ss-item-the-second}
\end{enumerate}

\subsubsection{Estimation of pressure difference coefficients}
The analysis of pressure data carried out showed that the contributions to the
right-hand side of Equation \ref{eqn:ss-flow-two-terms} can be written to first order
as
\begin{equation}
\Delta c_p=f(s',\theta)
\end{equation}
and
\begin{equation}
\sigma_{\Delta c_p}=g(s',\theta)
\end{equation}
where the dimensionless separation $s' = s/\textup{WB}$,
\begin{equation}
f(s',\theta)=s'\cdot\Pi(θ),
\end{equation}
and
\begin{equation}
g(s',\theta)=\Sigma_0+s'\cdot\Sigma(\theta).
\end{equation}
That is, to a first approximation the pressure difference mean and standard deviation vary linearly with opening separation, each modulated by its own function of relative wind angle $\theta$.

The formulae for the components are as follows.
\begin{equation}\label{eqn:ss-first-formula}
\Pi(\theta) = 
\begin{cases}
0.44\sgn(\theta)\sin(2.67|\theta|),& |\theta| \leq \theta_0 \\
-0.69 \sgn(\theta)\sin(288-1.6|\theta|),& \theta_0\leq|\theta|\leq\ang{180}.
\end{cases}
\end{equation}
\begin{equation}\label{eqn:ss-second-formula}
\theta_0 = \ang{67.5}
\end{equation}
\begin{equation}\label{eqn:ss-third-formula}
\Sigma(\theta) = 0.423 - 1.63\times 10^{-3}|\theta|
\end{equation}
\begin{equation}\label{eqn:ss-last-formula}
\Sigma_0=0.24
\end{equation}

Figures \ref{fig:ss-figure-2-upper} and \ref{fig:ss-figure-2-lower} show an example of $\Delta c_p$ and $\sigma_{\Delta c_p}$ generated by Equations \ref{eqn:ss-first-formula} through \ref{eqn:ss-last-formula}, respectively, compared with the data.
%
\begin{figure}[hbtp] % fig 2
\centering
\includegraphics[width=0.9\textwidth, height=0.9\textheight, keepaspectratio=true]{media/ss-chart1.png}
\caption{Mean pressure difference, $\Delta c_p$ predicted by Equations \ref{eqn:ss-first-formula} and \ref{eqn:ss-second-formula}, compared with experimental data. The fa\c{c}ade containing the openings is facing South and the separation is large ($s' = 0.85$). \protect \label{fig:ss-figure-2-upper}}
\end{figure}
%
\begin{figure}[hbtp] % fig 2
\centering
\includegraphics[width=0.9\textwidth, height=0.9\textheight, keepaspectratio=true]{media/ss-chart2.png}
\caption{Standard deviation of pressure difference, $\sigma_{\Delta c_p}$, predicted by Equations \ref{eqn:ss-third-formula} and \ref{eqn:ss-last-formula}, compared with experimental data. The fa\c{c}ade containing the openings is facing South and the separation is large ($s' = 0.85$). \protect \label{fig:ss-figure-2-lower}}
\end{figure}

This completes item \ref{ss-item-the-first} above, i.e. the approximation of the two pressure coefficient terms in Equation \ref{eqn:ss-flow-two-terms} in terms of the dimensionless opening separation and the relative wind angle. There remains item \ref{ss-item-the-second}, i.e. fitting the flow rate data to the formula-generated pressure parameters to determine the constants $a_p$ and $a_\sigma$. When this was carried out, the best fit was found to occur with $a_p = 0.173$ and $a_\sigma = 0.042$. Figure \ref{figure-3-in-original} shows the comparison between the flow rate prediction and the experimental data.

Combining with the previous results we have
\begin{equation}\label{eqn:ss-combined-eqn}
\frac{Q}{A_\text{in} U_\text{ref}}=\left(0.01+\left[0.173 |\Pi(\theta)|+0.042 \Sigma(\theta)\right]s'\right)^{1/2}
\end{equation}

The flow Q is defined to be positive when opening \#1 is the inlet and opening \#2 is the outlet. Since $\Delta p$ is defined as $p_1-p_2$, this can occur when $\Delta c_p > 0$, which in turn corresponds to certain relative wind angles, i.e.
\begin{equation}
\begin{gathered}
Q < 0\ \text{otherwise}\\ 			
Q > 0\ \text{if}\ \ang{0} < \theta < \theta_0\ \text{or}\  \ang{-180} < \theta < -\theta_0
\end{gathered}
\end{equation}

Note that the sense of $Q$ is not necessarily well-defined for all such angles, e.g.
when the unsteady contribution in Equation \ref{eqn:ss-flow-two-terms} is significant:
nevertheless, we retain this definition for all angles.

\begin{figure}[hbtp] % fig 3
\centering
\includegraphics[width=0.9\textwidth, height=0.9\textheight, keepaspectratio=true]{media/ss-figure3.png}
\caption{Comparison of predicted flow rates with experimental data. Each graph shows experimental data (red diamonds) plotted against model prediction (gray squares). Cases in the left-hand column, (a)-(c), are for $s'=0.75$, those in right-hand column, (d)-(f), $s'=0.32$. The three rows are for 2-story isolated, 4-story isolated and 2-story with low density surroundings, respectively. \protect \label{fig:ss-figure-3-in-original}}
\end{figure}

\subsubsection{Inversion of flow rate to equivalent pressure difference}
Finally, we invert the flow rate to obtain the equivalent pressure difference. The purpose of this procedure is to allow incorporation of the model into the network airflow part of EnergyPlus, which requires pressure coefficients to calculate the flow rate: in this way the model will become essentially transparent to the user.
A nodal representation of the 2-aperture single-sided case is shown in Figure \ref{fig:ss-pressure-network-two-openings}.

\begin{figure}[hbtp] % fig 4
\centering
\includegraphics[width=0.9\textwidth, height=0.9\textheight, keepaspectratio=true]{media/ss-figure4.png}
\caption{Pressure network for 2-opening single-sided room. \protect \label{fig:ss-pressure-network-two-openings}}
\end{figure}

Here, 0 is the  ambient, 1 is at the upstream outer surface, 2 is at the downstream outer surface, and R is the room.

Let the openings have areas $A_1$ and $A_2$, not necessarily equal, and a discharge coefficient $C_d$, which is assumed equal, and let $Q$ be the flow rate through the system. Then it can be shown that the pressure difference between the two openings, $\Delta p_{12}$, is related to the flow rate by
\begin{equation}\label{eqn:ss-delta-p-1-2}
\frac{\Delta p_{12}}{\frac{1}{2}\rho V^2} = \frac{1}{C_d^2}\left(\frac{Q/A_\text{eff}}{V}\right)^2
\end{equation}
where the effective area $A_\text{eff}$ is defined by
\begin{equation}
A_\text{eff}=\frac{A_1 A_2}{\left(A_1^2+A_2^2\right)^{1/2}}
\end{equation}
and $V$ is a reference velocity. The left-hand side of Equation \ref{eqn:ss-delta-p-1-2} is a pressure coefficient, $\Delta c_{p,12}^{E+}$, say, and our task is to express this in terms of the flow rate in Equation \ref{eqn:ss-combined-eqn} but using the definition of a pressure coefficient appropriate to EnergyPlus.

The first step is to re-write Equation \ref{eqn:ss-delta-p-1-2} as follows:
\begin{equation}
\Delta c_{p,12}^{E+} = \frac{1}{C_d^2}  \left(\frac{A_\text{in}}{A_\text{eff}}\right)^2 \left(\frac{U_\text{ref}}{V}\right)^2 Q'^2
\end{equation}
where $Q' = Q/(A_\text{in}U_\text{ref})$ is the non-dimensional flow rate of Equation \ref{eqn:ss-combined-eqn}, which is expressed in terms of $Q$, the window area $A_\text{in}$ and the wind speed at 10 m height, $U_\text{ref}$.

The factor $\left(A_\text{in}/A_\text{eff}\right)^2 = 2$, since $A_1 = A_2 = A_\text{in}$, and therefore $A_\text{eff} = A_\text{in}/\sqrt{2}$. The factor $(U_\text{ref}/V)$ relating the velocity scale used in the correlation with that used in the EnergyPlus pressure coefficient may be simplified by noting the following.

EnergyPlus assumes the surface pressure due to wind, $p_w$, is derived from a pressure
coefficient $C_p$ using the local wind speed at \textit{window height}, $V_\text{ref}(z_win)$,
i.e.
\begin{equation}
p_w=\rho_a \left[V_\text{ref}(z_\text{win}\right]^2 C_p/2
\end{equation}
Hence the reference velocity $V =  V_\text{ref}(z_\text{win})$.

EnergyPlus characterizes the vertical wind profile as a power law in z, defined in terms of the local atmospheric boundary layer depth $\delta$ and a power law exponent $\alpha$. EnergyPlus also uses an ``interpolation'' formula to derive the above local wind speed from that measured at the meteorological site, $U_\text{met}$, according to
\begin{equation}
V_\text{ref}(z)= U_\text{met} \left(\frac{\delta_\text{met}}{z}\right)^{\alpha_\text{met}}
\left(\frac{z}{\delta}\right)^\alpha
\end{equation}
where the subscript ``met'' denotes conditions at the meteorological site, and the default values are $z_\text{met} = 10\ \text{m}$, $\delta_\text{met} = 270\ \text{m}$ and $\alpha_\text{met} = 0.14$. The user can change the site parameters from their default values.

The correlation in Equation \ref{eqn:ss-combined-eqn} assumes the reference velocity is the wind speed at 10 m, which will be taken as the local velocity at the building, $V_\text{ref}(10)$.
Combining these results, the pressure coefficient representing the difference in pressure between the two openings is given by
\begin{equation}
\Delta c_{p,12}^{E+} = \frac{1}{C_d^2} \left[ \frac{V_\text{ref}(10)}{V_\text{ref}(z_\text{win})} \right]^2 \left(0.02+\left[0.346 |\Pi(\theta)|+0.084 \Sigma(\theta)\right]∙s'\right)
\end{equation}
where the velocity ratio
\begin{equation}
\frac{V_\text{ref}(10)}{V_\text{ref}(z_\text{win}} = \left(\frac{10}{z_\text{win}}\right)^\alpha
\end{equation}

Thus, if we assign a pressure coefficient $+0.5\Delta c_{p,12}^{E+}$ to opening 1 and $-0.5\Delta c_{p,12}^{E+}$ to opening 2, then this will provide the necessary pressure difference to give the flow in \ref{eqn:ss-combined-eqn} in a network context.

Note: the pressure coefficient $\Delta c_{p,12}^{E+}$ refers to the difference in pressure between the two openings and is defined in terms of the wind speed at window height. If this is to be combined with a background pressure coefficient for the fa\c{c}ade it is important to ensure the two are defined using the same reference velocity.

\begin{figure}[hbtp] % fig 5
\centering
\includegraphics[width=0.9\textwidth, height=0.9\textheight, keepaspectratio=true]{media/ss-figure5.png}
\caption{Local velocity as function of wind angle for wind tunnel data analyzed. On the 2-story graph `WP85' refers to Warren and Parkins (1985), whose measurement location is most closely matched by the ``Center'' curve. \protect \label{fig:ss-pressure-network-two-openings}}
\end{figure}
%
\begin{figure}[hbtp] % fig 6
\centering
\includegraphics[width=0.9\textwidth, height=0.9\textheight, keepaspectratio=true]{media/ss-figure6.png}
\caption{Wind tunnel data, averaged over all cases, plotted with curve to fit data. \protect \label{fig:ss-pressure-network-two-openings}}
\end{figure}

\subsubsection{Testing/Validation/Data Source(s)}
The SS model was developed using wind tunnel pressure and flow rate data acquired by CPP Wind Engineering \& Air Quality Consultants, who carried out two series of tests: (a) Closed Box tests, with a sealed building, to obtain pressure and velocity data and (b) Ventilation tests, with one or two apertures open, to obtain flow rate data. See Linden et al. (2013) for full details.

Each experimental configuration consisted of a cuboidal building with or without a set of blocks representing nearby buildings. 

\begin{itemize}
\item Building: either 2-story, with WB = 47.5 cm, DB = 19.3 cm, HB = 9.9 cm, or 4-story, with HB = 19.8 cm. The model has a nominal scale of 1:70 (based on earlier tests).
\item Environment: building either isolated or at center of a 3 x 3 array of equal-sized blocks, which were of same aspect ratio and either low (2-story height) or high (4-story height), and with either narrow spacing (approximately 0.5WB between blocks) or wide spacing (approximately WB), giving a total of 5 different environments.
\item Approach flow: suburban boundary layer, approximately 10 m/s at 1 m height. The model was mounted on a turntable allowing any desired wind direction to be used. Increments of $\ang{11.25}$ and $\ang{22.5}$ were used for the Closed Box and Ventilation runs, respectively.
\item Room and openings: the building envelope contained a second-floor room of dimensions WRM = 45.9 cm, DRM = 17.7 cm and HRM = 5.0 cm. A total of 15 opening positions were available, distributed around the room. These were sealed for the Closed Box tests but opened in various combinations for the Ventilation runs. 
\item Sensors: 56 pressure transducers, arranged around the opening positions, including 32 at mid-opening height, recorded static pressure, while hot film sensors at 6 different positions recorded flow speed near the surface. For Ventilation runs the decay in concentration of a tracer was measured at two locations inside the room.
\end{itemize}

The data used to develop the model consisted of 6 scenarios: 3 building/environment cases,
\begin{itemize}
\item 2-story isolated building
\item 4-story isolated building
\item 2-story building with low, widely-spaced blocks
\end{itemize}
combined with 2 choices of opening position on the South (long) fa\c{c}ade, namely wide separation ($s' = 0.75$) and medium separation ($s' = 0.32$). 

\subsubsection{References}
