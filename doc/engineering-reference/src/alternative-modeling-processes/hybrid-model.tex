\section{Hybrid Model}\label{hybrid-model}

\subsection{Overview}\label{overview-004}

Building energy retrofits offer the most cost effective means to address greenhouse gas reductions from improving energy efficiency. The global energy efficiency market in the building sector was $68.2 billion in 2014, and is estimated to grow $127.5 billion in 2023 [1]. Building retrofit projects typically rely on tools for analysis of the energy saving, energy cost saving, and life cycle cost, and etc. Retrofit tools are mostly coupled with energy modeling and simulation that quantify energy savings from energy conservation measures [2]. However, there are gaps and limitations on those retrofit tools caused mainly by the ease of use and accuracy as a coupled simulation engine requires many inputs that are not easy to collect thus leading to uncertain results [3,4]. 
Retrofit projects typically use calibrated energy models to ensure that building systems are properly modeled and integrated together to predict the building energy performance correctly. There are challenges on calibrating energy models in measurement and verification of energy savings. Firstly, it needs to filter parameters that significantly influence to energy consumption calculation results. [5,6] provide technical approaches to filter the most influencing parameters for energy model calibration. Calibrations are typically based on: 1) expert’s rule of thumb or evidence-based [7,8], and 2) the recent automated methods backed by mathematical algorithms considering uncertainties [9–12]. The challenge in expert’s calibration is that energy models are typically very complex and have far more input parameters that can be manually analyzed by users, thereby it needs a significant domain expertise. Automated calibrations are typically backed by systematic technical methods that identify parameters with predefined ranges. Calibration needs a careful application. The result of the automated calibration, the calibrated energy model may correspond to an unrealistic building configuration as it is from an automated problem solving that has many solutions. This addresses the need for a new energy modeling method for existing buildings for retrofit analysis, which reducesavoids uncertainties in inputs and overcomes gaps in calibrated energy models.
There are two common approaches to energy modeling: data-driven black-box models, and physics-based calculation white-box models [13]. In recent years, data-driven energy models are getting attention as they provide a discovery of models from large volumes of data. Data-driven models require monitored data and train them to fit an energy model derived inversely from controlled iteration process and statistical methods [14,15]. However, the method is building specific and requires significant amount of data to train and validate the model. The data-driven model is a pure black box building upon unique data of the selected building [16]. Data collection is a tedious process, and some input parameters are difficult to have right data. The approach requires detailed actual measured data of the building performance for more precise model development. In practice, a long- or short-term monitoring is usually conducted to train dynamic parameters, which helps enhance the calculation for existing buildings (Braun and Chaturvedi 2002; Yuna Zhang et al. 2015). The data-driven models require training data which may not be available for most of buildings, and it is always a question whether a data-driven model based on training data for specific buildings applies to others. Another use of the data-driven inverse model was introduced to replicate building stock energy consumption using energy consumption survey data based on inference methods [18]. This tends to be costly and time consuming, therefore, making it difficult to scale up to be applied in different buildings. 
On the other hand, the calculation-based approach is based on physics-based approachmethod to calculates building energy consumption coveroffering a wide spectrum of complexity from low fidelity, aka reduced order and steady-state calculations, models [19–21] to high fidelity, dynamic simulation, models depending on analysis purposes [22–24]. The physics-based models (forward modeling approach) generally takes the physical parameters that describe the building as input, which can include building location, local weather, geometry, envelope construction materials, operational schedule, and HVAC system type, etc. The physics-based models are typically used in the design phase to facilitate building designers to make design decisions from evaluations of the energy performance. Decades of research have brought development of various energy modeling methods and reviews of them [25–27], offering many calculation tools [28]. Typically, these physics-based energy models require a significant number of building input parameters that some of them are quite uncertain or difficult to obtain in reality. 
Part I of this paper explores a hybrid modeling approach that integrates physics-based and data-driven modeling methods. The hybrid approach aims to enhance the current energy retrofit practices not only offering more user-friendly energy modeling environments but also providing more accurate estimates of energy savings at the same time. The hybrid model approach in this paper avoids input parameters that are hard to measure and replace them to easily measurable data. Parameters such as interior thermal mass and air infiltration rates are required in physics-based models, and they have significant impacts as they are driving factors for the dynamic performance of buildings. An accurate estimate of interior thermal mass has been a difficult problem because the building usually has various amounts of furniture and changeable partitions. The air infiltration rate changes in time and dynamically interacts with indoor and outdoor climatic conditions. However the accurate estimation of the data is almost impossible to collect without a fan pressurized test, which can not be easily done by typical energy modelers [29]. Part 1 of this paper focuses on the hybrid modeling approach on interior thermal mass and infiltration airflow providing a comprehensive review of how these parameters are handled in current building energy simulations. It introduces an approach that represents the interior thermal mass in EnergyPlus simulation, one with traditional interior mass object and the other with capacitance multiplier approach and compares two approaches with controlled simulation results. Later sections present fundamentals of the infiltration modeling in energy simulation and provides mathematical derivation of the hybrid model. Part II of this paper provides further details of implementation of the hybrid modeling features in a custom branch of EnergyPlus, and experiment results using the LBNL's  Facility for Low Energy eXperiment (FLEXLAB)[46] that demonstrates and validates the developed hybrid model.

\subsection{Model Description}\label{model-description-001}

Solving building energy and environmental problems inversely using measured data gets more attention as more data are easily and freely available. [30]. Measurements are to supplement to reduce discrepancies or to identify model parameters, nevertheless the majority of efforts go into the derivation of the dynamic inverse modeling. Inverse modeling is a discipline that applies mathematical techniques to combine measurements and models. Inverse modeling can provide solutions when direct measurements of model parameters are not widely available, rendering the use of numerical techniques. It is used widely in the field of data assimilation where it serves as a method for estimating the state of a system and for determining optimal values of uncertain model parameters [31]. 
The paper explores a new hybrid modeling approach uses the inverse modeling method. The hybrid modeling applied to the building energy simulation aims to improve the accuracy of the building energy simulation for existing buildings, which adds measured data to solve uncertain model parameters. The hybrid modeling approach builds upon the virtue of the physics-based model taking advantage of measured data. The approach uses measured zone air temperature to replace highly uncertain parameters such as internal thermal mass and infiltration airflow rate in the zone heat balance calculations by solving the reformulated zone heat balance equations. Figure 1 illustrates a conceptual diagram that easily express the concept of the hybrid modeling approach. 
 
Figure 1 Hybrid model conceptual diagram
Temperature data are easily available nowadays and are used for controls of indoor environments due to a wider use of low-cost thermostats with data loggers. Thermal mass plays an important role in energy modeling in predicting the transient cooling or heating loads and in strategizing heating, ventilation, and air conditioning (HVAC) system controls. There has been numerous research on energy efficient design and reducing peak cooling demand using thermal mass  [32–36]. Building envelop takes a significant amount of thermal mass. Physical details of envelop model parameters such as density, volume, and specific heat capacity can be easily determined per construction documents. However the internal thermal mass has not been well highlighted in most of building energy simulation practices. Components such as furniture, partitions have mass with mass with thermal capacitance cannot ignorable in the thermal heat transfer models. Although internal thermal ass has a substantial influence in prediction of cooling or heating requirements, it is very difficult to obtain detailed physical properties required for inputs in the current simulation tools [37,38]. This is one key factor leading to high uncertainty for energy performance analysis and simulations results.
There have been previous research efforts in exploring ways to estimate the thermal mass of zones. Braun et al provided thermal mass control strategies, including interior thermal mass estimates that optimize the heating and cooling energy cost savings [17,39]. Their studies develop a simplified heat balance equation, and derived the inverse model, using short term measured data to identify control strategies shifting and reducing peak cooling loads. It provides background on the concept and the problem of optimizing zone temperature set-points. Wang et al provided a method to estimate the building interior thermal mass, using a thermal network structure of lumped thermal masses and operational data to estimate the lumped parameters [38]. A genetic algorithm was used to estimate the lumped interior thermal parameters. U.S. National Renewable Energy Laboratory (NREL) studied details of reference models for estimating the interior thermal mass in U.S. commercial buildings [40]. The reference model includes representative typical inputs for interior mass with assumptions of standard wood, medium smooth interior furnishing conditions. The references help understand the significance of the interior mass in the energy models. These approaches have limitations such as requiring longer hours of data and additional input data in their custom energy models, which burden the use in general cases. 
The hybrid modeling approach can overcome the problem, which does not require the input of interior thermal mass in energy simulations. The study explores an approach that derives physical characteristics of the interior thermal mass. The approach solves the zone heat balance equation using measured zone air temperature (an input to the hybrid model) without the need of traditional input of zone interior thermal mass (an output of the hybrid model). The zone interior thermal mass in the hybrid modeling is represented as a temperature capacitance multiplier for the zone air thermal mass. The capacitance multiplier is calculated based on the input of zone air temperature and all other traditional inputs to energy models. The calculated capacitance multiplier is then used in normal energy modeling and simulation to calculate cooling and heating loads and energy use in buildings. It should be noted that the derived algorithms are generic and can be adopted by other building energy modeling tools. The study uses EnergyPlus to demonstrate the approach. The paper provides technical details to derive algorithms for estimation of the interior thermal mass, validation, implementation in EnergyPlus, and verification of the simulation results.
	Zone air heat balance algorithm
The hybrid model algorithms are built upon the physics-based zone heat balance equation reformulated to solve a partially inverse problem. The approach inverses the physics-based energy model, reformulating the heat balance algorithm with measured zone air temperature data (traditionally results/output of the physics-based model) to solve highly unknown parameters for internal thermal mass and infiltration air flow rates (traditionally input of the physics-based model). 
EnergyPlus [41] was selected as a simulation engine for development, demonstration, and validation of the hybrid modeling. EnergyPlus, an open source program that can model ventilation, cooling, lighting, water use, renewable energy generation and other building energy flows. EnergyPlus includes many innovative simulation capabilities including time-steps less than an hour, modular systems and plant integrated with heat balance-based zone simulation, multi-zone air flow, thermal comfort, water use, natural ventilation, renewable energy systems, and user customizable energy management system. EnergyPlus has been extensively managed by DOE for new releases every six months for new simulation features, their example files and documentation. EnergyPlus enables scientists, engineers, building professionals evaluating the building energy performance and testing new simulation modules of new features [42]. These make EnergyPlus ideal for the implementation and test of the hybrid modeling method. 
EnergyPlus provides algorithms to solve the zone air energy balance equation that uses the analytical solution to calculate the derivative term respect to time. The basis for the zone air system integration is to formulate energy balances for the zone air as shown in Equation (1) and (2) and solve the resulting ordinary differential equations. 

C_z  〖dT〗_z/dt=∑▒Q_int +∑▒〖〖h_s A〗_s (T_s-T_z ) 〗  +∑▒〖m ̇_iz C_p (T_iz-T_z ) 〗+m ̇_inf C_p (T_o-T_z )+Q_sys	(1)

C_z=〖Vρ〗_air C_p C_T	(2)


The sum of zone loads and the provided air system energy equals the change in energy stored in the zone. Typically the zone capacitance, C_z includes the zone air only when formulating energy balances for the zone air. The internal thermal mass, including furniture, books, and changeable partitions, is assumed to be in thermal equilibrium with the zone air, thus it is added in the zone heat capacitance, C_z.  The infiltration airflow rate, q_inf changes for different conditions depending on outdoor temperature, wind speed, and HVAC system operations. The energy provided from systems to the zone is represented as Q_sys. 
EnergyPlus provides heat balance solution algorithms of 3rd order backward difference and analytical solution to solve the zone air energy balance equation. The 3rd order finite difference approximation provides stability without requiring a prohibitively small time step, the method still has truncation errors and requires a fixed time step length for the previous three simulation time steps. Therefore, different time step lengths for the previous three simulation time steps may make the temperature coefficients invalid. The analytical solution algorithm is an integration approach that provides a possible way to obtain solutions without truncation errors and independent of time step length and only requires the zone air temperature for one previous time step [24]. The hybrid modeling approach uses the analytical solution for internal thermal mass inverse calculation and the 3rd order backward difference for infiltration inverse calculation. EnergyPlus code for these heat balance algorithms are referenced to the ZoneTempPredictorCorrector module.
	Internal thermal mass hybrid modeling method
The EnergyPlus engineering reference [24] provides assumptions of handling the zone internal thermal mass in simulations. There are two approaches to model internal thermal mass in EnergyPlus. One approach is to use the internal mass objects to define construction specifications of internal furnishing materials, and the other is to use the temperature capacitance multipliers. The multiplier increases zone air capacity as it represents the effective storage capacity of the zone interior thermal mass. Figure 1 illustrates two approaches of representing interior thermal mass in EnergyPlus.
 
Figure 2 Two approaches of representing interior thermal mass in EnergyPlus

Interior mass objects in EnergyPlus modeling
The EnergyPlus object, “InternalMass”, is used to specify the construction materials and area of interior mass within the space, which are important to heat transfer calculations. Internal mass objects participate in the zone air heat balance and the longwave radiant exchange. The geometry of the internal mass construction is greatly simplified due to the difficulty of measurement. They do not directly interact with the solar heat gain because internal mass objects do not have a specific location in space. Internal mass objects can represent multiple pieces of interior mass (furniture, partitions) with different constructions. Internal mass exchanges energy through its both surfaces with the zone by convection. 
Zone capacitance multiplier
There is an EnergyPlus object, “ZoneCapacitanceMultiplier:ResearchSpecial”, an advanced feature to specify the effective storage capacity of a zone. The capacitance multiplier of 1.0 by default indicates the capacitance comes from only the air in the zone. This multiplier can be increased if the zone air capacitance needs to be increased for stability of the simulation or to allow modeling higher or lower levels of damping behavior over time. This multiplier is used in the zone predictor-correction algorithm to adjust the zone air thermal capacity. Currently EnergyPlus assumes the same constant capacitance multiplier for all zones. Although EnergyPlus allows users modifying this multiplier, it is not easy to determine the accurate value and not common for a typical use of EnergyPlus. It can be expanded to individual zones so that they can have a different capacitance multiplier to capture the effective internal mass depending on the interior furnishing configuration. 
The use of the internal mass multiplier, the zone temperature capacitance multiplier only corrects the zone air heat capacity reflecting heat stored in the internal mass. Assumptions are not different from the approach used in InternalMass object, which ignores the geometrical construction of the internal mass, and do not contribute to the heat transfer across surfaces and the solar heat gain through windows.  The approach in this hybrid modeling method derives the interior mass by solving the zone temperature capacity multiplier. The derivation is based on the inverse modeling method replacing the input of interior thermal mass with the measured zone air temperature. The zone air temperature is the only additional requirement for the proposed approach.
Inverse algorithm for zone capacitance multiplier
The interior thermal mass including furniture, books, and changeable partitions, is assumed to be in thermal equilibrium with the zone air, thus it is added in the zone heat capacitance, C_z.  The interior thermal mass is assumed to be in equilibrium with the zone air. The interior mass in the current EnergyPlus model uses a capacitance multiplier, C_T indicating the capacitance as part of the air in the volume of the specified zone. The default value is given as 1.0 corresponding to the total capacitance for the zone’s volume of air at current zone conditions. The hybrid model to derive internal thermal mass uses the capacitance multiplier that indicates the capacitance added to the zone air. The added internal thermal mass capacitance will be represented in the multiplier, which the value would be greater than 1.0. The formulation starts with the heat balance on the zone air. Equation (3) calculates the time-series zone air temperature, T_z reformulating Equation (1) using the analytical solution method. 
T_z^t=(T_z^(t-δt)-(∑▒Q_int +∑▒〖h_s A_s T_s 〗  +∑▒〖m ̇_iz C_p T_iz 〗+〖m ̇_inf C_p T_o+m ̇〗_sys C_p T_sup^t)/(∑▒〖h_s A〗_s +∑▒〖m ̇_iz C_p 〗+m ̇_inf C_p+m ̇_sys C_p ))×e^((-(∑▒〖h_s A〗_s   +∑▒〖m ̇_iz C_p 〗+ m ̇_inf C_p+m ̇_sys C_p)/(C_z^t ) δt))+(∑▒Q_int +∑▒〖h_s A_s T_s 〗  +∑▒〖m ̇_iz C_p T_iz 〗+〖m ̇_inf C_p T_o+m ̇〗_sys C_p T_sup^t)/(∑▒〖h_s A〗_s +∑▒〖m ̇_iz C_p 〗+m ̇_inf C_p+m ̇_sys C_p )	(3)

The hybrid modeling approach derives the internal mass by solving the heat capacity of zone air and internal thermal mass, C_z. Equation (4) shows the inverse heat balance algorithm that replacing the zone air temperature, T_z with the measured zone air temperature. The current timestep measured temperature, T_z^t  and the previous timestep measured temperature, T_z^(t-δt) are given from inputs, them the zone air heat capacity, C_z^t for each timestep is expressed as following. 
C_z^t=-(∑▒〖h_s A_s 〗+∑▒〖m ̇_iz C_p m ̇_inf C_p  〗+m ̇_sys C_p )δt/ln⁡[(T_z^t-(∑▒Q_int +∑▒〖h_s A_s T_s 〗  +∑▒〖m ̇_iz C_p T_iz 〗+〖m ̇_inf C_p T_o+m ̇〗_sys C_p T_sup^t)/(∑▒〖h_s A_s 〗+∑▒〖m ̇_iz C_p 〗+m ̇_inf C_p+m ̇_sys C_p ))/(T_z^(t-δt)-(∑▒Q_int +∑▒〖h_s A_s T_s 〗  +∑▒〖m ̇_iz C_p T_iz 〗+〖m ̇_inf C_p T_o+m ̇〗_sys C_p T_sup^t)/(∑▒〖h_s A_s 〗+∑▒〖m ̇_iz C_p 〗+m ̇_inf C_p+m ̇_sys C_p ))] 	(4)


There are different conditions in deriving the interior thermal mass depending on the air system operation. The initial process assumes that calculation is conducted when HVAC systems are off. If HVAC systems are turned off during unoccupied hours, the zone heat capacity can be determined without additional inputs of the supply air temperature for the above condition. The condition when the air system is off, Q_sys=0, this induces T_sup^t=T_z^t.  Zone air capacitance with internal mass, C_z  remains constant, not changing with the given time series. When the air system is operating, Q_sys is not zero. This requires input values of the supply air temperature as denoted as T_sup^t and supply air volume. These are additional inputs when estimating interior thermal mass under the condition when HVAC system is operating. When the air system is operating, C_z  〖dT〗_z/dt becomes zero or almost zero. The zone temperature is maintained at the set-point temperature, thus the temperature difference between T_z^t  and T_z^(t-δt) is zero. For those hours zone temperatures are maintained at the set-point temperature, the interior thermal mass, C_z^t is not calculated. 
Zone air heat capacity needs to be derived from the stabilized internal zone air temperature data that fully captures the stored heat in the air and internal thermal mass. It is recommended the measured zone temperature needs to be at least one week of data for more reliable result. Zone heat capacity is an important component for buildings as it stabilizes interior temperatures, thus at least one week of the measured interior temperature can capture the stored heat in the interior thermal mass. The temperature capacity multiplier i.e., internal mass multiplier, C_T^t is calculated for each time step using Equation (5).
C_T^t=(C_z^t)/(〖Vρ〗_air C_p )	(5)


The default value is 1.0. Ideally the zone heat capacity shall remain constant for the same condition of the interior environment in the zone heat balance equation. An underlying assumption is that the zone heat capacity is treated as constant for the equilibrium of the inversed heat balance model. However the measured temperatures are not the same as the simulated zone air temperatures which is the result of the energy simulation in Equation (3). This causes the internal mass multiplier, C_T^t, the result from the inverse model is not constant during the course of the simulation period. The hybrid model will determine a time span when |T_z^t ┤-├ T_z^(t-δt) ┤|>0.1°C that C_z^t remains more constant. Internal mass multiplier calculations are only done when the zone air temperature difference between timesteps meets the condition. This filter is needed for more reliable inverse calculation to avoid the anomaly conditions due to the use of the inverse model.  
Validation of the capacitance multiplier method
We investigated two approaches and compared their simulation results using EnergyPlus energy model under diverse climate conditions and physical constructions for the validation of the feasibility of the capacitance multiplier approach. The DOE small office reference model [40] was used for EnergyPlus simulations that represent internal masses in two different ways. Simulation uses four locations representing hot, mild, hot-summer and cold-winter, and cold climates, and two vintages for pre-1980 and 2004 conditions. Table 1 shows simulation settings for the two approaches to represent the internal thermal mass in EnergyPlus modeling. The first approach uses the InternalMass object to model the internal furnishing construction with material properties. The internal mass construction is uniformly applied to each zone, which has 6-inch wooden interior furnishing with a surface area double the zone floor area. The second approach uses the temperature capacity multiplier input in the ZoneCapacitanceMultiplier:ResearchSpecial object. The multiplier without any internal mass is 1.0, and it is assumed that the multiplier has a value of greater than 1.0 for a zone that is furnished with typical office equipment. To determine the best multipliers for different simulation settings, we conducted parametric runs using multipliers ranging from 1 to 20. 
Table 1 Simulation setup to compare two approaches representing interior mass
Two Approaches	Specification
EnergyPlus Object	Field	
InternalMass		Construction
	Surface Area		Construction name: Interior Furnishings
	Material: Standard 6 inch wood
	Surface area: Two times of the zone area
	Properties:
	Thickness: 0.15 m
	Conductivity: 0.12 W/m-K
	Density: 540 kg/m3
	Specific heat: 1,210 J/kg-K
ZoneCapacitanceMultiplier:ResearchSpecial	Temperature Capacity Multiplier	Parametric runs ranging from 1 to 20

 Firstly we conducted simulations of the reference models with InternalMass object to generate simulation results of the zone mean air temperature. Secondly multiple simulation runs were conducted using different temperature capacity multipliers in the ZoneCapacitanceMultiplier:ResearchSpecial object, and the same simulation results were collected from the parametric runs. Then the Normalized Mean Bias Error (NMBE) and Coefficient of Variance of Root Mean Square Error (CVRMSE) tests were used for validation. NMBE and CVRMSE are commonly used to determine the goodness of fit between two sets of data between energy simulation results [43].  Two sets of results are deemed to agree with each other if the NMBE and CVRMSE are no greater than 5% and 15% for monthly results or 10% and 30% for hourly results. For the validation the hourly zone mean air temperature from both approaches were compared. It was found that a multiplier of 8 provides the best fit between two results sets based on the NMBE and CVRMSE. Figure 3 presents the validation results for hourly zone air temperature for the multiplier of 8, which shows that the CVRMSEs are less than 5% by a wide margin and the NMBEs are less than 4%, which validates that the temperature capacity multiplier approach with a right multiplier represents well enough the internal mass modeled in InternalMass object with construction specifications.  

  

Figure 3 Hourly zone mean air temperature validation result for multiplier 8

	Infiltration hybrid modeling method
Infiltration airflow modeling
Infiltration is an uncontrolled outside air into a building depending on the air-tightness of the building envelope and indoor / outdoor climate conditions. A large percentage of the total energy loss of a building can occur through envelope leakage, resulting in excessive heating and cooling loads. Infiltration is represented as the volumetric flow rate of outside air into a building. Infiltration inputs in energy modeling rarely reflect the actual building operating conditions, as the infiltration airflow rate is dynamic and difficult to measure. It is generally understood that the infiltration rate of a building is a function of its age, its construction quality, and weather conditions. Wind speed and temperature at the zone height are driving factors which cause the pressure difference between the outside and the inside of the building. DOE reference energy models specify representative infiltration air flow rates that can be used for energy simulations for different vintages [40]. Table 2 shows infiltration air change per hour (ACH) values and their volumetric airflow rates used for the reference office models. The infiltration design air flow rate is based on the pressure difference of 4 Pa. This lower pressure was assumed to be the average pressure difference across the envelope without pressurization or depressurization from the HVAC and exhaust fans. EnergyPlus assumes that the uncontrolled infiltration design air flow rate is reduced to 25% when the ventilation system are running with a typical HVAC operation condition. When the ventilation system is off, the infiltration is assumed to be the full leakage design flow rate [29].
Table 2 Infiltration air flow rates represented in DOE reference office buildings
Vintage	Building type	Infiltration ACH in perimeter zones	Infiltration design air flow rate assumptions
Pre-1980	Small office	2.46	Infiltration in perimeter zones 0.001133 m3/s/m2 (0.22 cfm/ft2) per exterior surface area at 4 Pa
	Medium office	1.03	
	Large office	0.98	
Post-1980	Small office	2.46	
	Medium office	1.03	
	Large office	0.98	
New-2004	Small office	0.66	Infiltration in perimeter zones 0.000302 m3/s/m2 (0.06 cfm/ft2) per exterior surface area at 4 Pa
	Medium office	0.28	
	Large office	0.26	

The infiltration modeling is relatively simplified in the energy simulation because of the lack of knowledge about the sizes and distribution of cracks in the building envelope, the permeability of the envelope, the air flow to the building, and the pressure distribution in and around the building. EnergyPlus allows a simple approach to model the infiltration for users. This requires to define a design flow rate and coefficients for temperature and wind velocity using the ZoneInfiltration:DesignFlowRate object. EnergyPlus calculates airflow rates by adjusting for the indoor-outdoor temperature differences and the outdoor wind speed using Equation (6).
q_inf= q_(inf⁡_design ) F_schedule [A+ B|T_z-T_o |+C×v_wind+D×(v_wind )^2]	(6)


Where:
 A = constant coefficient 
B = temperature coefficient
C = velocity coefficient
D = velocity squared coefficient
v_wind	= wind velocity
F_schedule = a user-defined schedule value between 0 and 1
The simplified infiltration model using a constant infiltration flow rate is designed to capture the average effect over the year and in different locations. The simple infiltration approach has an empirical correlation that modifies the infiltration as a function of wind speed and temperature difference across the envelope. The difficulty in using this approach the determination of valid coefficients for each building type in each location. These coefficients vary and provide very different results that cause great uncertainty. This is not easy to identify correct ones for typical modeling practices. The current EnergyPlus simplified infiltration modeling method uses a fixed infiltration rate that can represent the average impact over a year. This may not be realistic for accurate energy modeling when capturing hourly dynamics. More complicated flow network simulations are necessary for detailed modeling. Other infiltration related EnergyPlus models that add details and complexities are the “Effective Leakage Area” model [44] using the ZoneInfiltration:EffectiveLeakageArea object  and “Flow Coefficient” model [45], using the ZoneInfiltration:FlowCoefficient object.
Infiltration inverse modeling
It is not easy to estimate as the infiltration is caused by various sources of unknown leakages. The development of the hybrid model fills the gap when estimating the infiltration that reflects all complexities of design flow rate, coefficients, and climate conditions by only requiring easily attainable zone air temperature data to derive zone. The development of the infiltration hybrid modeling algorithm is consistent with the EnergyPlus source code. The approach derives the infiltration mass flow rate, m ̇_inf by reformulating the zone air heat balance algorithm. The 3rd order backward difference method is used for the inverse model development for the infiltration hybrid modeling. The inverse model using the analytical solution cannot be realized in a mathematical form. The time-series zone air temperature, T_z using the 3rd order method is shown in Equation (7).
T_z^t=(∑▒Q_int +∑▒〖h_s A_s T_s 〗  +∑▒〖m ̇_iz C_p T_iz 〗+〖m ̇_inf C_p T_o+m ̇〗_sys C_p T_sup^t-(C_z/δt)(-3T_z^(t-δt)+3/2 T_z^(t-2δt)-1/3 T_z^(t-3δt)))/((11/6)  C_z/δt+∑▒〖h_s A_s 〗  +∑▒〖m ̇_iz C_p 〗+〖m ̇_inf C_p+m ̇〗_sys C_p )	(7)

Equation (8) shows the inverse algorithm for infiltration hybrid modeling method to derive the zone infiltration mass flow rate using the measured zone air temperature.
m ̇_inf  =(∑▒Q_int +∑▒〖h_s A_s T_s 〗  +∑▒〖m ̇_iz C_p T_iz 〗 〖+m ̇〗_sys C_p T_sup^t-(C_z/δt)(-3T_z^(t-δt)+3/2 T_z^(t-2δt)-1/3 T_z^(t-3δt) )-T_z^t ((11/6)  C_z/δt+∑▒〖h_s A_s 〗  +∑▒〖m ̇_iz C_p 〗 〖+m ̇〗_sys C_p))/(C_p (T_z^t-T_o))	(8)

The infiltration air flow rate, q_inf is then calculated from the derived infiltration mass flow rate from the Equation (9).
q_inf=m ̇_inf/ρ_air 
	(9)

For the infiltration mode of the hybrid model simulation, the calculation is only done when the zone air temperature difference between the current and previous timestep is less than 0.1°C and the zone air and outdoor air temperature difference is greater than 5 °C as depicted |T_z^t ┤-├ T_o^t ┤|>5.0 °C and |T_z^t ┤-├ T_z^(t-δt) ┤|<0.1°C
Discussions and Part II: Validation prologue
The hybrid modeling approach combines the physics-based forward model and the data-driven inverse models, taking advantage of the virtue of physics-based model and more available measured building data. A wider use of sensors to meet the need of better controls in existing buildings enables a new paradigm of the research in the building energy simulation. Interior thermal mass and infiltration are common uncertain parameters that hinder the accuracy of the energy simulation. Typical energy modelers do not have enough knowledge on these parameters, and even experienced ones often rely on default values provided by tools or values used in reference models, which do not necessarily reflect the real building conditions. The hybrid modeling is a novel simulation approach that eliminates this parameter with confident data, measured zone air temperature, which improves the accuracy of energy simulation for existing buildings with temperature data.  
The paper examines the temperature capacitance multiplier approach to capture the abstract of the physical characteristics for the interior thermal mass and infiltration airflow in the energy modeling and simulation. The mathematical expressions of the cause-effect relationships for physical systems are often a bottleneck when solving problems in complex built environments. When such mathematical expressions are representing physical systems, the correlations among various factors are intertwined with each other. This may cause an over-calibration on the internal mass parameter and infiltration which capture uncertainties of other modeling parameters when an energy model has other parameters not correct, or zone air temperature data are not correct. Thus, it is crucial to have good data to start with to improve the accuracy of the energy model using the hybrid model. The zone air temperature varies depending on where the temperature sensor is located. A temperature sensor near the exterior surfaces may be more sensitive to the exterior climate condition. Uncalibrated sensors also bring measurement errors.
The following paper, “A New Hybrid Modeling Approach Using Zone Air Temperature to Calculate Interior Thermal Mass and Infiltration, Part II: Validations” provides details of the validation of the hybrid modeling approach with controlled experiments using the LBNL's  Facility for Low Energy eXperiment (FLEXLAB) [46]. The Part II paper discusses the implementation of the hybrid model in a custom version of EnergyPlus and simulation results with new inputs of the measured data from the FLEXLAB experiment. The FLEXLAB experiment unveils the significance of the interior mass and air infiltration to the zone temperature and how they are captured inversely in the hybrid modeling. A guideline is also provided on the use of the hybrid model implemented in EnergyPlus.
Conclusions
The paper explores a new energy modeling method introducing a hybrid modeling approach that aims to overcome gaps in the current building energy simulations for existing buildings. The hybrid model improves the accuracy of simulated results and makes a simulation easier by replacing inputs such as internal thermal mass and infiltration airflow rates difficult to obtain with easily measurable zone air temperature data. The paper, Part I: Approaches presents details of the hybrid model algorithm that develops inverse models reformulating zone heat balance equations. The hybrid modeling approach is innovative and scientific, which combines the physics-based modeling in EnergyPlus with easy-to-obtain measured space air temperature data. Part II of this paper provides details of the validation of the developed hybrid modeling for internal thermal mass and infiltration using the controlled experiment using LBNL’s FLEXLAB. It is anticipated that the hybrid modeling in EnergyPlus enables improving the reliability of simulated result for existing buildings.
Acknowledgement 
This work was supported by the Assistant Secretary for Energy Efficiency and Renewable Energy, Building Technologies Office, of the U.S. Department of Energy under Contract No. DE-AC02-05CH11231. The authors wish to recognize Amir Roth, Technology Manager of the Building Technologies Office of the US Department of Energy, for his support and assistance in this work.

Reference
[1]	Navigant, Energy Efficiency Retrofits for Commercial and Public Buildings, (2014). http://www.navigantresearch.com/research/energy-efficiency-retrofits-for-commercial-and-public-buildings.
[2]	T. Hong, L. Yang, D. Hill, W. Feng, Data and Analytics to Inform Energy Retrofit of High Performance Buildings, Applied Energy. 126 (2014) 90–106.
[3]	S.H. Lee, T. Hong, M.A. Piette, G. Sawaya, Y. Chen, S.C. Taylor-Lange, Accelerating the energy retrofit of commercial buildings using a database of energy efficiency performance, Energy. 90 (2015) 738–747. doi:10.1016/j.energy.2015.07.107.
[4]	T. Hong, M.A. Piette, Y. Chen, S.H. Lee, S.C. Taylor-Lange, R. Zhang, K. Sun, P. Price, Commercial Building Energy Saver: An energy retrofit analysis toolkit, Applied Energy. 159 (2015) 298–309. doi:10.1016/j.apenergy.2015.09.002.
[5]	Y. Heo, Bayesian Calibration of Building Energy Models for Energy Retrofit Decision-making Under Uncertainty, Georgia Institute of Technology, 2011.
[6]	F. Zhao, Agent-based Modeling of Commercial Building Stocks for Energy Policy and Demand Response Analysis, Georgia Institute of Technology, 2012.
[7]	P. Raftery, M. Keane, J. O’Donnell, Calibrating whole building energy models: An evidence-based methodology, Energy and Buildings. 43 (2011) 2356–2364. doi:10.1016/j.enbuild.2011.05.020.
[8]	VisualDOE, VisualDOE 4.0 User Manual, (2004).
[9]	Y. Heo, R. Choudhary, G. Augenbroe, Calibration of building energy models for retrofit analysis under uncertainty, Energy and Buildings. 47 (2012) 550–560.
[10]	M. Manfren, N. Aste, R. Moshksar, Calibration and uncertainty analysis for computer models – A meta-model based approach for integrated building energy simulation, Applied Energy. 103 (2013) 627–641.
[11]	Z. O’Neill, B. Eisenhower, Leveraging the analysis of parametric uncertainty for building energy model calibration, Building Simulation. 6 (2013) 365–377.
[12]	K. Sun, T. Hong, S.C. Taylor-Lange, M.A. Piette, A pattern-based automated approach to building energy model calibration, Applied Energy. 165 (2016) 214–224. doi:10.1016/j.apenergy.2015.12.026.
[13]	ASHRAE, Ashrae Handbook: Fundamentals 2013, 2013.
[14]	Y. Zhang, Z. O’Neill, B. Dong, G. Augenbroe, Comparisons of inverse modeling approaches for predicting building energy performance, Building and Environment. 86 (2015) 177–190. doi:10.1016/j.buildenv.2014.12.023.
[15]	M.P. Vázquez, Y. Yang, H. Lin, Implementation of an energy model using inverse simulation procedure with building energy simulation tools, in: The 13th Asia Pacific Conference on the Built Environment, Hong Kong, 2015.
[16]	T.A. Reddy, I. Maor, S. Jian, Procedures for Reconciling Computer-Calculated Results With Measured Energy Data: ASHRAE Research Project 1051- RP, 2006.
[17]	J.E. Braun, N. Chaturvedi, An Inverse Gray-Box Model for Transient Building Load Prediction, HVAC&R Research. 8 (2002) 73–99.
[18]	F. Zhao, S.H. Lee, G. Augenbroe, Reconstructing building stock to replicate energy consumption data, Energy and Buildings. 117 (2015) 301–312. doi:10.1016/j.enbuild.2015.10.001.
[19]	ISO, ISO 13790: 2008 Energy performance of buildings —Calculation of energy use for spaceheating and cooling, (2008).
[20]	W.J. Cole, K.M. Powell, E.T. Hale, T.F. Edgar, Reduced-order residential home modeling for model predictive control, Energy and Buildings. 74 (2014) 69–77. doi:10.1016/j.enbuild.2014.01.033.
[21]	G. Kokogiannakis, P. Strachan, J. Clarke, Comparison of the simplified methods of the ISO 13790 standard and detailed modelling programs in a regulatory context, Journal of Building Performance Simulation. 1 (2008) 209–219.
[22]	J.A. Clarke, Energy Simulation in Building Design 2nd Edition, Butterworth-Heinemann, 2001.
[23]	B.D. Hunn, Fundamentals of Building Energy Dynamics, MIT Press, 1996.
[24]	DOE, EnergyPlus Engineering Reference: The Reference to EnergyPlus Calculations, 2015.
[25]	T. Hong, S.. Chou, T.. Bong, Building simulation: an overview of developments and information sources, Building and Environment. 35 (2000) 347–361. doi:10.1016/S0360-1323(99)00023-2.
[26]	G. Augenbroe, Trends in building simulation, Building and Environment. 37 (2002) 891–902. doi:10.1016/S0360-1323(02)00041-0.
[27]	H.X. Zhao, F. Magoulès, A review on the prediction of building energy consumption, Renewable and Sustainable Energy Reviews. 16 (2012) 3586–3592. doi:10.1016/j.rser.2012.02.049.
[28]	D.B. Crawley, J.W. Hand, M. Kummert, B.T. Griffith, Contrasting the capabilities of building energy performance simulation programs, Building and Environment. 43 (2008) 661–673. doi:10.1016/j.buildenv.2006.10.027.
[29]	K. Gowri, D. Winiarski, R. Jarnagin, PNNL-18898: Infiltration Modeling Guidelines for Commercial Building Energy Analysis, 2009.
[30]	Y. Zhang, Y. Zhang, W. Shi, R. Shang, R. Cheng, X. Wang, A new approach, based on the inverse problem and variation method, for solving building energy and environment problems: Preliminary study and illustrative examples, Building and Environment. 91 (2015) 204–218. doi:10.1016/j.buildenv.2015.02.016.
[31]	Swiss Federal Institute of Technology in Zurich, Inverse Modeling, (2015). http://www.up.ethz.ch/Research_Tools/Inverse_Modeling (accessed January 1, 2015).
[32]	C. a. Balaras, The role of thermal mass on the cooling load of buildings. An overview of computational methods, Energy and Buildings. 24 (1996) 1–10. doi:10.1016/0378-7788(95)00956-6.
[33]	P. Xu, L. Zagreus, Demand Shifting with Thermal Mass in Light and Heavy Mass Commercial Buildings, 2009 ASHRAE Annual Conference. (2010).
[34]	J. Karlsson, Possibilities of using thermal mass in buildings to save energy, cut power consumption peaks and increase the thermal comfort, 2012.
[35]	I. Building, P. Simulation, IMPACT OF THERMAL MASS ON SUMMER COMFORT IN BUILDING :, (n.d.) 2312–2319.
[36]	P. Ma, N. Guo, Modeling of Thermal Mass in a Small Commercial Building and Potential Improvement by Applying TABS, American Journal of Mechanical Engineering. 3 (2015) 55–62. doi:10.12691/ajme-3-2-4.
[37]	R. Zeng, X. Wang, H. Di, F. Jiang, Y. Zhang, New concepts and approach for developing energy efficient buildings: Ideal specific heat for building internal thermal mass, Energy and Buildings. 43 (2011) 1081–1090. doi:10.1016/j.enbuild.2010.08.035.
[38]	S. Wang, X. Xu, Parameter estimation of internal thermal mass of building dynamic models using genetic algorithm, Energy Conversion and Management. 47 (2006) 1927–1941.
[39]	K. Lee, J.E. Braun, Model-based demand-limiting control of building thermal mass, Building and Environment. 43 (2008) 1633–1646.
[40]	M. Deru, K. Field, D. Studer, K. Benne, B. Griffith, P. Torcellini, B. Liu, M. Halverson, D. Winiarski, M. Rosenberg, M. Yazdanian, J. Huang, D. Crawley, U . S . Department of Energy Commercial Reference Building Models of the National Building Stock, 2011.
[41]	DOE, EnergyPlus, (2016). https://energyplus.net/ (accessed May 18, 2016).
[42]	U.S. Department of Energy, Getting Started with EnergyPlus, Basic Concepts Manual-Essential Information You Need about Running EnergyPlus, The Board of Trustees of the University of Illinois and the Regents of the University of California through the Ernest Orlando Lawrence Berkele, 2013.
[43]	ASHRAE, ASHRAE Guideline 14-2002: Measurement of Energy and Demand Savings, 2002.
[44]	J. Jokisalo, J. Kurnitski, M. Korpi, T. Kalamees, J. Vinha, Building leakage, infiltration, and energy performance analyses for Finnish detached houses, Building and Environment. 44 (2009) 377–387. doi:10.1016/j.buildenv.2008.03.014.
[45]	I.S. Walker, D.J. Wilson, Field validation of equations for stack and wind driven air infiltration calculations, ASHRAE HVAC&R Research Journal. 4 (1998) 119–139.
[46]	LBNL, FLEXLAB The World’s Most Advanced Building Efficiency Test Bed, (2016). https://flexlab.lbl.gov/.

