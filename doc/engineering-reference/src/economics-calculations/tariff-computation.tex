\section{Tariff Computation}\label{tariff-computation}

The EnergyPlus economic (Utility Costs) objects related to computing utility bills include:

\begin{itemize}
\item
  UtilityCost:Tariff
\item
  UtilityCost:Qualify
\item
  UtilityCost:Charge:Simple
\item
  UtilityCost:Charge:Block
\item
  UtilityCost:Ratchet
\item
  UtilityCost:Variable
\item
  UtilityCost:Computation
\end{itemize}

This section builds upon the discussion that appears in the Input Output Reference under the heading ``EnergyPlus Economics.''~ The actual computation of monthly utility bills is not difficult since it is mostly consists of multiplying energy consumptions or demands by the price on a per unit basis and adding different bill components.~ The implementation in EnergyPlus becomes more complex since the objects were crafted to allow a great deal of~ flexibility in specifying a utility tariff while, at the same time, being as simple as possible.

The following discussion on variables and hierarchies is based on the text that appears in the Input Output Reference.

\subsection{Conceptual Framework -- Variables and Hierarchy}\label{conceptual-framework-variables-and-hierarchy}

To understand how to use the utility bill calculation portion of EnergyPlus you first need to understand some important concepts of variables and hierarchy.~ A variable, for the purposes of this section, is simply a named holder of a series of numbers. In most cases, the variable will be a named holder of 12 numbers, one number for each monthly utility bill. Here is a visualization of a variable called Electric Energy Use:

\begin{longtable}[c]{@{}ll@{}}
\toprule 
Month & Electric Energy Use \tabularnewline \midrule
\endhead
January & 12143 \tabularnewline
February & 13454 \tabularnewline
March & 14178 \tabularnewline
April & 14876 \tabularnewline
May & 15343 \tabularnewline
June & 16172 \tabularnewline
July & 16105 \tabularnewline
August & 15762 \tabularnewline
September & 14543 \tabularnewline
October & 13987 \tabularnewline
November & 13287 \tabularnewline
December & 12403 \tabularnewline
\bottomrule
\end{longtable}

In addition, the hierarchy shown in the first diagram in this section also represents dependencies that are included when determining the order of computation.

The resulting order of computation is shown at the bottom of the economics report.

\subsection{Computation Steps}\label{computation-steps}

Once the order that the formulas should be computed is known, the actual evaluation of the formulas is based on a simple Last In First Out (LIFO) stack.~ This is a common method to compute expressions where values are stored on the stack and operands work off of the top of the stack.
