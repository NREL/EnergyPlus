\section{Undisturbed Ground Temperature Model: Kusuda-Achenbach}\label{undisturbed-ground-temperature-model-kusuda-achenbach}

\subsubsection{Approach}\label{approach-004}

The model uses the correlation developed by Kusuda \& Achenbach, 1965 which is shown below. The average soil surface temperature, amplitude of the soil surface temperature, and day of minimum surface temperature are all required inputs. These can be determined by using the CalcSoilSurfTemp preprocessor. They can also automatically be generated by leaving the parameters blank and including a Site:GroundTemperature:Shallow object in the input file.

\begin{equation}
T(z,t) = \bar{T}_{s} - \Delta\bar{T}_{s} \cdot e^{-z \cdot \sqrt{\frac{\pi}{\alpha\tau}}} \cdot cos\left( \frac{2\pi t}{\alpha t} - \theta \right)
\end{equation}

Where:

\(T(z,t)\) is the undisturbed ground temperature as a function of time and depth

\(\bar{T}_{s}\) is the average annual soil surface temperature, in deg C

\(\Delta\bar{T}_{s}\) is the amplitude of the soil temperature change throughout the year, in deg C

\(\theta\) is the phase shift, or day of minimum surface temperature

\(\alpha\) is the themal diffusivity of the ground

\(\tau\) is time constant, 365.

\subsubsection{References}\label{references-049}

Kusuda, T. and P.R. Achenbach. 1965. `Earth Temperatures and Thermal Diffusivity at Selected Stations in the United States.' \emph{ASHRAE Transactions}. 71(1): 61-74.
