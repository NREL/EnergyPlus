\section{Use of Comma and Point in Numeric Output}\label{use-of-comma-and-point-in-numeric-output}

All EnergyPlus numeric output is written using the U.S. convention of a period or point ``.'' as the decimal separator. No thousands separator is used. For example, the numeric output for ``one thousand two hundred and one half'' would be 1200.5 in output files. The same conventions apply for EnergyPlus input files (idf), Exponent format (1.2005E+03) is also valid on input but is not used in output files.

Commas are used to separate values or fields in EnergyPlus input and output. They should not be used as part of any numeric value, not as a decimal separator and not as a thousands separator. This can cause problems for users in regions of the world which normally use comma as the decimal separator. This is especially important when viewing EnergyPlusvariables (*.csv) and meters (*Meter.csv) output files. Typically csv output files are viewed in a spreadsheet program, such as Excel. ``csv'' stands for ``comma separated values'', so the spreadsheet software needs to recognize comma as a list separator, not a decimal or thousands separator. If the values from a csv file appear to be nonsense when displayed in a spreadsheet program, this may be the source of the problem. Change the decimal separator to be ``.''~ in your system settings or in the spreadsheet program settings.
