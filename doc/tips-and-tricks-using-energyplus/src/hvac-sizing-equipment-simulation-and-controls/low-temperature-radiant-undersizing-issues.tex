\section{Low Temperature Radiant Undersizing Issues}\label{low-temperature-radiant-undersizing-issues}

Some users have noted difficulties when trying to size certain aspects of low temperature radiant systems, particulary in cooling mode for the hydronic (variable flow) and constant flow low temperature radiant systems when using autosize.  The problem appears to be that the system is not sizing properly or is undersizing, leaving zone conditions that are no where near the setpoint temperatures established by the radiant system.  This is confusing because the nature of autosizing should take some of the guesswork out of establishing certain parameters that are not always obvious and autosizing does not have issues with conventional forced air systems.

The problem stems from the nature of radiant systems and how they meet the load or reach the thermostatic condition set by the user.  A conventional forced air system delivers air to a space at a particular temperature and humidity condition.  In order to provide more heating or cooling, the flow rate is simply increased until the thermal load of the zone is met.  A low temperature radiant system is not as simple.  Increasing the water flow rate to the system does not necessarily increase the load that the system can meet.  In fact, in many situations, increasing the water flow rate in a hydronic system past a certain point does not provide any additional heating or cooling.  All such continued increases in flow rates provide are temperatures in the slab at the hydronic tubes getting closer and closer to the plant loop water temperature.  So, for a given chilled water temperature, for example, the system maximum capacity is limited in a sense and increased flow beyond a certain point will not produce any additional cooling like in a forced air system.  It is possible that the same problem could be encountered in heating mode as well.  If the hot water loop temperature is too low, then the system might not have enough theoretical capacity to meet whatever load is present depending upon the conditions of the zone and its physical characteristics.

So, given this limitation, the question becomes: how does one arrive at a solution that allows one to use a radiant system and still use autosizing?  The answer is that it will require some iteration, just as achieving thermal comfort using a low temperature radiant system and its setpoint temperatures requires some iteration.  For example, when coming up with the proper setpoint temperatures for a radiant system that will provide neutral comfort based on a thermal comfort model, one generally will have to try different setpoint temperatures to achieve acceptable comfort.  In the same way, different hot or chilled water temperatures may need to be tried to find the proper value that achieves the right capacity for the radiant system to meet the thermal loads of the zone.

Below are several steps or suggestions that can be followed to provide better success when autosizing low temperature radiant systems that meet the proper comfort conditions within those zones.

\subsection{Turn Off Condensation Controls}\label{turn-off-condensation-controls}

In many cooling situations, low temperature radiant systems may result in a radiant system surface temperature that drops below the dew-point temperature in the space.  When this happens, if the user has selected one of the condensation control methods, the radiant system could throttle back or turn off completely, leaving the zone without cooling.  Over time, this can lead to a situation where the zone temperature builds up to a point where the radiant system simply cannot catch up.  So, to avoid such a situation while attempting to size the system, it is recommended that the user turn the condensation controls to OFF.  While this is not realistic for an actual system since it may result in condensation on the surface and the energy associated with such condensation is not handled by EnergyPlus, having the system turning off unexpectedly will complicate the autosizing iterations further.  It is better to size the system first and then work out the issue of condensation.

\subsection{Adjust Chilled Water Loop Temperature}\label{adjust-chilled-water-loop-temperature}

Once the low temperature system is no longer turning off due to condensation and the user feels fairly confident that the radiant system setpoint temperatures for the zone will provide reasonable comfort, run the input file and check to see whether or not the temperatures are achieved and also potentially whether the thermal comfort criteria has been met (using one of the thermal comfort models available in EnergyPlus).  If it is NOT being met, try adjusting the water temperature of the water loop serving the radiant system.  Keep in mind that you may need to adjust this in a variety of places in the input file to get everything to agree.  This includes, but is not limited to, the low temperature radiant system water temperature setpoints, the setpoint temperature of the plant supply loop serving the radiant system, temperatures in the plant sizing input, and any limit temperatures for the individual supply side equipment.  If the user does not adjust all of these, the loop temperature may not change as anticipated.  The adjustment process may be somewhat iterative as it is not possible to predict exactly how much higher or lower the loop temperature needs to be to meet the load.  A second run will often give the user an idea of how much a change in loop temperature will impact the zone temperature and can use this information to interpolate/extrapolate for a new loop temperature guess.
When the loop temperature has been adjusted to a value that results in zone temperatures that are acceptable or meet the temperature setpoints, the user must then evaluate whether or not these temperatures are realistic.  In cooling, lower water temperatures are associated with higher chiller energy consumption and thus may not be desirable.  In low energy passive heating systems, high temperatures may not be achievable without auxiliary heating equipment.  Thus, the user needs to consider the implications of the loop temperatures needed to meet the loads.

\subsection{Analyze Zone Characteristics}\label{analyze-zone-characteristics}

Finally, another thing to consider is the zone itself.  Is a low temperature radiant system appropriate for the zone it is serving?  Low temperature radiant systems can be excellent choices for many situations.  However some situations like high internal gains or other high load situations may result in a space where the radiant system simply does not have enough area and temperature difference to provide adequate heating or cooling to meet the thermal needs of the zone.  The user should critically evaluate all aspects of the zone including the physical characteristics of the zone, constructions, windows, internal gains from people/lights/equipment, etc.  While a forced air system can provide "unlimited" conditioning by simply increasing the flow rate of air to the space, a radiant system cannot do this as has been discussed above.

