\section{Recurring Errors}\label{recurring-errors}

The recurring error category is employed during the actual simulation periods. Usually, a heading message will appear:

\begin{lstlisting}
   ** Warning ** Coil:Cooling:DX:SingleSpeed "DXCOOLINGCOIL_SOUTHZONE_2NDFLOOR" - Full load outlet air dry-bulb temperature < 2C. This indicates the possibility of coil frost/freeze. Outlet temperature = -4.60 C.
   **   ~~~   **  ...Occurrence info = Washington Dc Dulles IntL Ar VA USA TMY3 WMO# = 724030, 01/02 06:01 - 06:02
   **   ~~~   ** ... Possible reasons for low outlet air dry-bulb temperatures are: This DX coil
   **   ~~~   **    1) may have a low inlet air dry-bulb temperature. Inlet air temperature = 9.778 C.
   **   ~~~   **    2) may have a low air flow rate per watt of cooling capacity. Check inputs.
   **   ~~~   **    3) is used as part of a HX assisted cooling coil which uses a high sensible effectiveness. Check inputs.
\end{lstlisting}

This message contains quite a bit of information: the basic object and name of the object, the context of the error, the time of the error as well as some reasons why this might have occurred.

At the end of the simulation, the summary appears:

\begin{lstlisting}
   *************  ** Warning ** Coil:Cooling:DX:SingleSpeed "DXCOOLINGCOIL_SOUTHZONE_2NDFLOOR" - Full load outlet temperature indicates a possibility of frost/freeze error continues. Outlet air temperature statistics follow:
   *************  **   ~~~   **   This error occurred 1240 total times;
   *************  **   ~~~   **   during Warmup 0 times;
   *************  **   ~~~   **   during Sizing 0 times.
   *************  **   ~~~   **   Max = 1.995912  Min = -4.60024
\end{lstlisting}

Here you see a summary of how many times the error occurred (1240) as well as how many times during Warmup (0) and how many times during Sizing (0). Plus a minimum (-4.6) and maximum (1.99) for the terms of the message.
