\section{Schedules And Availability Manager}\label{schedules-and-availability-manager}
Regarding component schedules, the general rule is don't schedule any components except the supply fan and the corresponding availability manager(s). Beyond that, every component should always be available and let the controls determine what runs or doesn't run. If a component other than the supply fan is scheduled off, then it will remain off even if the night cycle manager turns on the system.

For unitary systems, don't use the night cycle manager. Use a scheduled availability manager and let the system be always on. Then use the Supply Air Fan Operating Mode Schedule Name in the unitary system to switch between continuous fan (for ventilation) during occupied periods and switch to cycling fan for unoccupied. The system will cycle on as the thermostat requests, and this way it will run just enough to meet the load - no need for a minimum cycle time.