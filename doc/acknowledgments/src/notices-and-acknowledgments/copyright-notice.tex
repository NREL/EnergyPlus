\section{Copyright Notice}\label{copyright-notice}

Copyright (c) 1996-2015 The Board of Trustees of the University of Illinois and The Regents of the University of California through Ernest Orlando Lawrence Berkeley National Laboratory. All rights reserved.

Portions of the EnergyPlus\textsuperscript{TM} software package have been developed and copyrighted by other individuals, companies and institutions. These portions have been incorporated into the EnergyPlus software package under license.

\textbf{In addition to the primary authorship of the LBNL Simulation Research Group (}\href{http://simulationresearch.lbl.gov/}{\textbf{http://simulationresearch.lbl.gov/}}\textbf{) and the UIUC Building Systems Laboratory, the following have contributed to EnergyPlus Versions (includes all minor revisions):}

Portions of the EnergyPlus weather processor were developed by US Department of Energy, Office of Building Technologies.

Portions of the input processing, output processing, weather processor, BLAST Translator were developed by US Army Corps of Engineers, Construction Engineering Research Laboratories, 2902 Newmark Drive, Champaign IL~ 61821. \url{http://www.erdc.usace.army.mil/Locations/ConstructionEngineeringResearchLaboratory.aspx}

Portions of this software package were developed for Ernest Orlando Lawrence Berkeley National Laboratory and Florida Solar Energy Center by Linda Lawrie of DHL Consulting.

Portions of this software package were developed for Ernest Orlando Lawrence Berkeley National Laboratory and Florida Solar Energy Center by C.O. Pedersen Associates.

Portions of the EnergyPlus utility software (EP-Launch, IDFEditor, DOE2Translator, HVAC-Diagram, ExpandObjects, CSVProc, ParametricPreprocessor, AppGPostProcess, EP-Compare, and convertESOMTR) were developed by GARD Analytics, Inc. 115 S. Wilke Road, Suite 105, Arlington Heights, IL, USA, \textless{}info@gard.com\textgreater{}, \href{http://www.gard.com/}{www.gard.com}. GARD Analytics performed independent verification and validation testing of the software after developing the testing strategy and plan. GARD Analytics was also responsible for gas absorption chiller, desiccant dehumidifier, ice storage (simple), table reports, economics and life cycle costing.

Portions of flow resolver, chiller models (absorption, electric, const cop, engine-driven, gas-turbine), generator models (diesel electric, gas turbine), furnace models, heat recovery loop, plant loop, plant condenser loop, air-change dependent inside film coefficients were developed by Oklahoma State University, 110 Engineering North, Stillwater, OK 74078.

Portions of EnergyPlus related to the models for EMPD moisture calculations, DX coils, furnace/unitary systems, air-to-air heat pumps, changeover-bypass VAV systems, packaged terminal heat pumps, cooling towers, AirflowNetwork, refrigerated cases, reformulated and electric EIR chillers, desuperheater air and water heating coils, heat pump water heaters, desiccant and generic air-to-air heat exchangers, window screens, and thermal comfort controls were developed by University of Central Florida, Florida Solar Energy Center (FSEC), 1679 Clearlake Road, Cocoa, FL~ 32922, \href{http://www.fsec.ucf.edu/}{www.fsec.ucf.edu/}.

Portions of the refrigeration model and the exhaust-fired absorption chiller model were developed by Oak Ridge National Laboratory, Bethel Valley Road, Oak Ridge, Tennessee 37831.

Portions of EnergyPlus were developed by the National Renewable Energy Laboratory (NREL), 1617 Cole Blvd, Golden, CO 80401.

Portions of EnergyPlus related to transformer losses model, autosizing calculations, life cycle costing and chemical battery storage model were developed by Pacific Northwest National Laboratory (PNNL), P.O. Box 999, Richland, WA 99352.

EnergyPlus v1.0.1, v1.0.2, v1.0.3, v1.1, v1.1.1 (Wintel platform) included a link to TRNSYS (The Transient Energy System Simulation Tool) for photovoltaic calculations developed by Thermal Energy System Specialists, 2916 Marketplace Drive, Suite 104, Madison, WI 53719; Tel: (608) 274-2577. EnergyPlus v1.2 and later includes Photovoltaic calculations implemented in EnergyPlus by Thermal Energy System Specialists. This model was originally developed by Oystein Ulleberg, Institute for Energy Technology, Norway -- based on the Duffie and Beckman equivalent one-diode model.

Portions of this software package that convert certain stand-alone heat transfer models for slab-on-grade and basement foundations were developed by William Bahnfleth, Cynthia Cogil, and Edward Clements, Department of Architectural Engineering, Pennsylvania State University, 224 Engineering Unit A, University Park, Pennsylvania 16802-1416, (814) 863-2076.

The concept and initial implementation for the EnergyPlus COM/DLL version (Wintel platform) was made possible through cooperation with DesignBuilder Software, Ltd, Andy Tindale -- an EnergyPlus collaborative developer.

The thickness, conductivity, density and specific heat values of the material layers for the constructions in the Composite Wall Construction reference data set have been taken from the ASHRAE report ``Modeling Two- and Three-Dimensional Heat Transfer through Composite Wall and Roof Assemblies in Hourly Energy Simulation Programs (1145-TRP),'' by Enermodal Engineering Limited, Oak Ridge National Laboratory, and the Polish Academy of Sciences, January 2001.

EnergyPlus v1.2 and later versions contains DELight2, a simulation engine for daylighting and electric lighting system analysis developed at Ernest Orlando Lawrence Berkeley National Laboratory.

EnergyPlus v1.2.2 through v3.1 contained links to SPARK, a simulation engine for detailed system modeling developed at Ernest Orlando Lawrence Berkeley National Laboratory in conjunction with Ayres Sowell Associates, Inc.

The airflow calculation portion of the EnergyPlus AirflowNetwork model was based on AIRNET written by George Walton of the National Institute for Standards and Technology (NIST), 100 Bureau Drive, Gaithersburg, MD 20899. The EnergyPlus AirflowNetwork model also includes portions of stack effect and detailed large opening from an early version of COMIS (Conjunction Of Multizone Infiltration Specialists) developed by a multinational, multi-institutional effort under the auspices of the International Energy Agency's Buildings and Community Systems Agreement working group focusing on multizone air flow modeling (Annex 23) and now administered by the Swiss Federal Laboratories for Materials Testing and Research (EMPA), Division 175, Uberlandstrasse 129, CH-8600 Dubendorf, Switzerland.

The EnergyPlus model for displacement ventilation and cross-ventilation (version v1.2 and later) was developed by Guilherme Carrilho da Graca (Department of Mechanical and Aerospace Engineering, University of California, San Diego and NaturalWorks) and Paul Linden (Department of Mechanical and Aerospace Engineering, University of California, San Diego).

The EnergyPlus models for UFAD served zones were developed by Anna Liu and Paul Linden at the Department of Mechanical and Aerospace Engineering, University of California, San Diego.

ASHRAE research project 1254-RP supported the development of the following features first added in EnergyPlus v1.2.2: 
\begin{itemize}
  \item DXSystem:AirLoop enhancements (valid as OA system equipment, new humidity control options)
  \item New set point managers: SET POINT MANAGER:SINGLE ZONE HEATING, SET POINT MANAGER:SINGLE ZONE COOLING, and SET POINT MANAGER:OUTSIDE AIR PRETREAT
  \item New 2-stage DX coil with enhanced dehumidification option:
  \begin{itemize}
    \item COIL:DX:MultiMode:CoolingEmpirical
  \end{itemize}
  \item Additional DESICCANT DEHUMIDIFIER:SOLID setpoint control option
\end{itemize}

American Society of Heating Refrigerating and Air-Conditioning Engineers, Inc., 1791 Tullie Circle, N.E., Atlanta, GA 30329\footnote{\url{http://www.ashrae.org/}}. Work performed by GARD Analytics, Inc., 115 S. Wilke Road, Suite 105, Arlington Heights, IL, USA\footnote{url{info@gard.com}; \url{http://www.gard.com/}}, November 2004. These items were renamed in V3.0 to:
\begin{itemize}
  \item SetpointManager:SingleZone:Heating
  \item SetpointManager:SingleZone:Cooling
  \item SetpointManager:OutdoorAirPretreat
  \item Coil:Cooling:DX:TwoStageWithHumidityControlMode
  \item Dehumidifier:Desiccant:NoFans
\end{itemize}

The Ecoroof (Green Roof) model, first introduced in EnergyPlus v2.0, was developed at Portland State University, by David Sailor and his students. It is based on the FASST vegetation models developed by Frankenstein and Koenig for the US Army Corps of Engineers.

The HAMT (Heat And Moisture Transfer) model, first introduced in EnergyPlus v3.0.0 was developed by Phillip Biddulph, Complex Built Environment Systems, The Bartlett School of Graduate Studies, University College London, Gower Street, London WC1E 6BT, United Kingdom. \url{http://www.cbes.ucl.ac.uk/}.

The SQLite output module, first introduced in EnergyPlus v3.0.0, was developed by Gregory B. Stark, P.E., Building Synergies, LLC, 1860 Washington Street, Suite 208, Denver, Colorado 80203, United States.

Refrigeration compressor performance data and refrigeration practices were provided by CDH Energy, Cazenovia, NY 12035.

The external interface was developed by Michael Wetter and Philip Haves (Lawrence Berkeley National Laboratory) and by Rui Zhang (Carnegie Mellon University). An earlier upgrade to a development version of EnergyPlus 3.0 was implemented by Charles Corbin, Anthony Florita, Gregor Henze and Peter May-Ostendorp (University of Colorado at Boulder).

Various suggestions for time reduction, improved documentation and other items have been incorporated from Autodesk, Inc., Bentley Systems, and others.

Particular recognition goes to Noel Keen (LBNL Computational Research Division) and Geof Sawaya (Oak Ridge National Laboratory fellow) who have done extensive profiling and creation of time reduction features that have gone into the code.

Second Law modified the WaterToAirHeatPump:EquationFit module to include the variable ``WaterCyclingMode''. This variable determines whether the heat pump water flow is constant, whether it cycles with the compressor, or whether it is constant when the heat pump is active. WaterFlowMode is set by the HVAC wrapper object; either ZoneHVAC:WaterToAirHeatPump or AirLoopHVAC:UnitaryHeatPump:WaterToAir. Second Law, Burlington, VT, Karen Walkerman.

\textbf{NOTICE:} The U.S. Government is granted for itself and others acting on its behalf a paid-up, nonexclusive, irrevocable, worldwide license in this data to reproduce, prepare derivative works, and perform publicly and display publicly. Beginning five (5) years after permission to assert copyright is granted, subject to two possible five year renewals, the U.S. Government is granted for itself and others acting on its behalf a paid-up, non-exclusive, irrevocable worldwide license in this data to reproduce, prepare derivative works, distribute copies to the public, perform publicly and display publicly, and to permit others to do so.

\textbf{TRADEMARKS:} EnergyPlus is a trademark of the US Department of Energy.
