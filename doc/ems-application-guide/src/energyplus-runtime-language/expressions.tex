\section{Expressions}\label{expressions}

An expression is a sequence of variables and/or constants linked together by operators. An expression is always evaluated to a single value.

The rules for expressions are:

\begin{itemize}
\item
  An expression is a sequence of variables and/or constants linked by operators.
\item
  Expressions always evaluate to a single value.
\item
  Comparison operators evaluate to 1.0 for ``true'' or 0.0 for ``false.''
\item
  Compound expressions are allowed and can be organized with parentheses.
\item
  Variables must be initialized prior to being used in an expression.
\end{itemize}

The operators shown in Table~\ref{table:operators-for-erl} are available for use in Erl programs.

% table 3
\begin{longtable}[c]{p{1.5in}p{1.5in}p{1.5in}p{1.5in}}
\caption{Operators for Erl \label{table:operators-for-erl}} \tabularnewline
\toprule 
Operator Symbol & Description & Evaluation Order & Example \tabularnewline
\midrule
\endfirsthead

\caption[]{Operators for Erl} \tabularnewline
\toprule 
Operator Symbol & Description & Evaluation Order & Example \tabularnewline
\midrule
\endhead

( ) & Parentheses & left-to-right & SET z = 23/(3 + 2) \tabularnewline
+ & Addition & right-to-left & SET a = 4 + 5 \tabularnewline
- & Subtraction & right-to-left & SET b = a - 3 \tabularnewline
* & Multiplication & right-to-left & SET c = a * b \tabularnewline
/ & Division & left-to-right & SET d = b/a \tabularnewline
\^{} & Raise to a power & left-to-right & SET e = c \^{} 0.5 \tabularnewline
== & Equality comparison & left-to-right & IF a == b \tabularnewline
< > ~ & Inequality comparison & left-to-right & IF c  < >  d \tabularnewline
> ~ & Greater than comparison & left-to-right & IF a  >  e \tabularnewline
> = & Greater than or equal to comparison & left-to-right & IF a  > = 6 \tabularnewline
< ~ & Less than comparison & left-to-right & IF b  <  2 \tabularnewline
< = & Less than or equal to comparison & left-to-right & IF b  < = f \tabularnewline
\& \& & Logical AND & right-to-left & IF c  \& \&  d \tabularnewline
|| & Logical OR & right-to-left & IF c || d \tabularnewline
\bottomrule
\end{longtable}

Because expressions can be evaluated to a single value, they can be used in SET and IF statements. That means both of the following instructions are allowed:

\begin{lstlisting}
SET a = c < d
IF a - 1
\end{lstlisting}

In the case of the SET example, the value of ``a'' is set to 1 if ``c'' is less than ``d''; otherwise, it is set to 0. For the IF example, the IF block of instructions are executed if a -- 1 is greater than zero.

Compound expressions allow multiple operators to be sequenced or nested. For example:

\begin{lstlisting}
a + b \* 7 / 4.5
(a \* 3 + 4) ^ 2
(a > b) && (c < d)
\end{lstlisting}

For complicated expressions, it helps to make heavy use of parentheses in your equations.~ By using parentheses with proper algebraic evaluation in mind to group terms, you can help the Erl parser.~ The language processor is simplistic compared to a full-blown programming language and sometimes has problems applying the rules of algebra.~ It is safer to err on the side of extra parentheses and to inspect the results of complex expressions in the EDD output.
