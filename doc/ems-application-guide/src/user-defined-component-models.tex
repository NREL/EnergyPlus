\chapter{User-Defined Component Models}\label{user-defined-component-models}

This section provides an overview of how you can use EMS to create your own custom models for HVAC and plant equipment.~ EMS can be used not only for controls and overriding the behavior of existing models, but also to implement entirely new component models of your own formulation.~ Such user-defined component models are implemented by writing Erl programs, setting up internal variables, sensors, actuators and output variables that work in conjunction with a set of special input objects in the group called ``User Defined HVAC and Plant Component Models.''

This system provides a means of modeling new types of equipment that do not yet have models implemented in EnergyPlus.~ The capability to add new custom models should have a wide variety of creative applications such as evaluating the annual energy performance implications of new types of equipment and providing a mechanism for including ``exceptional calculation methods'' in your EnergyPlus models.

This section first introduces common characteristics of the user-defined component models and then goes into more detail on each of available component modeling shells that are used to connect user-defined models and algorithms to the rest of EnergyPlus' HVAC and plant simulations.
