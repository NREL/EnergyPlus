\section{ERR File}\label{err-file}

A key output file to review is the ERR file (eplusout.err), the one with the ``.err'' file extension. This is the common error file for all of EnergyPlus, and many EMS-related errors will appear there. The file might contain critical problems that arose while the Erl programs were being read in and processed. Although the EDD file will likely be the focus of most debugging, remember the ERR file. Also, sometimes no EDD file is produced from a run. This occurs when problems are captured early during input processing and the program fatals out before an Erl program is run. Depending on the run manager you use to execute EnergyPlus, the EDD file may be from a previous run, so check the file creation times for ERR and EDD.

An especially important error revealed in the ERR file is truncation from too long input. Each program line in Erl is limited to 100 characters. (It becomes useful to keep variable names shorter in Erl because the line length limit can be onerous.)~ If there are more than 100 characters, the program truncates the line to the first 100. This will often throw a severe error that halts because the truncated line is not a valid statement. But an unlucky truncation may form a viable line of code and the program will run. Truncation of any Erl program line is surely a bad thing, so it is important to check the ERR file.
