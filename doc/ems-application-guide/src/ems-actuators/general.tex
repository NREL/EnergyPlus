\section{General}\label{general}

\subsection{Schedules}\label{schedules}

A series of actuators is available for overriding schedule values. The following actuators are available with the control type called ``Schedule Value'':~ Schedule:Year, Schedule:Compact, Schedule:File, and Schedule:Constant. The units are not known by the schedule and are determined by the model that references the schedule. The unique identifier is the name of schedule.

If you try to use a particular schedule as input to calculations that modify that schedule, you will be in a circular situation with unexpected results. The modified schedule will lose the original information (unless the actuator is set to Null) and the modifications will be reapplied on top of previous modifications. When this situation arises, use a copy of the original schedule as input to the Erl program so you have the original schedule values.

\subsection{Curves}\label{curves}

An advanced actuator called ``Curve'' with a control type called ``Curve Result'' is available whenever any generic curve objects are used. This allows you to override the results generated by these curves. The units are not known by the actuator and depend on how the curve is being used by the component model that calls it.

This actuator must be used with caution. The EMS does not necessarily have access to the independent variables used by the models when the curves are evaluated during normal evaluation, so in most situations you will probably need to examine EnergyPlus source code to use this actuator correctly.

\subsection{Weather Data}\label{weather-data}

A series of actuators are available for overriding the values of weather data that are normally derived from the .epw weather file.~ These provide the ability to alter weather data and were originally requested for use with ExternalInterface for using measured data.~ The following can be overridden: outdoor drybulb, outdoor dewpoint, outdoor relative humidity, diffuse solar, direct solar, wind speed, and wind direction.
